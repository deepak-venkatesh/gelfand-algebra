\documentclass[10pt]{article}
\usepackage{amsmath}
\usepackage{hyperref}
\usepackage{listings}
\usepackage{longdivision}
\usepackage{booktabs}
\usepackage{polynom}
\usepackage{graphicx}
\usepackage{tikz}
\usepackage{float}
\usepackage{cancel}
\usepackage{xlop}
\usepackage{xcolor} % for colors
\lstdefinelanguage{Scheme}{
  keywords={define,lambda,if,cond,let,begin,car,cdr,cons,quote},
  sensitive=true,
  morecomment=[l]{;},
  morestring=[b]",
}

\lstset{
  language=Scheme,
  basicstyle=\ttfamily\small,
  keywordstyle=\color{blue}\bfseries,
  commentstyle=\color{gray},
  stringstyle=\color{orange},
  frame=none,
  columns=flexible,
  showstringspaces=false,
}

\setlength{\parindent}{0pt}
\setlength{\parskip}{0.5em}

% Add license
\usepackage{ccicons}
\usepackage[
    type={CC},
    modifier={by-sa},
    version={4.0},
]{doclicense}

\title {Solutions to \textit{Algebra}\\
[0.5em]\large by I.M. Gelfand \& A. Shen}

\author{Deepak Venkatesh}
\date{\today}


% --- Custom Problem Environment ---
\newcounter{problem}
\newenvironment{problem}[1][]{
  \refstepcounter{problem}
  \par\noindent\textbf{Problem~\theproblem. #1}\par
  \vspace{0.5em}
}{\vspace{1em}}



\begin{document}
\noindent
\maketitle
\begin{abstract}
\noindent
\textit{Algebra} by I.M.Gelfand and A.Shen, first published in September 1993, is a 150 page book
covering 72 topics related to school level algebra. The book presents 342 problems some with
solutions, and others without. This booklet aims to provide correct solutions to all
the 342 problems listed in \textit{Algebra}. Each solution is carefully checked, either by hand (particularly for proofs)
or programmatically using Scheme (a dialect of Lisp). LLMs have helped
me in drawing figures in \LaTeX. All errors are my own, please report
any issue \href{https://github.com/deepak-venkatesh/gcpm/issues}{here}.
\end{abstract}
\vspace{27em}
\doclicenseThis
\newpage
\tableofcontents
\newpage % start sections on a fresh page

%%%%%
%%%% Difficult Problems for Middle Schoolers:
%%%% 31, 32, 42, 43, 45, 90, 97, 101, 102, 103, 104, 105, 107 (c), 111, 116, 122, 159, 166
%%%%%

%%%%%%%%%% 1
\section{Introduction}
No problems
%%%%%%%%%%%%%%%%%%%%%%%%%%%%%%%%%%%%%%%%%
% Chapter End
%%%%%%%%%%%%%%%%%%%%%%%%%%%%%%%%%%%%%%%%%

%%%%%%%%%% 2
\section{Exchange of terms in addition}

No problems

%%%%%%%%%%%%%%%%%%%%%%%%%%%%%%%%%%%%%%%%%
% Chapter End
%%%%%%%%%%%%%%%%%%%%%%%%%%%%%%%%%%%%%%%%%

%%%%%%%%%% 3
\section{Exchange of terms in multiplication}

No problems

%%%%%%%%%%%%%%%%%%%%%%%%%%%%%%%%%%%%%%%%%
% Chapter End
%%%%%%%%%%%%%%%%%%%%%%%%%%%%%%%%%%%%%%%%%

%%%%%%%%%% 4
\section{Addition in the decimal number system}

\begin{problem} %1
Stack 8s knowing that $8\times5$ ends with a 0 (that is 40). This gives a carry over of 4.
So we need $4 + 8 + 8$ to get a number that ends with 0 for the tens place. This gives a carry
over of 2. This 2 can get added to 8 in the hundreds place. The tens place structure shown
below.

\begin{center}
\[
\begin{array}{r}
\cdots 8\\
\cdots 8\\
\cdots 8\\
+\,\cdots 8\\
\hline
\cdots 0
\end{array}
\]
\end{center}

\begin{center}
\[
\begin{array}{r}
888\\
088\\
008\\
+$ $008\\
\hline
1000
\end{array}
\]
\end{center}

Answer is self verifiable.

\end{problem}

\begin{problem} %2

\begin{center}
\[
\begin{array}{r}
AAA\\
+$ $BBB\\
\hline
AAAC
\end{array}
\]
\end{center}

The solution lies in the fact that in the answer the thousandth's place A has to be 1. This
is so because whenever there is a carry over the tens digit will be 1 in addition (in this structure).
At the maximum level it would be 9 + 9 = 18 for instance.\\
So we have A as 1.

\begin{center}
\[
\begin{array}{r}
111\\
+$ $BBB\\
\hline
111C
\end{array}
\]
\end{center}

Now for B it has to be 9 because if it was any other number then the answer could not have 1s
in the places it has now. If B was 0 then the thousandth place 1 in the answer would not
materialize. So we have now.

\begin{center}
\[
\begin{array}{r}
111\\
+$ $999\\
\hline
111C
\end{array}
\]
\end{center}

We can easily see that C is 0 now. So we have
\[A = 1\]
\[B = 9\]
\[C = 0\]

Answer is self verifiable.
\end{problem}

%%%%%%%%%%%%%%%%%%%%%%%%%%%%%%%%%%%%%%%%%
% Chapter End
%%%%%%%%%%%%%%%%%%%%%%%%%%%%%%%%%%%%%%%%%

%%%%%%%%%% 5
\section{The multiplication table and the multiplication algorithm}

\begin{problem} %3
This looks tricky but is fairly easy to understand the pattern once written down. 1001 multiplied by any 3 digit
number will be that 3 digit number repeating twice. This is so because the `001' in 1001 when multiplied by the
3 digit number gives itself and then the `1' in the thousandth's place in 1001 and gives the 3 digit number. It 
is like concatenation of a 3 digit number to itself when multiplied by 1001.

\begin{center}
\[
\begin{array}{r}
715\\
\times$ $001\\
\hline
715
\end{array}
\]
\end{center}

\begin{center}
\[
\begin{array}{r}
715\\
\times$ $1001\\
\hline
715\\
+$ $715000\\
\hline
715715
\end{array}
\]
\end{center}

Answer is self verifiable.

Answer is 715715.

\end{problem}

\begin{problem} %4
This is similar to the previous problem except that we have a 2 digit number getting multiplied by `01'. It will
still result in a concatenation.

Verified in Scheme
\begin{lstlisting}
> (* 101010101 57)
5757575757
\end{lstlisting}

Answer is 5757575757

\end{problem}

\begin{problem} %5
This is on the same lines as previous two problems.
\begin{center}
\[
\begin{array}{r}
1020304050\\
\times$ $10001\\
\hline
1020304050\\
+$ $10203040500000\\
\hline
10204060804050
\end{array}
\]
\end{center}

Verified in Scheme
\begin{lstlisting}
> (* 10001 1020304050)
10204060804050
\end{lstlisting}

Answer is 10204060804050

\end{problem}

\begin{problem} %6
This is a trick I have been teaching all kids.

To look at an easier version of the problem say we have to $11 * 11$. This is 121. Two 1s is getting
multiplied by two 1s (eleven in this case). So we have the mnemonic $1..2...then..reverse$. When we
make one of the numbers as three 1s that is $111 * 11$ then we repeat the center digit
$1..2..2...then..reverse$, the answer being 1221.

In this example we have 11111 multiplied by 1111. This should give us 12344321. Two 4s in center.

Let us look at a pattern in Scheme to verify.
\begin{lstlisting}
> (* 1111 1111)
1234321

> (* 11111 1111)
12344321

> (* 111111 1111)
123444321

> (* 1111111 1111)
1234444321
\end{lstlisting}

Answer is 12344321

\end{problem}

\begin{problem} %7
The solution is provided in the book. Its an easy problem where we use the last digits of the 3s
multiplication table.

\begin{center}
\[
\begin{array}{r}
1ABCDE\\
\times$ $3\\
\hline
ABCDE1\\
\end{array}
\]
\end{center}

The only was to get 1 as the answer when 3 is multiplied by $E$ is $3 * 7$. The carryover is 2. So now
we have $(3 * D) + 2$ which should end with E which we know is 7 now. So we get 3 times 5 plus 2 which ends
in 7. Therefore D is 5. We keep going till we reach the end at the final number.

This number is actually important since this is the number which repeats when we divide a number by 7. This number is starting with
1 and with only 0s thereafter. Next few problems have this trick involved.

Answer is verified.

Answer is 142857.

\end{problem}

%%%%%%%%%%%%%%%%%%%%%%%%%%%%%%%%%%%%%%%%%
% Chapter End
%%%%%%%%%%%%%%%%%%%%%%%%%%%%%%%%%%%%%%%%%

%%%%%%%%%% 6
\section{The division algorithm}
\begin{problem} %8
This is inverse of the previous chapter. 

\begin{center}
\longdivision{123123123}{123}
\end{center}

We can see the pattern of 001..001..001 in the answer.
Suppose we had 1234123412341234 divided by 1234.
What would we get? It would be 0001..0001..0001..0001

Verified in Scheme
\begin{center}
\begin{lstlisting}
> (/ 1234123412341234 1234)
1000100010001
> (/ 123123123 123)
1001001
\end{lstlisting}
\end{center}


Answer is 1001001

\end{problem}

\begin{problem} %9
We can simplify the problem here. We have 1111111 (seven 1s) which will divide a long series of 1s.
That means for every group of seven 1s the quotient will be 1. Since there are 100 1s we will
have 14 groups of seven 1s that makes it 98 1s. The last two 1s will be the remainder. So the remainder
is 11.

A smaller pattern here
\begin{center}
\longdivision{111111111}{1111111}
\end{center}

Answer is 11

\end{problem}

\begin{problem} %10
We get here the cyclical nature of the quotient when we divide by 7. example

\begin{center}
\longdivision{1000000}{7}
\end{center}

So 1000000 (7 digit number) when divided by 7 will give a recurring quotient with 142857.
Therefore when we divide 1000...0 (20 zeroes) we have 18 zeroes consumed with 3 times 142857 appearing in the
quotient. Then the last 2 zeroes will give the quotient of 14 and a remainder of 2. Thus the quotient
should be 14285714285714285714 and a remainder of 2.

Verified in Scheme
\begin{center}
\begin{lstlisting}
> (/ 100000000000000000000 7)
14285714285714285714 2/7
\end{lstlisting}
\end{center}

\end{problem}

\begin{problem} %11
As shown in previous problem number 10 this pattern of 142857 will repeat. This the cyclical number we get in this instance.

\end{problem}

\begin{problem} %12
This is fairly similar to the previous two problems. The only difference is that when we start with a
different number the cyclical pattern of division by 7 starts with a different digit, but the pattern holds.
Let us take the example of 2000000 divided by 7.

\begin{center}
\longdivision{2000000}{7}
\end{center}

We see the same pattern but it starts at 2. So the pattern is 285714.

So the answers are\\
2000...00(20 0s): Quotient is 8571428571428571428. Remainder is 4.\\
3000...00(20 0s): Quotient is 42857142857142857142. Remainder is 6.\\
4000...00(20 0s): Quotient is 57142857142857142857. Remainder is 1.\\
5000...00(20 0s): Quotient is 71428571428571428571. Remainder is 3.\\
6000...00(20 0s): Quotient is 85714285714285714285. Remainder is 5.\\

Verified in Scheme
\begin{center}
\begin{lstlisting}
> (/ 200000000000000000000 7)
28571428571428571428 4/7
> (/ 300000000000000000000 7)
42857142857142857142 6/7
> (/ 400000000000000000000 7)
57142857142857142857 1/7
> (/ 500000000000000000000 7)
71428571428571428571 3/7
> (/ 600000000000000000000 7)
85714285714285714285 5/7
\end{lstlisting}
\end{center}

\end{problem}

\begin{problem} %13
The guess here should be that each of the answers will be in some permutation of 142857 barring when multiplied by 7.
Let us check.

\begin{center}
\opmul{142857}{1}
\end{center}

\begin{center}
\opmul{142857}{2}
\end{center}

\begin{center}
\opmul{142857}{3}
\end{center}

\begin{center}
\opmul{142857}{4}
\end{center}

\begin{center}
\opmul{142857}{5}
\end{center}

\begin{center}
\opmul{142857}{6}
\end{center}

\begin{center}
\opmul{142857}{7}
\end{center}

The way to answer this is to start at the one's place digit and work backwards.

Answer is verified above.

\end{problem}

\begin{problem} %14
Let us go for the first 10 natural numbers from 1 to 10.

Case for 1:

Anything divided by 1 is the same thing. So the dividend and quotient is same and there is no remainder.

\begin{center}
\longdivision{1000000}{1}
\end{center}

Case for 2:

Since the dividend ends with 0 it is an even number. So half of dividend is the quotient and remainder is 0.

\begin{center}
\longdivision{1000000}{2}
\end{center}

Case for 3:

In this case we will always get a remainder of 1 since 1 less than 10 or 100 or 1000 is divisible by 3.

\begin{center}
\longdivision{1000000}{3}
\end{center}

Case for 4:

Except for 10 as the dividend where we will get a remainder of 2 and a quotient fo 2 also, the rest of
the dividends will always be one fourths of the dividend since the dividend ends with 2 zeroes. The
remainder will be 0.

\begin{center}
\longdivision{10}{4}
\longdivision{1000000}{4}
\end{center}

Case for 5:

Here every number from 10 onwards will be divisible by 5. There will be no remainder.

\begin{center}
\longdivision{1000000}{5}
\end{center}

Case for 6:

In this case the remainder will always be 4. We will be stuck in an infinite loop of dividing 40 by 6 once we
finish our first division of 10 by 6. The quotient therefore will be 1 followed by all 6s. Example below.

\begin{center}
\longdivision{1000000}{6}
\end{center}

Case for 7 done earlier.

Case for 8:

The pattern in this case is that we have 1, 2, 5 as the quotient. And once there are no remainders
left we keep appending 0s to the quotient of 125.

\begin{center}
\longdivision{10}{8}
\longdivision{100}{8}
\longdivision{1000}{8}
\longdivision{10000}{8}
\longdivision{100000}{8}
\longdivision{1000000}{8}
\end{center}

Case for 9:

The first division by 9 gives a remainder of 1 and then an endless loop of 10 divided by 9. Remainder will
always be 1 and quotient will be 1111....

\begin{center}
\longdivision{1000000}{9}
\end{center}

Case for 10:

Fairly simple. Just remove the last zero from the dividend to get the quotient and there is no remainder.

\begin{center}
\longdivision{1000000}{10}
\end{center}

\end{problem}

%%%%%%%%%%%%%%%%%%%%%%%%%%%%%%%%%%%%%%%%%
% Chapter End
%%%%%%%%%%%%%%%%%%%%%%%%%%%%%%%%%%%%%%%%%

%%%%%%%%%% 7
\section{The binary system}

\begin{problem} %15
Binary to Decimal conversion can be written by:

[(0 or 1) $\times$ $2^0$] + [(0 or 1) $\times$ $2^1$] + [(0 or 1) $\times$ $2^2$]...

The numbers in the given list are basically the set of whole numbers.

\begin{table}[h!]
\centering
\begin{tabular}{c c l}
\textbf{Binary} & \textbf{Decimal} & \textbf{Expanded form} \\ \hline
0     & 0  & 0 \\
1     & 1  & $2^0$ \\
10    & 2  & $2^1$ \\
11    & 3  & $2^1 + 2^0$ \\
100   & 4  & $2^2$ \\
101   & 5  & $2^2 + 2^0$ \\
110   & 6  & $2^2 + 2^1$ \\
111   & 7  & $2^2 + 2^1 + 2^0$ \\
1000  & 8  & $2^3$ \\
1001  & 9  & $2^3 + 2^0$ \\
1010  & 10 & $2^3 + 2^1$ \\
1011  & 11 & $2^3 + 2^1 + 2^0$ \\
1100  & 12 & $2^3 + 2^2$ \\
1101  & 13 & $2^3 + 2^2 + 2^0$ \\
1110  & 14 & $2^3 + 2^2 + 2^1$ \\
1111  & 15 & $2^3 + 2^2 + 2^1 + 2^0$ \\
10000 & 16 & $2^4$ \\
10001 & 17 & $2^4 + 2^0$ \\
10010 & 18 & $2^4 + 2^1$ \\
10011 & 19 & $2^4 + 2^1 + 2^0$ \\
10100 & 20 & $2^4 + 2^2$ \\
10101 & 21 & $2^4 + 2^2 + 2^0$ \\
10110 & 22 & $2^4 + 2^2 + 2^1$ \\
\end{tabular}
\end{table}

Answer is verified

\end{problem}

\begin{problem} %16
This problem is basically binary representation of a natural number. 

Let $S=\{2^0,2^1,\dots,2^{n-1}\}$. Every integer $m$ with $0\le m\le (2^n-1)$
can be written uniquely as a sum of distinct elements of $S$.

The proof for this can be demonstrated using induction but we will skip that here. The
3\textsuperscript{rd} column in the previous problem (problem number 15) already shows the solution for
this problem.

\end{problem}

\begin{problem} %17
We refer back to the table in problem 15. For 14 in decimal the equivalent binary representation is
1110. 10000 binary is $ (1 * 2^4) + (0 * 2^3) + (0 * 2^2) + (0 * 2^1) + (0 * 2^0) $ and that is 16.

\end{problem}

\begin{problem} %18
From the binary representation theorem given in problem 16 let us look at a number of the form
$2^n \le 45$. The biggest $n$ here is 5 where we get $2^5$ = 32. So we have 32 as 100000. We need 13 more. We
apply the same logic and arrive at 1101 for 13. Thus adding the binary representations we get 101101 as the binary form 
of 45.

Answer is 101101
\end{problem}

\begin{problem} %19
10101101 in binary can be converted to decimal easily.

$ 10101101 = (1 * 2^7) + (0 * 2^6) + (1 * 2^5) + (0 * 2^4) + (1 * 2^3) + (1 * 2^2) + (0 * 2^1) + (1 * 2^0) $
$ 10101101 = 128 + 32 + 8 + 4 + 1 = 173 $

Answer is 173

\end{problem}

\begin{problem} %20

Binary Addition

0 + 0 = 0 this is so because $(0 * 2^0) + (0 * 2^0)$\\
0 + 1 = 1 this is so because $(0 * 2^0) + (1 * 2^0)$\\
1 + 1 = 0 this is so because $(1 * 2^0) + (1 * 2^0)$ the answer is 2 which is $(1 * 2^1) + (0 * 2^0)$ thus carry 1\\

\[
\begin{array}{r}
   1010\\
+ \,101\\
\hline
   1111
\end{array}
\]
This is 10 + 5 = 15 in base 10.

\[
\begin{array}{r}
   1111\\
+ \,1\\
\hline
   10000
\end{array}
\]
This is 15 + 1 = 16 in base 10.

\[
\begin{array}{r}
   1011\\
+ \,1\\
\hline
   1100
\end{array}
\]
This is 11 + 1 = 12 in base 10.

\[
\begin{array}{r}
   1111\\
+ \,1111\\
\hline
   11110
\end{array}
\]
This is 15 + 15 = 30 in base 10.

\end{problem}



\begin{problem} %21
Binary Subtraction

0 - 0 = 0 this is so because $(0 * 2^0) - (0 * 2^0)$\\
1 - 0 = 1 this is so because $(1 * 2^0) - (0 * 2^0)$\\
0 - 1 = 1 this is so because $(0 * 2^0) - (1 * 2^0)$ results in a borrow of 10\textsubscript{2}. So now
we have a 2 in decimal subtracted with 1. Thus there is a borrow of 1 in this case\\
1 - 1 = 0 this is so because $(1 * 2^0) - (1 * 2^0)$\\

\[
\begin{array}{r}
   1101\\
- \,101\\
\hline
   1000
\end{array}
\]
This is 13 - 5 = 8 in base 10.

\[
\begin{array}{r}
   110\\
- \,1\\
\hline
   101
\end{array}
\]
This is 6 - 1 = 5 in base 10.

\[
\begin{array}{r}
   1000\\
- \,1\\
\hline
   111
\end{array}
\]
This is 8 - 1 = 7 in base 10.

\end{problem}

\begin{problem} %22
Binary multiplication

0 multiplied by anything is 0 and 1 multiplied by 1 is 1. In the binary case we have\\
0 * 0 = 0 this is so because $(0 * 2^0) * (0 * 2^0)$\\
0 * 1 = 0 this is so because $(0 * 2^0) * (1 * 2^0)$\\
1 * 1 = 1 this is so because $(1 * 2^0) * (1 * 2^0)$\\

\[
\begin{array}{r}
   1101\\
\times \,1010\\
\hline
   0000\\
  11010\\
  000000\\
+ \,1101000\\
\hline
  10000010
\end{array}
\]

In base 10 this is $13 * 10 = 130$.

\end{problem}



\begin{problem} %23
Binary Division

This is same as long division in base 10.

\[
11001_2 \div 101_2 = 101_2 \text{ remainder } 0_2
\]

This is $\frac{25}{5}$ in base 10

\end{problem}



\begin{problem} %24
Binary fractions

For fractions in binary the only difference from that in base 10 is that only when the denominator in binary division
is a power of 2 that is of the form $2^n$ then we get a terminating fraction else we do not.

$\frac{1}{3}$ is 0.3333.. in decimal. In Binary we can do a division of $\frac{1}{11}$. 

\[
\frac{1_2}{11_2} = 0.\overline{01}_2 = \frac{1}{3}.
\]

\end{problem}

%%%%%%%%%%%%%%%%%%%%%%%%%%%%%%%%%%%%%%%%%
% Chapter End
%%%%%%%%%%%%%%%%%%%%%%%%%%%%%%%%%%%%%%%%%




%%%%%%%%%% 8
\section{The commutative law}
No problems
%%%%%%%%%%%%%%%%%%%%%%%%%%%%%%%%%%%%%%%%%
% Chapter End
%%%%%%%%%%%%%%%%%%%%%%%%%%%%%%%%%%%%%%%%%

%%%%%%%%%% 9
\section{The associative law}

\begin{problem} %25
  Tried it. Beg to differ from Gelfand and Shen on this one. The flavor and aroma are different
  in the two processes described in the equation.
\end{problem}

\begin{problem} %26
  First do the addition of 17999 + 1 to get 18000 then add 357 since the last 3 digits of 18000
  are 0s. The answer is 18357.
\end{problem}

\begin{problem} %27
In such cases add 1 and subtract 1 at the end. So we get 18357 from the same steps as problem 26
then we subtract 1. The answer is 18356.
\end{problem}

\begin{problem} %28
  Here we add 899 + 101 first. It is 900 + 100 which is a thousand. 1000 + 1343 is 2343.
\end{problem}

\begin{problem} %29
  $25 . 4$ is done first to give 100. Then then answer is 3700.
\end{problem}

\begin{problem} %30
  In this we do $125.8$ first which again gives us 1000. The final answer is 37000.
\end{problem}

%%%%%%%%%%%%%%%%%%%%%%%%%%%%%%%%%%%%%%%%%
% Chapter End
%%%%%%%%%%%%%%%%%%%%%%%%%%%%%%%%%%%%%%%%%


%%%%%%%%%% 10
\section{The use of parentheses}

\begin{problem} %31
This is not a useful problem from an algebra perspective. This is a combinatorics problem and a good one. Let us try and build a reasoning around how to solve for the simplest cases. 

For every case we need to partition the numbers into 2 groups. Each group will have to stand on
its own. And in each of these sub groups we have a situation for which we would have already done the count prior.

Case 0: No number - We do not need to put any parentheses.
\begin{center}
0 $\rightarrow$ 0 
\end{center}

Case 1: 2 - We do not need any parentheses but can put one like (2).
\begin{center}
1 $\rightarrow$ 1 
\end{center}

Case 2: 2.3 - Just one like (2.3)
\begin{center}
2 $\rightarrow$ 1 
\end{center}

Case 3: 2.3.4 - Partitioning into smaller groups gives a group of 2 and 1. It is (2.3).4 and 2.(3.4)
Thus we get 1 + 1 = 2.\\
Case 3 $\rightarrow$ $1 + 2$ - 2.(3.4) $\rightarrow$ $1 + 1$\\
\begin{center}
3 $\rightarrow$ 2 
\end{center}

Case 4: 2.3.4.5 - The book solves this. Let us make partitions 2.(3.4.5). We notice that we have
simplified it to case 3 here which will repeat twice. The other partition is (2.3).(4.5) which is
two cases before i.e. case 2. So we get the following:
$2 + 2$ from Case 3 + $1$ from Case 1.\\
Case 4 $\rightarrow$ $1 + 3$ - 2.(3.4.5) $\rightarrow$ $2 + 2$\\
Case 4 $\rightarrow$ $2 + 2$ - (2.3).(4.5) $\rightarrow$ $1$\\
\begin{center}
4 $\rightarrow$ 5 
\end{center}

Case 5: 2.3.4.5.6 - This is the question we are asked. The algorithm requires us to partition.
2.(3.4.5.6) is one way to partition and we have reduced the problem to Case 4 above which repeats 
twice. The other partition is (2.3).(4.5.6). In this case we go two steps back.
So we get $5 + 5 + 2 + 2$.\\
Case 5 $\rightarrow$ $1 + 4$ - 2.(3.4.5.6) $\rightarrow$ $5 + 5$\\
Case 5 $\rightarrow$ $2 + 3$ - (2.3).(4.5.6) $\rightarrow$ $2 + 2$\\
\begin{center}
5 $\rightarrow$ 14
\end{center}

Case 6: 2.3.4.5.6.7 - Let us take it a notch higher. The sub cases which can be built are\\
Case 6 $\rightarrow$ $1 + 5$ - 2.(3.4.5.6.7) $\rightarrow$ $14 + 14$\\
Case 6 $\rightarrow$ $2 + 4$ - (2.3).(4.5.6.7) $\rightarrow$ $5 + 5$\\
Case 6 $\rightarrow$ $3 + 3$ - (2.3.4).(5.6.7) $\rightarrow$ $2 \times 2$\\
\begin{center}
6 $\rightarrow$ 42
\end{center}

In fact we can even make it generic. Essentially what we are doing is partitioning numbers.
Then for each partition we are looking back at previous permutations and adding. Note that in some cases
we need to multiply too (for instance last sub case in Case 6). Can we generalize this? Yes, these are
basically \href{https://en.wikipedia.org/wiki/Catalan_number}{Catalan} numbers! The $nth$ Catalan number is given by the expression for all $n\ge 0$
\[
\frac{(2n)!}{(n + 1)! \; n!}
\]
\end{problem}

\begin{problem} %32
This too is a combinatorics problem. 

Let us take a simple trivial case of the problem.\\
2 $\rightarrow$ we do not need any parentheses\\
2.3 $\rightarrow$ we still do not need any parentheses\\
2.3.4 $\rightarrow$ now we need to put the first pair so that it becomes (2.3).4. So for $n = 3$ digits we have 2 i.e. $2(n-2)$ parentheses.\\
2.3.4.5 $\rightarrow$ we need to put one additional pair ((2.3).4).5. So for $n = 4$ digits we have 4 i.e. $2(n-2)$ parentheses.\\


Generalizing, for $n>2$ number of parentheses is
\[
2(n - 2)
\]

In the given question we have the question as 2.3.4.5.....97.98.99.100. These are $(100 - 2) + 1$ digits.
Therefore $n = 99$ and total number of parentheses are $2 \times (99 - 2)$ which is 194.\\

Answer is 194.

\end{problem}

\begin{problem} %33
This is easily doable like the way the child Gauss did. Basically there are 100 elements in
this series

\[
\begin{array}{rcl}
S   &=& 1 + 2 + 3 + \ldots + 99 + 100 \\
+ S   &=& 100 + 99 + 98 + \ldots + 2 + 1 \\
\hline
2S  &=& 101 + 101 + 101 + \ldots + 101\\
\end{array}
\]
\\
\[
2S = 100\times 101 \\
\]
\\
\[
S = \dfrac{100\times 101}{2} = 5050
\]

\end{problem}

%%%%%%%%%%%%%%%%%%%%%%%%%%%%%%%%%%%%%%%%%
% Chapter End
%%%%%%%%%%%%%%%%%%%%%%%%%%%%%%%%%%%%%%%%%


%%%%%%%%%% 11
\section{The distributive law}
\begin{problem} %34
\begin{gather*}
= 1001 \times 20 \\
= (1000 + 1) \times 20 \\
= 20000 + 20 \\
= 20020
\end{gather*}

In this we could simply multiply 1001 by 2 to get 2002 and then append a 0 behind it too.

\end{problem}

\begin{problem} %35
\begin{gather*}
= 1001 \times 102 \\
= (1000 + 1) \times 102 \\
= 102000 + 102 \\
= 102102
\end{gather*}
\end{problem}

\begin{problem} %36
\begin{gather*}
(a + b + c + d + e) (x + y + z)
\end{gather*}
For each number from a to e we will get 3 terms one for each x, y and z. Thus total of 3 * 5 terms. 15 terms.
\end{problem}

%%%%%%%%%%%%%%%%%%%%%%%%%%%%%%%%%%%%%%%%%
% Chapter End
%%%%%%%%%%%%%%%%%%%%%%%%%%%%%%%%%%%%%%%%%

%%%%%%%%%% 12
\section{Letters in algebra}
\begin{problem} %37
Let small vessel volume be $x$ and big vessel volume be $y$, then
\[
x + y = 5
\]
\[
2x + 3y = 13
\]
Solving these two linear equations we get $y = 3$ and $x = 2$. To solve such equations make the
coefficients of one of the unknown same and subtract one equation from the other.
\end{problem}

\begin{problem}%38
The simple explanation is given in the book.
\begin{gather*}
x \\
x + 3 \\
2 . (x + 3) \\
(2x + 6) - x \\
(x + 6) - 4 = x + 2 \\
(x + 2) - x \\
2 
\end{gather*}
\end{problem}

%%%%%%%%%%%%%%%%%%%%%%%%%%%%%%%%%%%%%%%%%
% Chapter End
%%%%%%%%%%%%%%%%%%%%%%%%%%%%%%%%%%%%%%%%%


%%%%%%%%%% 13
\section{The addition of negative numbers}
No problems
%%%%%%%%%%%%%%%%%%%%%%%%%%%%%%%%%%%%%%%%%
% Chapter End
%%%%%%%%%%%%%%%%%%%%%%%%%%%%%%%%%%%%%%%%%

%%%%%%%%%% 14
\section{The multiplication of negative numbers}
No problems
%%%%%%%%%%%%%%%%%%%%%%%%%%%%%%%%%%%%%%%%%
% Chapter End
%%%%%%%%%%%%%%%%%%%%%%%%%%%%%%%%%%%%%%%%%

%%%%%%%%%% 15
\section{Dealing with fractions}
\begin{problem} %39
The explanation of this problem is humorous! But such a nice way to put it.

If only vodka bottles were used in the explanation everything would have fallen in place
with this soviet era content.

Anyways.

$\frac{1}{3} \times \frac{7}{7}$ and $\frac{2}{7} \times \frac{3}{3}$ \\


$\frac{7}{21}$ and $\frac{6}{21}$ \\

We conclude that $\frac{1}{3}$ is bigger.

\end{problem}

\begin{problem} %40
This is a problem also found in the initial assessment for the Gelfand Correspondence Program.

\begin{gather*}
\frac{10001}{10002} = 1 - \frac{1}{10002}
\end{gather*}

Similarly,

\begin{gather*}
\frac{100001}{100002} = 1 - \frac{1}{100002}
\end{gather*}

As we can see the $\frac{1}{100002}$ much more smaller than $\frac{1}{10002}$. Thus
when we subtract a smaller number from 1 we are left with a bigger number.

Answer is $\frac{100001}{100002}$
\end{problem}

\begin{problem} %41
Good problem.

\begin{gather*}
\frac{12345}{54321} \times \frac{54322}{54322}\\
\end{gather*}

\begin{gather*}
\frac{12346}{54322} \times \frac{54321}{54321}\\
\end{gather*}

Now lets only look at the numerators since the denominators are equal.

\begin{gather*}
(12345) \times (54321 + 1) \\
(12345 + 1) \times (54321)
\end{gather*}

Make them both have common terms

\begin{gather*}
(12345 \times 54321) + 12345 \\
(12345 \times 54321) + 54321
\end{gather*}

It is clear now that the second fraction is bigger because it has 54321 in the numerator vs 
12345 in the first fraction when all other terms are same in numerator and denominator.

Answer is $\frac{12346}{54322}$
\\

Just to verify this in Scheme. We get a positive fraction when we subtract the first fraction
from the second one.

\begin{center}
\begin{lstlisting}
> (- (/ 12346 54322) (/ 12345 54321))
6996/491804227
\end{lstlisting}
\end{center}

\end{problem}

\begin{problem} %42
These 3 problems are pretty difficult actually for middle school kids.

(a) We will use proof by contradiction to prove this.

Assume the greatest common divisor of $a$ and $b$ is $m$.
\[
gcd(a,b) = m, m>1
\]

So $m | a$ and $m | b$ ($x|y$ reads \textit{x divides y})

Therefore $m$ should also divide $ad - bc$ i.e. $m|(ad -bc)$

But we know $ad - bc = \pm 1$. So in $\frac{(ad - bc)}{m}$ denominator has to be $+1$

Thus $m$ is $+1$ and $gcd(a,b) = 1$

Hence $\frac{a}{b}$ cannot be further simplified. The same proof can be used for $\frac{c}{d}$.

(b) Now to second part of this problem. I struggled with this one quite a bit. A little bit
background on \href{https://en.wikipedia.org/wiki/Farey_sequence}{Farey Sequence}. Niven and Zuckerman (1972)
defined Farey Sequence as
\begin{quotation}
  \textit{The sequence of all reduced fractions with denominators not exceeding n, listed in order of their size, is called the Farey sequence of order n.}
\end{quotation}
Sometimes the definition is restricted to the interval 0 to 1. In this interval say we look at
Farey number 3 which is given by 
\[
F_3 = \Bigl\{ \frac{0}{1}, \frac{1}{3}, \frac{1}{2}, \frac{2}{3}, \frac{1}{1} \Bigl\}
\]
The third term minus the second term here is
\[
\frac{1}{2} - \frac{1}{3} = \frac{1}{6}
\]
The numerator is 1. In Farey Sequences the numerator on subtraction of two consecutive elements is always $\pm1$

The problem says that in a Farey sequence of $\frac{a}{b}$ and $\frac{c}{d}$ the fraction $\frac{a+b}{c+d}$ is always between
$\frac{a}{b}$ and $\frac{c}{d}$. Let us plug in the numbers from the $F_3$ sequence quoted above.

\[
\frac{a+b}{c+d} = \frac{1 + 3}{1 + 2} = \frac{4}{3}
\]

So the inequality is now

\[
\frac{1}{3} < \frac{4}{3} < \frac{1}{2}
\]

But $\frac{4}{3}$ is greater than 1 and does not lie there! The point is that there is a typographical mistake in this problem.
The actual fraction between $\frac{a}{b}$ and $\frac{c}{d}$ is called the Mediant fraction and is given as
\[
\frac{a}{b} < \frac{a + c}{b + d}<\frac{c}{d}
\]

To verify
\[
\frac{1}{3} < \frac{2}{5} < \frac{1}{2}
\]

And this is correct.

Back to part (b) of the problem now. We need to prove the below.
\[
\frac{a}{b} < \frac{a + c}{b + d}<\frac{c}{d}
\]

Suppose there existed a fraction $\frac{p}{q}$ in between $\frac{a}{b}$ and $\frac{c}{d}$, then
\[
\frac{a}{b} < \frac{p}{q}<\frac{c}{d}
\]
then, since they are Farey neighbors.
\[
pb - aq = 1
\]
\[
cq - pd = 1
\]
\begin{gather*}
pb - aq = cq - pd\\ 
pb + pd = cq + aq\\
p (b + d) = q (a +c)\\
\frac{p}{q} = \frac{a+c}{b+d}
\end{gather*}

Hence we proved that the mediant fraction between $\frac{a}{b}$ and $\frac{c}{d}$ has to be $\frac{a+c}{b+d}$

(c) Last part of this challenging problem. We have say
\[
\frac{a}{b} < \frac{e}{f}<\frac{c}{d}
\]
\begin{gather*}
\frac{e}{f} - \frac{a}{b} = \frac{be - af}{bf}\\
\end{gather*}
Since $be - af$  will be a positive integer and therefore at least 1, we can say
\begin{gather*}
\frac{e}{f} - \frac{a}{b} \ge \frac{1}{bf}\\
\end{gather*}
Similarly we get
\begin{gather*}
\frac{c}{d} - \frac{e}{f} \ge \frac{1}{df}\\
\end{gather*}

Now we have to visualize. Moving from $\frac{a}{b}$ to $\frac{e}{f}$ and from that finally to $\frac{c}{d}$.
Total Distance is $\frac{1}{bd}$ (from the denominator between $\frac{a}{b}$ and $\frac{c}{d}$). So we get
\[
\frac{1}{bd} = \Bigl( \frac{e}{f} - \frac{a}{b} \Bigl) + \Bigl( \frac{c}{d} - \frac{e}{f} \Bigl)
\]

But we know from the inequalities above.
\[
\frac{1}{bd} \ge  \frac{1}{bf} + \frac{1}{df} = \frac{b + d}{bdf}
\]
\[
\frac{1}{bd} \ge  \frac{b + d}{bdf}
\]
\[
1 \ge  \frac{b + d}{f}
\]
\[
f \ge  b + d
\]

Hence proved that $f$ cannot be less than $b + d$

Difficult problems for middle schoolers!
\end{problem}

\begin{problem}
This is application of mediant formula we derived in problem 42.

\[
\frac{a}{b} < \frac{a + c}{b + d}<\frac{c}{d}
\]

The end 2 pieces of $\frac{1}{20}$ can be omitted so we are left with 18 pieces. There are 6 red marks (7 equal segments will require 6 marks) and
12 green marks (13 equal segments). We can visualize from the left of the stick the first cut would be at $\frac{1}{20}$ which we have omitted
then the next cut will be at $\frac{1}{13}$ and then at $\frac{1}{7}$. Let us find mediant between $\frac{1}{13}$ and $\frac{1}{7}$.
\[
\frac{1}{13} < \frac{1 + 1}{13 + 7}<\frac{1}{7}
\]
\[
\frac{1}{13} < \frac{2}{20}<\frac{1}{7}
\]

Note that the mediant will always lie at $\frac{k}{20}$. Thus the cut will be there and each of the 18 pieces will have
only color either green or red.

The problem is solved in the book.

\end{problem}

\begin{problem} %44
First expression is
\begin{gather*}
5\% \times 7 \times 10^9\\
35 \times 10^7
\end{gather*}

Second expression is
\begin{gather*}
7\% \times 5 \times 10^9\\
35 \times 10^7
\end{gather*}

Thus they are equal.

\end{problem}

\begin{problem} %45
The more systematic way to reason is to say what number $k$ when multiplied by $\frac{2}{3}$ gives $\frac{1}{2}$.

\begin{gather*}
k \times \frac{2}{3} = \frac{1}{2}\\
k = \frac{1}{2} \times \frac{3}{2} = \frac{3}{4}
\end{gather*}

So we fold the string in half once, then re fold it. We get quarter of the original two thirds. Now we cut off one of the one fourths and then
we are left with three fourths of the original two thirds which is now half.

The problem is solved in the book.
\end{problem}

%%%%%%%%%%%%%%%%%%%%%%%%%%%%%%%%%%%%%%%%%
% Chapter End
%%%%%%%%%%%%%%%%%%%%%%%%%%%%%%%%%%%%%%%%%

  

%%%%%%%%%% 16
\section{Powers}

\begin{problem} %46
(a) 1024 is the answer. It is good to have the powers of 2 memorized for quick computation - 1, 2, 4, 8, 16, 32, 64, 128, 256, 512, 1024 (for folks
interested in computers this will be second nature).

(b) 1000 - a thousand.

(c) 10000000 - 10 million - In India this is also called a Crore.

\end{problem}

\begin{problem} %47
Assuming by \textit{decimal digits} the authors mean digits. Then the answer is 10001. 1000 zeroes and a one 1.

\end{problem}

\begin{problem} %48
Total number of seconds in a year

\[
60 \times 60 \times 24 \times 365 = 31536000
\]

Distance traveled in 4 light years will be

\[
4 \times 3 \times 10^5 \times 31536000 = 37843200000000
\]

That is a whopping 37.8432 Trillion Kilometers 

\end{problem}

%%%%%%%%%%%%%%%%%%%%%%%%%%%%%%%%%%%%%%%%%
% Chapter End
%%%%%%%%%%%%%%%%%%%%%%%%%%%%%%%%%%%%%%%%%


%%%%%%%%%% 17
\section{Big numbers around us}

\begin{problem} %49
(a) $2^{10}$ is 1024 so $2^{20} = 2^{10} \times 2^{10} = 1024 \times 1024$\\
This will 1 followed by 6 other digits. So total digits will be 7.

Verified in Scheme
\begin{lstlisting}
> (expt 2 20)
1048576
\end{lstlisting}

(b) $2^{100}$ = $2^{10} \times 2^{10}$ ....ten times\\
$1024 \times 1024 \times 1024$..... ten times\\
$(10^3 + 24)^2 \times (10^3 + 24)^2$....five times\\
$(1000000 + 576 + 48000) \times (1000000 + 576 + 48000)$ five times\\
$1048576 \times 1048576 \times 1048576 \times 1048576 \times 1048576 $\\
So we get 31 digits. Why? Because the $1000000$ will grow much bigger than the $48576$. This is intuitively speaking.
Ideally we should do it properly. Let us try that.\\

A natural number $N$ is given. How many digits does this have?\\
If it is between 1 (included) to less than 10 it has 1 digit. If it is from 10 (included) to less than 100 then it has 2 digits.
If it is between 100 (included) to less than 1000 it has 3 digits and so on. Representing it in inequality form.\\
\begin{gather*}
10^0 \le N < 10^1 \; \rightarrow 1 \;digit\\
10^1 \le N < 10^2 \; \rightarrow 2 \;digits\\
10^2 \le N < 10^3 \; \rightarrow 3 \;digits\\
10^3 \le N < 10^4 \; \rightarrow 4 \;digits\\
.....\\
10^{k-1} \le N < 10^k \; \rightarrow k \;digits\\
\end{gather*}

So we have an inequality for the number of digits $k$ for a number N. 
\[
10^{k-1} \le N < 10^k
\]

Let us take logarithms on both sides for the number $2^n$. I understand students would not have been taught this yet. But I am sure if kids
are working through this book they are smart enough to pick this up.
\begin{gather*}
\log_{10} 10^{k-1}  \le \log_{10} 2^n < \log_{10} 10^k\\
k - 1 \le n\log 2 < k
\end{gather*}
rearranging the inequality above we get
\begin{gather*}
n \log 2 < k \le n \log 2 + 1
\end{gather*}
To reiterate the number $2^n$ will have $k$ digits and this $k$ is certainly bigger than $n \log 2$ and an integer and also lesser than or equal to $n \log 2 + 1$.\\
We can write $k$ as 
\[
k = \lfloor n \log 2 \rfloor + 1
\]
So we conclude that the number $2^n$ has $\lfloor n \log 2 \rfloor + 1$ digits.\\

Let us try out some examples.\\

$2^{10} = \lfloor 10 \log 2 \rfloor + 1 = \lfloor 10 \times 0.30102999 \rfloor + 1 = 3 + 1 = 4$\\
Indeed $2^{10} = 1024$ which has 4 digits.\\
Note value of $\log 2$ is $0.3010999$.\\

$2^{100} = \lfloor 100 \log 2 \rfloor + 1 = \lfloor 100 \times 0.30102999 \rfloor + 1 = 30 + 1 = 31$\\
We can verify with certainty that now the answer to this question is $31$.\\

(c) For this part of the problem I will have to use programming. I am unsure how else to construct this graph.\\
Using Racket (a Scheme)
\begin{lstlisting}
#lang racket
(require plot)

;; Function for the actual computation
(define (digits-of-2^n n)
  (add1 (floor (/ (* n (log 2)) (log 10)))))

;; Plot digits for n from 1 to 1000
(plot
 (function digits-of-2^n 1 1000)
 #:x-label "n"
 #:y-label "digits of 2^n"
 #:title  "Number of decimal digits in 2^n")
\end{lstlisting}

\begin{figure}[H]
\centering
\includegraphics[width=0.7\linewidth]{images/problem49c.png}
\label{fig:digits2n}
\end{figure} 

\end{problem}

%%%%%%%%%%%%%%%%%%%%%%%%%%%%%%%%%%%%%%%%%
% Chapter End
%%%%%%%%%%%%%%%%%%%%%%%%%%%%%%%%%%%%%%%%%


%%%%%%%%%% 18
\section{Negative powers}
\begin{problem} %50
(a) $\frac{1}{10}$ or 0.1 \\

(b) $\frac{1}{100}$ or 0.01 \\

(c) $\frac{1}{1000}$ or 0.001 \\

\end{problem}

%%%%%%%%%%%%%%%%%%%%%%%%%%%%%%%%%%%%%%%%%
% Chapter End
%%%%%%%%%%%%%%%%%%%%%%%%%%%%%%%%%%%%%%%%%

%%%%%%%%%% 19
\section{Small numbers around us}

\begin{problem} %51
As per notation yes both are true.\\

\begin{gather*}
a^{-n} = \frac{1}{a^n}\\
a^{-(-k)} = \frac{1}{a^{-k}}\\
a^k = \frac{1}{a^{-k}}
\end{gather*}
$a^0$ is 1 so $\frac{1}{1}$ is 1 again.
\end{problem}

\begin{problem} %52
(a) $a^{10}b^4$\\

(b) $2.a^3b^{-2}$\\
\end{problem}

\begin{problem} %53
(a) $\frac{a^3}{b^5}$ \\

(b) $\frac{1}{a^2b^7}$ \\
\end{problem}

%%%%%%%%%%%%%%%%%%%%%%%%%%%%%%%%%%%%%%%%%
% Chapter End
%%%%%%%%%%%%%%%%%%%%%%%%%%%%%%%%%%%%%%%%%

%%%%%%%%%% 20
\section{How to multiply $a^m$ by $a^n$, or why our definition is convenient}

\begin{problem} %54
Not sure of the ask of this question. Probably the answer is  $a^{m-n}$.
\end{problem}

%%%%%%%%%%%%%%%%%%%%%%%%%%%%%%%%%%%%%%%%%
% Chapter End
%%%%%%%%%%%%%%%%%%%%%%%%%%%%%%%%%%%%%%%%%




%%%%%%%%%% 21
\section{The rule of multiplication for powers}
\begin{problem} %55
  (a) $n = 2000 - 1001 = 999$ \\

  (b) $1001 + n = -2$ thus $n = -1003$ \\

  (c) $\frac{1}{1000}$ vs $\frac{1}{1024}$. Thus $10^{-3}$ is bigger.\\

  (d) $1000 - n = 501$, $n = 499$ \\

  (e) $1000 - n = -4$, $n = 1004$ \\

  (f) $2 \times 100 = n$, $n =200$ \\

  (g) $(2 \times 3)^{100} = a^{100}$, $a = 6$ \\

  (h) $10 \times 15 = n$, $n = 150$\\

\end{problem}

\begin{problem} %56
It does not matter what sign $m$ and $n$ have as such. Specifically if $m > 0$ and $n < 0$ then
$(a^m)^{-n} = \frac{1}{(a^m)^n}$.\\

For either of them to be zero the answer would be 1 since one of the powers is zero.
\end{problem}

\begin{problem} %57
Again signs do not make any difference to the formula $(ab)^n = a^n . b^n$
\end{problem}

\begin{problem} %58
  If $a = 0$ then it will $0$.\\
  If $a > 0$ then it will be $-a^{775}$\\
  If $a < 0$ then it will be $a^{775}$
\end{problem}

\begin{problem} %59
  Ideally $b \ne 0$ is the first call out.\\
  Otherwise it is easy to put any number whether integer or fraction as $n$ here.\\
  So not sure of the intent of the problem in this case.
\end{problem}

\begin{problem} %60
  We can manipulate the base $4^{\frac{1}{2}}$ to $(2^2)^{\frac{1}{2}}$. Thus this can be simplified to
  $2^{(2 \times \frac{1}{2})}$ giving $2^1$. But we need to be careful here since $-2 \times -2 = (-2)^2$ also.
  Therefore the answer will $\pm 2$.\\

  Similarly for $27^{\frac{1}{3}}$ should give the third root because of $3 \times 3 \times 3$ giving $3^3$. Here we will not get $-3$ else that would
  make the answer negative and incorrect.
\end{problem}


%%%%%%%%%%%%%%%%%%%%%%%%%%%%%%%%%%%%%%%%%
% Chapter End
%%%%%%%%%%%%%%%%%%%%%%%%%%%%%%%%%%%%%%%%%


%%%%%%%%%% 22
\section{Formula for short multiplication: The square of a sum}

\begin{problem} %61
  Application of $(a + b)^2 = a^2 + b^2 + 2ab$\\

(a) 
\begin{gather*}
  101^2\\
  = (100 + 1) ^2\\
  = 10000 + 1 + 200\\
  = 10201
\end{gather*}

(b)

\begin{gather*}
   1002^2\\
  = (1000 + 2) ^2\\
  = 1000000 + 4 + 4000\\
  = 1004004
\end{gather*}

\end{problem}

\begin{problem} %62
  Let product $p$ be\\
\[
p = m \times n
\]

Now $m$ and $n$ the factors become $10\%$ bigger\\

\begin{gather*}
  (m + 10\% m) \times (n + 10\% n)\\
  1.1m \times 1.1n\\
  1.21 m \times n\\
  1.21 p\\
  (p + 21\% p)
\end{gather*}

Thus the product becomes $21\%$ bigger.
\end{problem}

\begin{problem} %63
  This question is to drive home the point made in the text that "The
  square of the sum of two terms is the sum of their squares plus two times the product of the terms."

  The core message is that \textit{square of the sum} and \textit{sum of the squares} are
  two different things. Rightly so. Students need to be careful, that is all.

  The answer to the problem is `No'. Why?\\

  Case 1: NN is me a man. I, the father, have a son. The father of the son is me. So this refers to me.
  Now my father has a son but he could have more than one son. In my case we are actually two brothers. So not always true.

  Case 2: NN is my wife. My wife's son has a father which is me. But I am not my wife. So this is incorrect already.
  My wife's father does have a son who is my wife's brother. But my brother in law and wife are not the same person.

  Luckily my real family is good to answer this question!
\end{problem}


%%%%%%%%%%%%%%%%%%%%%%%%%%%%%%%%%%%%%%%%%
% Chapter End
%%%%%%%%%%%%%%%%%%%%%%%%%%%%%%%%%%%%%%%%%

%%%%%%%%%% 23
\section{How to explain the square of the sum formula to your younger brother or sister}

\begin{problem} %64
This is a simple representation of the formula $(a + b)^2 = a^2 + b^2 + 2ab$.\\

\begin{center}
\begin{tikzpicture}[scale=0.9, every node/.style={font=\small}]
  % define lengths
  \def\a{3}
  \def\b{2}

  % outer square (a+b) × (a+b)
  \draw (0,0) rectangle (\a+\b,\a+\b);

  % inner grid lines
  \draw (\a,0) -- (\a,\a+\b);
  \draw (0,\a) -- (\a+\b,\a);

  % color the regions and add labels
  \fill[blue!20] (0,0) rectangle (\a,\a);
  \node at ({\a/2},{\a/2}) {$a^2$};

  \fill[red!20] (\a,\a) rectangle (\a+\b,\a+\b);
  \node at ({\a+\b/2},{\a+\b/2}) {$b^2$};

  \fill[green!20] (\a,0) rectangle (\a+\b,\a);
  \node at ({\a+\b/2},{\a/2}) {$ab$};

  \fill[green!20] (0,\a) rectangle (\a,\a+\b);
  \node at ({\a/2},{\a+\b/2}) {$ab$};

  % dimension markers
  \draw[<->] (0,-0.5) -- (\a,-0.5) node[midway,below] {$a$};
  \draw[<->] (\a,-0.5) -- (\a+\b,-0.5) node[midway,below] {$b$};
  \node at ({0.5*(\a+\b)}, -1.1) {$a+b$};

  \draw[<->] (-0.5,0) -- (-0.5,\a) node[midway,left] {$a$};
  \draw[<->] (-0.5,\a) -- (-0.5,\a+\b) node[midway,left] {$b$};
  \node[rotate=90] at (-1.1,{0.5*(\a+\b)}) {$a+b$};
\end{tikzpicture}
\end{center}
\end{problem}

\begin{problem} %65
  (a) $99^2 = (100 - 1)^2 = 10000 + 1 - 200 = 9801 $\\

  (b) $998^2 = (1000 - 2)^2 = 1000000 + 4 - 4000 = 996004 $\\
\end{problem}

\begin{problem} %66

  (a) When $a=b$ then\\
  The square of the sums gives $4a^2$ or $4b^2$\\
  The square of the difference gives $0$\\

  (b) When $a=2b$ then\\
  The square of the sums gives $\frac{9}{4}a^2$ or $9b^2$\\
  The square of the difference gives $\frac{a^2}{4}$ or $b^2$\\
  
\end{problem}

%%%%%%%%%%%%%%%%%%%%%%%%%%%%%%%%%%%%%%%%%
% Chapter End
%%%%%%%%%%%%%%%%%%%%%%%%%%%%%%%%%%%%%%%%%

%%%%%%%%%% 24
\section{The difference of squares}

\begin{problem} %67
\begin{gather*}
(a + b) (a - b) = a^2 \cancel{- ab} \cancel{+ ba} - b^2 = a^2 - b^2
\end{gather*}
  
\end{problem}

\begin{problem} %68
  \begin{gather*}
    101 \times 99 = (100 + 1) (100 - 1) = 100^2 - 1^2 = 9999
  \end{gather*}
\end{problem}

\begin{problem} %69
  We just cut it vertically as shown with the dotted line and then stack the two rectangles with
  $(a-b)$ side matching.

\begin{center}
\usetikzlibrary{calc}

\begin{tikzpicture}[scale=0.8, every node/.style={font=\small}]
  %--------------------------------
  % Parameters
  %--------------------------------
  \def\a{4}     % side of big square
  \def\b{1.5}   % removed square side

  %================================
  % FIRST FIGURE: L-shaped region
  %================================

  %--- Shade the left tall rectangle (a-b) × a
  \fill[blue!15] (0,0) rectangle (\a-\b,\a);

  %--- Shade the lower right rectangle b × (a-b)
  \fill[green!20] (\a-\b,0) rectangle (\a,\a-\b);

  %--- The removed b × b square stays white
  \fill[white] (\a-\b,\a-\b) rectangle (\a,\a);
  \draw (\a-\b,\a-\b) rectangle (\a,\a); % outline it

  %--- Draw the L-shape outline
  \draw (0,0) --
        (\a,0) --
        (\a,\a-\b) --
        (\a-\b,\a-\b) --
        (\a-\b,\a) --
        (0,\a) -- cycle;

  %--- Dotted cut line at x = a - b
  \draw[dashed] (\a-\b,0) -- (\a-\b,\a-\b);

  %--- Labels
  \node[left]  at ($(0,0)!0.5!(0,\a)$) {$a$};
  \node[below] at ($(0,0)!0.5!(\a,0)$) {$a$};
  \node[right] at ($( \a-\b,\a-\b)!0.5!(\a-\b,\a)$) {$b$};
  \node[above] at ($( \a-\b,\a-\b)!0.5!(\a,\a-\b)$) {$b$};


  %================================
  % SECOND FIGURE: rearranged rectangle
  %================================
  \begin{scope}[xshift=8.5cm]

    %--- Blue rectangle (a-b) × a
    \fill[blue!15] (0,0) rectangle (\a-\b,\a);

    %--- Green rectangle (a-b) × b on top
    \fill[green!20] (0,\a) rectangle (\a-\b,\a+\b);

    %--- Outer boundary of total rectangle (a-b) × (a+b)
    \draw (0,0) rectangle (\a-\b,\a+\b);

    %--- Line dividing a and b parts
    \draw (0,\a) -- (\a-\b,\a);

    %--- Side labels
    \node[below] at ($(0,0)!0.5!(\a-\b,0)$) {$a-b$};
    \node[left]  at ($(0,0)!0.5!(0,\a+\b)$) {$a+b$};

    %--- Internal labels
    \node[right] at (\a-\b,0.5*\a) {$a$};
    \node[right] at (\a-\b,\a+0.5*\b) {$b$};
  \end{scope}

\end{tikzpicture}
\end{center}

\end{problem}

\begin{problem} %70
  Let the larger number be $n$ then the other number will be $(n - 2)$. We can write:
  \begin{gather*}
    n (n - 2) + 1 = n^2 - 2n + 1 = (n - 1)^2
  \end{gather*}
  $(n - 1)^2$ is a perfect square and the number $(n - 1)$ is between $n$ and $(n - 2)$.
\end{problem}

\begin{problem} %71
  The difference between the squares of two consecutive numbers $n$ and $(n + 1)$ is
  \begin{gather*}
    (n + 1)^2 - n^2 = n^2 + 1 + 2n - n^2 = 2n + 1
  \end{gather*}

  Now difference between the squares of the next two consecutive numbers $(n + 1)$ and $(n + 2)$ is
  \begin{gather*}
    (n + 2)^2 - (n + 1)^2 = n^2 + 4 + 4n - n^2 - 1 - 2n = 2n + 3
  \end{gather*}

  So the difference between the two differences is
  \begin{gather*}
    (2n + 3) - (2n + 1) = 3 - 1 = 2
  \end{gather*}

  $2$ is the constant difference. This is called an arithmetic progression.
\end{problem}

\begin{problem} %72
  This is a nice trick. Let a number be of the form $n5$. This is a two digit number but we can extend
  the logic for higher digit numbers.

  $n5$ can be written as $10n + 5$.

  So the square will be $(10n + 5)^2$. This can be rearranged as.

  \begin{gather*}
    (10n + 5)^2 = 100n^2 + 25 + 100n = 100n (n + 1) + 25
  \end{gather*}

  The $100 n (n + 1)$ is a number $n$ times $(n + 1)$ i.e. two consecutive numbers. Multiplying
  by 100 gives it the correct place in decimal value system as thousandth for this two digit square.
  We already have the left over 25 for the ending two digits. So we get
  \begin{gather*}
    (n5)^2 = (n \times (n + 1))25
  \end{gather*}

  Thus we can prove this trick.
\end{problem}

\begin{problem} %73
  \begin{gather*}
    (a + b + c)^2 = (a + b + c) \times (a + b + c)\\
    = a^2 + ab + ac + ab + b^2 + bc + ac + bc + c^2\\
    = a^2 + b^2 + c^2 + 2 (ab + bc + ca)
  \end{gather*}

  Visually we see it as below.
  \begin{center}
\begin{tikzpicture}[scale=0.9, every node/.style={font=\small}]
  %--------------------------------
  % Parameters
  %--------------------------------
  \def\a{3}
  \def\b{2}
  \def\c{1.5}

  %================================
  % FIRST: SHADING (background)
  %================================

  % Row 1 (height a)
  \fill[blue!20] (0,\b+\c) rectangle (\a,\a+\b+\c);             % a^2
  \fill[green!20] (\a,\b+\c) rectangle (\a+\b,\a+\b+\c);        % ab
  \fill[green!20] (\a+\b,\b+\c) rectangle (\a+\b+\c,\a+\b+\c);  % ac

  % Row 2 (height b)
  \fill[green!20] (0,\c) rectangle (\a,\b+\c);                  % ba
  \fill[red!20]   (\a,\c) rectangle (\a+\b,\b+\c);              % b^2
  \fill[green!20] (\a+\b,\c) rectangle (\a+\b+\c,\b+\c);        % bc

  % Row 3 (height c)
  \fill[green!20] (0,0) rectangle (\a,\c);                      % ca
  \fill[green!20] (\a,0) rectangle (\a+\b,\c);                  % cb
  \fill[purple!20] (\a+\b,0) rectangle (\a+\b+\c,\c);           % c^2


  %================================
  % SECOND: GRID LINES (foreground)
  %================================

  % Outer boundary
  \draw[thick] (0,0) rectangle (\a+\b+\c,\a+\b+\c);

  % Vertical grid lines
  \draw[thick] (\a,0) -- (\a,\a+\b+\c);
  \draw[thick] (\a+\b,0) -- (\a+\b,\a+\b+\c);

  % Horizontal grid lines
  \draw[thick] (0,\c) -- (\a+\b+\c,\c);
  \draw[thick] (0,\b+\c) -- (\a+\b+\c,\b+\c);


  %================================
  % LABELS
  %================================

  % Row labels (left)
  \node[left] at (0, \c/2) {$c$};
  \node[left] at (0, \c + \b/2) {$b$};
  \node[left] at (0, \c + \b + \a/2) {$a$};

  % Column labels (top)
  \node[above] at (\a/2, \a+\b+\c) {$a$};
  \node[above] at (\a+\b/2, \a+\b+\c) {$b$};
  \node[above] at (\a+\b+\c/2, \a+\b+\c) {$c$};

  % Cell labels
  \node at (\a/2, \c+\b+\a/2) {$a^2$};
  \node at (\a+\b/2, \c+\b+\a/2) {$ab$};
  \node at (\a+\b+\c/2, \c+\b+\a/2) {$ac$};

  \node at (\a/2, \c+\b/2) {$ab$};
  \node at (\a+\b/2, \c+\b/2) {$b^2$};
  \node at (\a+\b+\c/2, \c+\b/2) {$bc$};

  \node at (\a/2, \c/2) {$ac$};
  \node at (\a+\b/2, \c/2) {$bc$};
  \node at (\a+\b+\c/2, \c/2) {$c^2$};

\end{tikzpicture}
\end{center}
\end{problem}

\begin{problem} %74
  \begin{gather*}
    (a + b - c)^2 = a^2 + b^2 + c^2 + 2(ab - bc -ac)
  \end{gather*}
\end{problem}

\begin{problem} %75
Consider $(a + b)$ as say $A$ so we have $(A + c) (A - c)$. We get
\begin{gather*}
  (A + c) (A - c) = A^2 - c^2\\
  (a + b)^2 - c^2\\
  a^2 + b^2 - c^2 + 2ab
\end{gather*}
\end{problem}

\begin{problem} %76
  Consider $(b + c)$ as say $B$ so we have $(a + B) (a - B)$. We get
  \begin{gather*}
    (a + B) (a - B) = a^2 - B^2\\
    a^2 - (b + c)^2\\
    a^2 - b^2 - c^2 - 2bc
  \end{gather*}
  
\end{problem}

\begin{problem} %77
  We can change it to $(a + (b - c)) (a - (b - c))$, this is of the form $(a + B) (a - B)$.
  \begin{gather*}
    a^2 - B^2\\
    a^2 - b^2 + c^2 + 2bc
  \end{gather*}
\end{problem}

\begin{problem} %78
  We see a pattern here and can avoid long multiplications
  \begin{gather*}
    (a^2 -2ab + b^2)(a^2 + 2ab + b^2)\\
    (a - b)^2 (a + b)^2\\
    ((a - b)(a + b))^2\\
    (a^2 - b^2)^2\\
    a^4 - 2a^{2}b^{2} + b^4
  \end{gather*}
\end{problem}

%%%%%%%%%%%%%%%%%%%%%%%%%%%%%%%%%%%%%%%%%
% Chapter End
%%%%%%%%%%%%%%%%%%%%%%%%%%%%%%%%%%%%%%%%%


%%%%%%%%%% 25
\section{The cube of the sum formula}

\begin{problem} %79
  Essentially the use of the formula $(a +b)^3 = a^3 + b^3 + 3ab (a +b) $
  \begin{gather*}
    11^3 = (10 + 1)^3\\
    = 10^3 + 1^3 + 3.10.1 (10 + 1)\\
    = 1000 + 1 + 330\\
    = 1331
  \end{gather*}

\end{problem}

\begin {problem} %80
Same as the previous problem.
\begin{gather*}
101^3 = 100^3 + 1^3 + 3.100.1 (100 + 1)\\
= 1000000 + 1 + 30300\\
= 1030301
\end{gather*}

\end{problem}

\begin{problem} %81
  This is again failry simple
  \begin{gather*}
    (a - b)^3 = (a - b)^2 \times (a - b)\\
    = (a^2 + b^2 - 2ab) \times (a - b)\\
    = a^3 + ab^2 - 2a^{2}b -a^{2}b - b^3 + 2ab^2\\
    = a^3 - b^3 + 3ab^2 - 3a^{2}b\\
    = a^3 - b^3 - 3ab(a -b)
  \end{gather*}
  
\end{problem}

%%%%%%%%%%%%%%%%%%%%%%%%%%%%%%%%%%%%%%%%%
% Chapter End
%%%%%%%%%%%%%%%%%%%%%%%%%%%%%%%%%%%%%%%%%


%%%%%%%%%% 26
\section{The formula for $(a + b)^4$}

No problems.

%%%%%%%%%%%%%%%%%%%%%%%%%%%%%%%%%%%%%%%%%
% Chapter End
%%%%%%%%%%%%%%%%%%%%%%%%%%%%%%%%%%%%%%%%%


%%%%%%%%%% 27
\section{Formulas for $(a + b)^5$, $(a + b)^6$,... and Pascal's triangle}

\begin{problem} %82
  For this problem we will use the Pascal's triangle for each of the questions asked.

  \begin{gather*}
    11^3 = (10 + 1)^3 = 10^3 + 3.10^2.1 + 3.10.1^2 + 1^3 = 1000 + 1 + 300 + 30 = 1331
  \end{gather*}
  \begin{gather*}
    11^4 = (10 + 1)^4 = 10^4 + 4.10^3.1 + 6.10^2.1^2 + 4.10.1^3 + 1^4\\
    = 10000 + 4000 + 600 + 40 + 1 = 14641
  \end{gather*}
  \begin{gather*}
    11^5 = (10 + 1)^5 = 10^5 + 5.10^4.1 + 10.10^3.1^2 + 10.10^2.1^3 + 5.10.1^4 + 1^5\\
    = 100000 + 50000 + 10000 + 1000 + 50 + 1 = 161051
  \end{gather*}
  \begin{gather*}
    11^6 = (10 + 1)^6 = 10^6 + 6.10^5.1 + 15.10^4.1^2 + 20.10^3.1^3 + 15.10^2.1^4 + 6.10.1^5 + 1^6\\
    = 1000000 + 600000 + 150000 + 20000 + 1500 + 60 + 1 = 1771561
  \end{gather*}
\end{problem}

\begin{problem} %83
  \begin{gather*}
    (a + b)^7 = a^7 + 7a^6b + 21a^5b^2 + 35a^4b^3 + 35a^3b^4 + 21a^2b^5 + 7ab^6 + b^7
  \end{gather*}
  
\end{problem}

\begin{problem} %84
  In this set of 3 problems the $b$ is replaced by $-b$ and for odd powers we can take that into consideration.
  \begin{gather*}
    (a - b)^4  = a^4 + 4a^3(-b)^1 + 6a^2(-b)^2 + 4a(-b)^3 + (-b)^4\\
    = a^4 -4a^3b + 6a^2b^2 -4ab^3 + b^4
  \end{gather*}

  \begin{gather*}
    (a - b)^5 = a^5 + 5a^4(-b)^1 + 10a^3(-b)^2 + 10a^2(-b)^3 + 5a(-b)^4 + (-b)^5\\
    = a^5 - 5a^4b + 10a^3b^2 - 10a^2b^3 + 5ab^4 - b^5
  \end{gather*}

  \begin{gather*}
    (a - b)^6 = a^6 + 6a^5(-b)^1 + 15a^4(-b)^2 + 20a^3(-b)^3 + 15a^2(-b)^4 + 6a(-b)^5 + (-b)^6\\
    = a^6 - 6a^5b + 15a^4b^2 - 20a^3b^3 + 15a^2b^4 - 6ab^5 + b^6
  \end{gather*}

\end{problem}

\begin{problem} %85
The sum of the coefficients in Pascal's triangle will be a power of 2 as shown in the table below.

\begin{tabular}{ccc}
\toprule
Row & Coefficients  & Sum of Coefficients \\
\midrule
1 & 1 & 1 \\
2 & 1 + 1 & 2 \\
3 & 1 + 2 + 1 & 4 \\
4 & 1 + 3 + 3 + 1 & 8 \\
5 & 1 + 4 + 6 + 4 + 1 & 16 \\
6 & 1 + 5 + 10 + 10 + 5 + 1 & 32\\
7 & 1 + 6 + 15 + 20 + 15 + 6 + 1 &  64\\
$n$ & ... & $2^{n - 1}$\\
\bottomrule
\end{tabular}

\end{problem}

\begin{problem} %86
  In this scenario the variable $a$ or $b$ has a coefficient which is the sum of the coefficients
  in that specific row of the Pascal's triangle.
  \begin{gather*}
    (a + b)^2 = 4a^2\\
    (a + b)^3 = 8a^3\\
    (a + b)^4 = 16a^4
  \end{gather*}
  
\end{problem}

\begin{problem} %87
  Yes, each of the expressions collapses to the sum of the coefficients pertaining to it as per the
  Pascal's triangle.
\end{problem}

\begin{problem} %88
  Every expression turns to $0$ since $a - b = 0$
\end{problem}

%%%%%%%%%%%%%%%%%%%%%%%%%%%%%%%%%%%%%%%%%
% Chapter End
%%%%%%%%%%%%%%%%%%%%%%%%%%%%%%%%%%%%%%%%%

%%%%%%%%%% 28
\section{Polynomials}

\begin{problem} %89
  \begin{gather*}
  (1 + x - y)(12 - zx - y)\\
  = 12 - xz - y + 12x - x^2z - xy - 12y + xyz + y^2\\
  = 12 - xz - 13y + 12x - x^2z - xy + xyz + y^2
  \end{gather*}
  
\end{problem}

\begin{problem} %90
  (a)
  \begin{gather*}
    (1 + x)(1 + x^2) = 1 + x^2 + x + x^3 = 1 + x + x^2 + x^3
  \end{gather*}

  (b)
  \begin{gather*}
    (1 + x)(1 + x^2)(1 + x^3)(1 + x^4) = (1 + x + x^2 + x^3)(1 + x^3 + x^4 + x^7)\\
    = 1 + 0x + 0x^2 + 1x^3 + 1x^4 + + 0x^5 + 0x^6 + 1x^7 + 0x^8 + 0x^9 + 0x^{10}\\
    + 0 + 1x + 0x^2 + 0x^3 + 1x^4 + + 1x^5 + 0x^6 + 0x^7 + 1x^8 + 0x^9 + 0x^{10}\\
    + 0 + 0x + 1x^2 + 0x^3 + 0x^4 + + 1x^5 + 1x^6 + 0x^7 + 0x^8 + 1x^9 + 0x^{10}\\
    + 0 + 0x + 0x^2 + 1x^3 + 0x^4 + + 0x^5 + 1x^6 + 1x^7 + 1x^8 + 0x^9 + 1x^{10}\\
    = 1 + x + x^2 + 2x^3 + 2x^4 + 2x^5 + 2x^6 + 2x^7 + 2x^8 + x^9 + x^{10}
  \end{gather*}
\end{problem}

 (c)
 \begin{gather*}
  (1 + x + x^2 + x^3)^2\\
  = 1 + 1x + 1x^2 + 1x^3 + 0x^4 + 0x^5 + 0x^6\\
  + 0 + 1x + 1x^2 + 1x^3 + 1x^4 + 0x^5 + 0x^6\\
  + 0 + 0x + 1x^2 + 1x^3 + 1x^4 + 1x^5 + 0x^6\\
  + 0 + 0x + 0x^2 + 1x^3 + 1x^4 + 1x^5 + 1x^6\\
  = 1 + 2x + 3x^2 + 4x^3 + 3x^4 + 2x^5 + x^6
 \end{gather*}

 (d)
 This is essentially the previous problem. There is a pattern in this - the coefficients go from 1 to $n$ where $n$ is
 the number of terms. In this case $n = 11$ (1 and all the 10 $x^y$ terms). Then the $n$ goes back in reverse counting
 to 1. Meanwhile the $x^y$ keep increasing. We can simply write the answer as.
 \begin{gather*}
  1 + 2x + 3x^2 + 4x^3 + 5x^4 + 6x^5 + 7x^6 + 8x^7 + 9x^8 + 10x^9 + 11x^{10} + 10x^{11} + 9 x^{12} + 8x^{13}\\
  + 7x^{14} + 6x^{15} + 5x^{16} + 4x^{17} + 3x^{18} + 2x^{19} + x^{20}
 \end{gather*}

 (e) We need not multiply these long polynomials. To get $x^{30}$ there is only one way possible where the $x^{10}$ of the
 three expressions multiply with itself. Thus coefficient of $x^{30}$ is 1.

 For $x^{29}$ too we can reason out. The only way to get 29 is a sum of $(9, 10, 10)$ in various permutations. That is
 only 3 ways $(9, 10, 10)$, $(10, 9, 10)$, and $(10, 10, 9)$. Anything lower that 8 is not possible (we will need
 one more since 8, 10, 10 will give 28).

 (f) This one too we do not need to multiply at all. When multiplying by 1 we would get the same term
 $(1 + x + x^2 ... + x^{10})$ but when we multiply by $-x$ we shift the expression all with minus signs like
 $(- x - x^2 - x^3 - ... - x^{10} - x^{11})$. Adding them all up would just leave the extreme terms intact rest
 will cancel out each other. The answer will be $(1 - x^{11})$.

 (g) This is a quick multiplication and cancelling out of terms
 \begin{gather*}
  = a^3 - a^2b + ab^2 + a^2b - ab^2 + b^3\\
  = a^3 + b^3
 \end{gather*}

 (h) This one we cannot do any quick multiplication. But we can reason it out.

  $= 1 + x + x^2 + x^3 + ... + x^{10}$\\
  $ + 0 - x - x^2 - x^3 - ... - x^{10} - x^{11}$\\
  $ + 0 + 0 + x^2 + x^3 + ... + x^{10} + x^{11} + x^{12}$\\
  $ - ............................................................$\\
  $ + ............................................................$\\
  $................................................................... - x^{18} - x^{19}$\\
  $................................................................... + x^{18} + x^{19} + x^{20}$\\

  All the odd terms will cancel out when added together. Thus the answer is\\
  $1 + x^2 + x^4 + x^6 + x^8 + ...+ x^{18} + x^{20}$

  For some of the problems in this section application of sum of a geometric series makes the problem much easier.
  But the book has not introduced it yet. So we must skip it.


%%%%%%%%%%%%%%%%%%%%%%%%%%%%%%%%%%%%%%%%%
% Chapter End
%%%%%%%%%%%%%%%%%%%%%%%%%%%%%%%%%%%%%%%%%

%%%%%%%%%% 29
\section{A digression: When are polynomials equal?}
\begin{problem} %91
  Put $x = -1$, this makes the left hand as $0$ but the right hand side of the equation is non zero. Thus, these
  two are not equal polynomials.
\end{problem}

\begin{problem} %92
  In this instance $x = 1$ or $x = -1$ does not cut it. But $x = -3$ makes the right hand side zero and left hand side
  as non zero. Thus, these two are not equal polynomials.

\end{problem}

\begin{problem} %93
  $(x + 1)^2 - (x - 1)^2$\\
  Not a good idea. $x = 2$ shows that both sides when evaluated give different values. It is better to expand the left
  hand side since it is simple.
  
\end{problem}

%%%%%%%%%%%%%%%%%%%%%%%%%%%%%%%%%%%%%%%%%
% Chapter End
%%%%%%%%%%%%%%%%%%%%%%%%%%%%%%%%%%%%%%%%%

%%%%%%%%%% 30
\section{How many monomials do we get?}

\begin{problem} %94
  A polynomial with 4 monomials when multiplied by another polynomial with 4 terms will yield 16 monomial terms in total.
\end{problem}

\begin{problem} %95
  Yes, they can yield lower than 16 monomials if the monomial are similar terms.
\end{problem}

\begin{problem} %96
  Not at all possible. I like the recommendation by the author 'If you think so, please reconsider
  the problem several years from now.'
  
\end{problem}

\begin{problem} %97
  This is not a trivial problem for a middle school student. This proof requires a little bit of effort.
  But the student needs to know proof by contradiction. Let us try it.

  Assume there are two polynomials $P(x)$ and $Q(x)$. Both of these two polynomials have at least two non
  zero terms. We can factor out $x$ from each of these polynomials and we assume that the product of these two
  indeed give us just one term. Representing it as below.
  \begin{gather*}
    P(x) \times Q(x) = cx^k\\
    x^a A(x) \times x^b B(x) = cx^k\\
    A(x) \times B(x) = cx^{k - a - b}
  \end{gather*}
  The right hand side of the above equation has to yield a constant $c$ at $x = 0$. It also means the left
  hand side of the equation $A(x) \times B(x)$ is also a constant which is contradictory to our initial statement
  that these two are polynomials with a factored out $x$. Thus, there can be no possible way in which when two
  polynomials with at least two terms are multiplied will yield an answer with only one term.
\end{problem}

\begin{problem} %98
  Yes, it is possible. A good example is given in the book. But how can we prove it? If we can show that
  even a single case exists then we can say that this assertion is true.

  $(x^2 + 2xy + 2y^2)(x^2 - 2xy + 2y^2)$ is the example shown.
  \begin{gather*}
    (x^2 + 2y^2 + 2xy)(x^2 + 2y^2 - 2xy)\\
    (x^2 + 2y^2)^2 - (2xy)^2\\
    x^4 + 4y^4 + 4x^2y^2 - 4x^2y^2\\
    x^4 + 4y^4
  \end{gather*}

\end{problem}

%%%%%%%%%%%%%%%%%%%%%%%%%%%%%%%%%%%%%%%%%
% Chapter End
%%%%%%%%%%%%%%%%%%%%%%%%%%%%%%%%%%%%%%%%%

%%%%%%%%%% 31
\section{Coefficients and values}

\begin{problem} %99

  For $a = 1$ and $b = 1$ we simply get the sum of the coefficients for each row in the Pascal's triangle. That is\\
  1, 2, 4, 8, 16, 32, 64 ... $2^n - 1$ for the $n^{th}$ row.
  
\end{problem}

\begin{problem} %100
  Adding the numbers with alternating signs will make the sum of coefficients $0$ as they will cancel each other out.
  This is happening because the odd powers end up as a minus sign for one of the terms.

\end{problem}

\begin{problem} %101
  The hint is good to solve this problem.
  
  At $x = 1$ the polynomial will be of the form $(1 + 2.1)^{200}$. This gives us the answer $3^{200}$.

  The polynomial can be written as below for $x = 1$
  \begin{gather*}
    P(x) = a_0 + a_1 x + a_2 x^2 + ...\\
    P(x) = a_0 + a_1 + a_2 + ...
  \end{gather*}
\end{problem}

\begin{problem} %102
  Similar to the last problem we put $x = 1$ and get $(1 - 2.1)^{200}$. The answer is 1.
  
\end{problem}

\begin{problem} %103
  Same logic, we substitute $x, y = 1$. We get $(1 + 1 - 1)^3$ which is $1$.
  
\end{problem}

\begin{problem} %104
  For terms not containing $y$ we can simply put $y = 0$ and work it out at $x = 1$.\\
  This gives us $(1 + 1 - 0)^3$ which is 8.
\end{problem}

\begin{problem} %105
  This one is slightly tricky and we should use concepts of sets. There are essentially 4 types of sets of monomials which is possible:
  \begin{itemize}
    \item Constant term that is 1
    \item Only $x$ terms
    \item Only $y$ terms
    \item $xy$ terms
  \end{itemize}

  We know certain sums already
\\

\begin{tabular}{ccc}
\toprule
Terms & Sums of Coefficients \\
\midrule
1 & 1 \\
$x$ & ? \\
$y$ & ? \\
$xy$ & ? \\
all terms & 1 \\
\bottomrule
\end{tabular}
\\

We know that 1 and the $x$ terms only have a sum of 8, therefore only $x$ will be $8 - 1$ which is 7. So let us update the table.
\\

\begin{tabular}{ccc}
\toprule
Terms & Sums of Coefficients \\
\midrule
1 & 1 \\
$x$ & 7 \\
$y$ & ? \\
$xy$ & ? \\
all terms & 1 \\
\bottomrule
\end{tabular}
\\

Now we can figure out the sum of the coefficients of terms not containing $x$. We put $x = 0$.
We get $(1 + 0 - 1)^3$ since $y = 1$ and that is 0. Next we remove the constant 1 from this 0 to get
only the sum from $y$ terms, the answer is $-1$. Updating the table.
\\

\begin{tabular}{ccc}
\toprule
Terms & Sums of Coefficients \\
\midrule
1 & 1 \\
$x$ & 7 \\
$y$ & -1 \\
$xy$ & ? \\
all terms & 1 \\
\bottomrule
\end{tabular}
\\

Finally to get the coefficients for the $xy$ terms we can simply equate the individual terms to the total sum as below.
\begin{gather*}
  1 + 7 - 1 + coeff_{xy}  = 1\\
  coeff_{xy} = - 6
\end{gather*}
\\

So finally to get the sum of coefficients of terms containing $x$ we add the $x$ only sums and $xy$ sums.
This gives $7 - 6$ and the answer is 1.

\end{problem}

%%%%%%%%%%%%%%%%%%%%%%%%%%%%%%%%%%%%%%%%%
% Chapter End
%%%%%%%%%%%%%%%%%%%%%%%%%%%%%%%%%%%%%%%%%

%%%%%%%%%% 32
\section{Factoring}

\begin{problem} %106
  Fairly easy
   \begin{gather*}
    (ac + ad + bc + bd) = a(c + d) + b(c + d)\\
    =(c + d)(a + b)
  \end{gather*}
 
\end{problem}

\begin{problem} %107
  (a)
  \begin{gather*}
    ac + bc - ad - bd = c(a + b) - d(a + b)\\
    (a + b)(c -d)
  \end{gather*}
\end{problem}

(b)
\begin{gather*}
  1 + a + a^2 + a^3 = 1 + a^2 + a + a^3\\
  = (1 + a^2) + a(1 + a^2)\\
  = (1 + a^2)(1 + a)
\end{gather*}

(c) This is not an easy one to get it. One of the ways we can solve is to work backwards from a geometric series.
First let us derive the sum of a geometric series.

 $1.S = 1 + a + a^2 + a^3 + a^4 + ... + a^{13} + a^{14}$\\
 $aS = 0 + a + a^2 + a^3 + a^4 + ... + a^{13} + a^{14} + a^{15}$\\
 
 Subtract the above two.\\

 $(S - aS) = 1 - a^{15}$\\

 $S = \frac{(1 - a^{15})}{(1 - a)}$\\

Now we can do the following manipulations\\

 $S = \frac{(1^3 - (a^5)^3)}{(1 - a)}$\\

 $S = \frac{(1 - a^5)(1^2 + (a^5)^2 + 1.a^5)}{(1 - a)}$\\

 $S = \frac{(1 - a^5)(1 + a^{10} + a^5)}{(1 - a)}$\\

Look at the term $\frac{(1 - a^5)}{(1 - a)}$. This itself is a sum of geometric series up to $a^4$.\\
So we substitute that geometric series.\\

$S = (1 + a + a^2 + a^3 + a^4) (1 + a^5 + a^{10})$

Thus the factorization is $(1 + a + a^2 + a^3 + a^4) (1 + a^5 + a^{10})$

(d) 
\begin{gather*}
  x^4 - x^3 + 2x - 2\\
  x^3(x - 1) + 2(x -1)\\
  (x^3 + 2)(x - 1)\\
\end{gather*}

\begin{problem} %108
\begin{gather*}
  a^2 + 3ab + 2b^2\\
  a^2 + 2ab + b^2 + ab + b^2\\
  (a + b)^2 + b(a + b)\\
  (a + b)(a + 2b)
\end{gather*}
\end{problem}

\begin{problem} %109
  (a)
  \begin{gather*}
    a^2 - 3ab + 2b^2\\
    a^2 - 2ab + b^2 - ab + b^2\\
    (a - b)^2 - b(a - b)\\
    (a - b)(a - 2b)
  \end{gather*}

  (b)
  \begin{gather*}
    a^2 + 3a + 2\\
    a^2 + a + 2a + 2\\
    a(a + 1) + 2(a + 1)\\
    (a + 1) (a + 2)
  \end{gather*}
\end{problem}


\begin{problem} %110
  (a)
  \begin{gather*}
    a^2 + 4ab + 4b^2\\
    (a + 2b)^2
  \end{gather*}

  (b)
  \begin{gather*}
    a^4 + 2a^2b^2 + b^4\\
    (a^2 + b^2)^2
  \end{gather*}

  (c)
  \begin{gather*}
    a^2 - 2a + 1\\
    (a - 1)^2
  \end{gather*}
\end{problem}

\begin{problem} %111
  One thing to know for kids/students
  is that when they see a solution in a book they might wonder and be impressed that in one attempt the book came up with an elegant solution. This inductive
  thinking is not born solely from intelligence but from pattern recognition and pattern recognition in turn comes from practice. And when
  anyone practices they make mistakes, hit roadblocks, turn around and try again till they find a good correct proof. That's all on this at this time.

  Now the problem.

  \begin{gather*}
    x^5 + x + 1\\
    x^5 + x^4 + x^3 + x^2 + x + 1 - x^4 - x^3 - x^2\\
    x^3(x^2 + x + 1) + (x^2 + x + 1) - x^2 (x^2 + x +1)\\
    (x^3 + 1 - x^2) (x^2 + x + 1)
  \end{gather*}

\end{problem}

\begin{problem} %112
  We can simply substitute $b$ instead of $a$ and then $-b$ instead of $a$ in $a^2$. In both cases we will get $b^2$.
\end{problem}

\begin{problem} %113
  In such problems we look at what could be a factor and here the authors point out that when $a = b$ the answer is 0, thus $(a - b)$
  should be a factor. We should divide the given polynomial by $(a - b)$. This gives us $(a^2 + ab + b^2)$. Thus the factor
  is simply $(a - b)(a^2 + ab + b^2)$.
\end{problem}

\begin{problem} %114
  Similar logic as the last problem. In this case both $a$ and $b$ should be $0$ to be a factor. They need to be of opposite signs,
  thus dividing $(a^3 + b^3)$ by $(a + b)$ we get $(a^2 - ab + b^2)$. Thus the factor of $(a^3 + b^3)$ is $(a + b)(a^2 - ab + b^2)$.

\end{problem}

\begin{problem} % 115
  The book has this solved but this is probably not what should be the answer, there is an additional factorization possible.
  \begin{gather*}
    a^4 - b^4\\
    (a^2)^2 - (b^2)^2\\
    (a^2 + b^2) (a^2 - b^2)\\
    (a^2 + b^2)(a + b)(a - b)
  \end{gather*}  
\end{problem}

\begin{problem} %116
  (a)
  Again in this case if $a = b$ we get a factor. So we divide $(a^5 - b^5)$ with $(a - b)$. We get the factors as
  $(a - b)(a^4 + a^3b + a^2b^2 + ab^3 + b^4)$.

  A general formula can be derived for the factorization of $(a^n -b^n)$ seeing the last few questions. This is true for all $n$.

  $(a^n - b^n) = (a - b) (a^{(n - 1)} + a^{(n - 2)}b + a^{(n - 3)}b^2 + ... + ab^{(n - 2)} + b^{(n - 1)})$

  (b) 
  We can use the general formula derived earlier, but first

  $(a^{10}  - b^{10}) = ((a^5)^2 - (b^5)^2) = (a^5 + b^5) (a^5 - b^5)$

  We can also derive the factorization formula for $(a^n + b^n)$ as below. But this is true only when $n$ is odd.

  $(a^n + b^n) = (a + b) (a^{(n - 1)} - a^{(n - 2)}b + a^{(n - 3)}b^2 - ... - ab^{(n - 2)} + b^{(n - 1)})$

  Applying these two to the given problem we get

  $(a^{10}  - b^{10}) = (a + b) (a - b) (a^4 - a^3b + a^2b^2 - ab^3 + b^4) (a^4 + a^3b + a^2b^2 + ab^3 + b^4)$

  (c)

  This can be stated as $(a^7 - 1^7)$, thus we could use the same logic as above

  $(a^7 - 1) = (a - 1) (a^6 + a^5 + a^4 + a^3 + a^2 + a +1)$

  We can now use the sum of the geometric series derived earlier to check our factorization too.

  $(a^7 - 1) = (a - 1) (\frac{1 - a^{6 + 1}}{(1 - a)})$
\\

  An important takeaway from this set of problems are these two factorizations.

\[
(a^n - b^n) = (a - b) (a^{(n - 1)} + a^{(n - 2)}b + a^{(n - 3)}b^2 + ... + ab^{(n - 2)} + b^{(n - 1)})
\]
true for all $n$\\

\[
(a^n + b^n) = (a + b) (a^{(n - 1)} - a^{(n - 2)}b + a^{(n - 3)}b^2 - ... - ab^{(n - 2)} + b^{(n - 1)})
\]
true only when $n$ is odd





\end{problem}

\begin{problem} %117
  \begin{gather*}
    a^2 - 4b^2 = a^2 - (2b)^2 = (a + 2b) (a - 2b)
  \end{gather*}
\end{problem}

\begin{problem} %118
  (a)
  $a^2 - 2 = (a + \sqrt{2})(a - \sqrt{2})$

  (b)
  $a^2 - 3b^2 = (a + \sqrt{3}b)(a - \sqrt{3}b)$

  (c)
  $a^2 + 2ab + b^2 - c^2 = (a + b)^2 - c^2 = (a + b + c) (a + b - c)$

  (d)
  $a^2 + 4ab + 3b^2 = a^2 + 4ab + 4b^2 - b^2 = (a + 2b)^2 - b^2 \\= (a + 2b + b) (a + 2b -b) = (a + 3b) (a + b)$
\end{problem}

\begin{problem} %119
  In this case we could use our earlier derived formula but we are going to use square roots as explained in the solution to this problem.

  \begin{gather*}
    a^4 + b^4 = (a^2 + b^2)^2 - 2a^2b^2 = (a^2 + b^2)^2 - (\sqrt{2}ab)^2\\
    (a^2 + b^2 + \sqrt{2}ab)(a^2 + b^2 - \sqrt{2}ab)
  \end{gather*}
\end{problem}

\begin{problem} %120
  Polynomials of the form $a^{2n} + b^{2n}$ factor only when $n$ has an odd divisor as shown by the authors. If $n$ is not an odd number
  then the factorization can be only done over real numbers such as square roots and/or complex numbers. The authors introduce complex numbers
  immediately after this problem.
\end{problem}

\begin{problem} %121
  \begin{gather*}
    (2 + 3\sqrt{-1})(2 - 3\sqrt{-1})\\
    (2^2 - (3\sqrt{-1})^2)\\
    (4 - (9 \times (-1)))\\
    (4 + 9)\\
    13
  \end{gather*}
\end{problem}

\begin{problem} %122
  The authors say these sets of problems are more difficult.\\

  (a)
  \begin{gather*}
    x^4 + 1 = (x^2 +1)^2 - 2x^2\\
    (x^2 +1)^2 - (\sqrt{2}x)^2\\
    (x^2 + 1 - \sqrt{2}x)(x^2 + 1 + \sqrt{2}x)\\
    ((x + 1)^2 - 2x - \sqrt{2}x)((x + 1)^2 - 2x + \sqrt{2}x)\\
    ((x + 1)^2 - (2 + \sqrt{2})x)((x + 1)^2 - (2 - \sqrt{2})x)
  \end{gather*}

  (b)
  \begin{gather*}
    x(y^2 - z^2) + y(z^2 - x^2) + z(x^2 - y^2)\\
    xy^2 - xz^2 + yz^2 - yx^2 + zx^2 - zy^2\\
    zx^2 -yx^2 + xy^2 - xz^2 + yz^2 - zy^2\\
    x^2(z - y) - x (z^2 - y^2) + zy(z - y)\\
    x^2(z - y) - x (z - y)(z + y) + zy(z - y)\\
    (z - y)(x^2 - x(z + y) + zy)\\
    (z - y)(x - y)(x - z)
  \end{gather*}

  (c)
  This is similar to an earlier problem (number 111)\\
  Let us do some addition of terms to make it consistent across so that we can derive a common term. It is the same
  as division in a way. Arranging the terms basis their powers we can see a clean pattern.\\

  $a^{10} + a^9 + a^8$\\
  $.....- a^9 - a^8 - a^7$\\
  $.................... + a^7 + a^6 + a^5$\\
  $............................ - a^6 - a^5 - a^4$\\
  $.................................... + a^5 + a^4 + a^3$\\
  $.................................................... - a^3 - a^2 -a$\\
  $............................................................ + a^2 + a + 1$

  We can take the term $(a^2 + a + 1)$ out from each row above.\\
  This leaves us with $(a^8 - a^7 + a^5 - a^4 + a^3 - 1 + 1)$\\

  Therefore the factors are $(a^2 + a + 1)(a^8 - a^7 + a^5 - a^4 + a^3 - 1 + 1)$.\\

  The actual method to solve such problems requires cyclotomic factorization which is a part of under graduate abstract algebra so we should skip that.

  (d) 
  We can see that at $a + b + c = 0$ the given polynomial should yield no remainder if divided. Thus $a + b + c$ should be a factor. This is
  similar to the reasoning in the problem 113 and 114.\\

  Now this $a + b + c$ has to at least multiply by $a^2 + b^2 + c^2$ and other terms so that we end up with $-3 abc$
  while other terms such as $ab^2 + ac^2$ cancel out. This is also fairly visible if we have the term $(- ab - bc - ca)$.
  Finally, we get\\
  \[
  a^3 + b^3 + c^3 - 3abc = (a + b + c)(a^2 + b^2 + c^2 - (ab + bc + ca))
  \]

  (e)
  This is an easy problem actually after a series of difficult problems.\\

  Let us expand $(a + b + c)^3$ in a way where we initially treat $(b + c)$ as one term. So we have to basically do
  $(a + k)^3$ which is $(a^3 + k^3 + 3ak(a + k))$. Now put back $(b + c)$ instead of $k$. $(a^3 + (b + c)^3 + 3a(b + c)(a + b +c))$ is what we get.
  Further expanding\\
\begin{gather*}
    (a + b + c)^3 = a^3 + b^3 + c^3 + 3bc(b + c) + 3a(b + c)(a + b +c)\\
    (a + b + c)^3 = a^3 + b^3 + c^3 + 3(b + c) (bc + a(a + b +c))\\
    (a + b + c)^3 = a^3 + b^3 + c^3 + 3(b + c) (bc + a^2 + ab + ac)\\
    (a + b + c)^3 = a^3 + b^3 + c^3 + 3(b + c) (a^2 + a(b + c) +bc)\\
    (a + b + c)^3 = a^3 + b^3 + c^3 + 3(b + c) (a + b) (a + c)\\
    (a + b + c)^3 - a^3 - b^3 - c^3 = 3(b + c) (a + b) (a + c)
\end{gather*}

(f)
This is another easy problem and we can see an easy substitution will collapse this problem's solution into a few lines.\\

Substitute the following\\
$P = a - b$\\
$Q = b - c$\\
$R = c - a$\\

We can transform the given polynomial now:
\begin{gather*}
  P^3 + Q^3 + R^3
  = (P + Q + R)^3 - 3 (P + Q) (Q + R)(R + P)\\
  = (a - b + b - c + c - a)^3 - 3 (a - b + b - c)(b - c + c - a)(c - a + a - b)\\
  = 0^3 - 3 (a - c) (b - a) (c - b)\\
  = 3 (a - b)(b - c)(c - a)
\end{gather*}
\end{problem}

\begin{problem} %123
  A hint is given which leads us to in an easy factorization.\\

  \begin{gather*}
    (a + b) < 1 + ab\\
    1 + ab - (a + b) > 0\\
    (1 - a) (1 - b) > 0
  \end{gather*}

  Now both $a$ and $b$ are greater than 1 then both $(1 - a)$ and $(1 - b)$ when multiplied will always be positive. Hence proved.

\end{problem}

\begin{problem} %124
  We know the below factorization.\\

  \[
  a^3 - b^3 = (a - b) (a^2 + ab + b^2)
  \]

  If $(a^2 + ab + b^2) = 0$ which is the right hand side then the left hand side will also be $0$. Thus,
  \[
  a^3 - b^3 = 0
  \]
  In this case $a = b = 0$ since they are both odd powers.

\end{problem}

\begin{problem} %125
  This is an easy problem. We have already derived it earlier.

  \[
  (a + b + c)^3 = a^3 + b^3 + c^3 + 3(a + b)(b + c)(c + a)
  \]

  If $a + b + c = 0$ then we can also write:\\
  $a + b = - c$\\
  $b + c = - a$\\
  $c + a = - b$\\

  Putting these three into the above equation we get\\
  \[
  0^3 = a^3 + b^3 + c^3 + 3 (- c)(- a)(- b)
  \]
  Thus,
  \[
  a^3 + b^3 + c^3 = 3abc
  \]
\end{problem}

\begin{problem} %126
  Simplifying both the sides\\

  \[
  abc = (a + b + c)(ab + bc + ca)
  \]
  We observe a pattern on the right hand side
  \begin{gather*}
    abc = 3abc + (a^2b + a^2c + b^2a + b^2c + c^2a + c^2b)
  \end{gather*}
  The term $a^2b + a^2c + b^2a + b^2c + c^2a + c^2b$ should equate to $-2abc$.\\
  Given that $a = - b and a = - c$ then substituting $b = - a$ and $c = - a$\\
  $= a^2(- a - a) + (- a)^2 (a - a) + (- a)^2 (a - a)$\\
  $= -2a^3$\\
  This is same as $-2abc$ because $-2a(-a)(-a)$ i.e. $-2a^3$. Same holds true for other two combinations.
\end{problem}

%%%%%%%%%%%%%%%%%%%%%%%%%%%%%%%%%%%%%%%%%
% Chapter End
%%%%%%%%%%%%%%%%%%%%%%%%%%%%%%%%%%%%%%%%%

%%%%%%%%%% 33
\section{Rational expressions}
No problems

%%%%%%%%%%%%%%%%%%%%%%%%%%%%%%%%%%%%%%%%%
% Chapter End
%%%%%%%%%%%%%%%%%%%%%%%%%%%%%%%%%%%%%%%%%

%%%%%%%%%% 34
\section{Converting a rational expression into the quotient of two polynomials}

\begin{problem} %127
  The objective is to get a fraction in which both numerator and denominator are polynomials.

  (b) $\dfrac{ac}{b^2}$\\

  (c) $\dfrac{x}{(1 + x)}$\\

  (d)
  \[
  \dfrac{1}{1 + \dfrac{1}{1 + \dfrac{1}{1 + \dfrac{1}{x}}}}
  \]

  \[
  \dfrac{1}{1 + \dfrac{1}{1 + \dfrac{x}{x + 1}}}
  \]

  \[
  \dfrac{1}{1 + \dfrac{x + 1}{2x + 1}}
  \]

  \[
  \dfrac{2x + 1}{3x + 2}
  \]

  (e)
  \[
  \dfrac{\dfrac{x}{y} + \dfrac{y}{z} + \dfrac{z}{x}}{\dfrac{y}{x} + \dfrac{z}{y} + \dfrac{x}{z}} + 1
  \]

  \[
  \dfrac{x^2z + y^2x + z^2y}{y^2z + z^2x + x^2y} + 1
  \]

  \[
  \dfrac{x^2z + x^2y + y^2x + y^2z + z^2x + z^2y}{y^2z + z^2x + x^2y}
  \]

  (g)

  \[
  \dfrac{1}{\left(\dfrac{\frac{1}{a} + \frac{1}{b}}{2}\right)}
  \]

  \[
  \dfrac{2ab}{a + b}
  \]
\end{problem}

\begin{problem} %128
  \[
  \dfrac{(x - a)(x - b)}{(c - a)(c - b)} + \dfrac{(x - a)(x - c)}{(b - a)(b - c)} + \dfrac{(x - b)(x - c)}{(a - b)(a - c)}
  \]

  In the book the authors expand right at the start. We should defer it to the end as far as possible. A lot of terms
  cancel out in basic manipulations itself.

  Numerator when cancels out with denominator:

  \[
  (x - a)(x - b)(a - b)\cancel{(b - c)}\cancel{(a - b)}\cancel{(a - c)}
  \]
  \[
  + (x - a)(x - c)\cancel{(c - a)}\cancel{(c - b)}\cancel{(a - b)}(c - a)
  \]
  \[
  + (x - b)(x - c)\cancel{(c - a)}(b - c)\cancel{(b - a)}\cancel{(b - c)}
  \]

  The denominator becomes:

  \[
  \cancel{(c - a)}(c - b)(b - a)\cancel{(b - c)}\cancel{(a - b)}(a - c)
  \]

  Working on the numerator now:
  \[
  (x - a)(x - b)(a - b) + (x - a)(x - c)(c - a) + (x - b)(x - c)(b - c)
  \]

  There is a certain symmetry we observe here. The $a$ and $b$ logically grouped with $x$, so with the other combinations.\\
  We notice that $x^2$ on the first addition term yields $ax^2 - bx^2$, this will cancel out with the $x^2$ terms in the next two operands between the two addition signs. Thus all $x^2$s cancel out.
  Similarly we also notice all the $x$ terms also cancel out through, we are left with a compact expression in the numerator now where we can easily factorize.\\
  
  \[
  a^2b - ab^2 + ac^2 - a^2c + b^2c - bc^2
  \]

  Assume $a^2$ to be the unknown for the quadratic factorization then
  \[
  a^2(b - c) + a(c^2 - b^2) + (cb^2 - bc^2)
  \]

  Take $(c - b)$ out.

  \[
  (c - b)[- a^2 + a (c + b) - bc]
  \]

  \[
  (c - b) (b - a) (a - c) 
  \]

  Both the numerator and denominator are same. Thus, the answer is 1.
\end{problem}

\begin{problem} %129
  Case $x = a$: The first two terms in the addition becomes $0$. The last term evaluates to $1$.\\

  Case $x = b$: Same logic as above. The expression evaluates to $1$.\\

  Case $x = c$: Same logic as above. The expression evaluates to $1$.\\
  
\end{problem}


\begin{problem} %130
  These problems should be solved in general terms as that does not allow any confusion to creep in.

  Let the volume of each half of the pool be $X$, rate of flow in first half be $p$ and second half be $q$. Thus
  the following equations hold true:

  \begin{gather*}
    X = ap\\
    Y = bq
  \end{gather*}

  To fill the full tank we can frame the equation:
  
  \[
    2X = t \times (p + q)
  \]
  where $t$ is the total time taken to fill the pool. Now substituting the initial equations into this we get.

  \[
    2X = t \times (\dfrac{X}{a} + \dfrac{X}{b})
  \]
  \[
    t = \dfrac{2a}{a + b}
  \]
\end{problem}

\begin{problem} %131
  This is similar to the previous problem.

  Assume the length from point $A$ to $B$ be $D$ and the speed of the motor boat be $v$ and river be $r$. Then we can formulate the following equations.

  \begin{gather*}
    D = (v + r)a\\
    D = (v - r)b  
  \end{gather*}

  Students should note that while going with the stream the boat's speed is added to the river's speed. But while going against the 
  current of the river the river's speed needs to be deducted from the boat's speed. A side question is what if the speed of the boat was lower than that
  of the river and what if in another case it was equal?

  Now when we need to find the time taken for the boat to travel $D$ when the speed of the river is $0$ then we have to find $t$ in the below equation.

  \[
  D = t \times v
  \]

  From the earlier two equations equating $r$ we get

  \[
  \dfrac{D}{a} - v = v - \dfrac{D}{b}
  \]
  \[
  D = 2\left(\dfrac{1}{a} + \dfrac{1}{b}\right) V
  \]
  Thus we now have
  \[
  t = 2\left(\dfrac{1}{a} + \dfrac{1}{b}\right)
  \]

\end{problem}

\begin{problem} %132
  This is again similar to earlier problems. Assume half the trip is $D$ distance and it takes $t_1$ time while
  the other half takes $t_2$ time. We can write
  \begin{gather*}
    2D = \left(\dfrac{D}{v_1} + \dfrac{D}{v_2}\right)v\\
    v = \dfrac{2v_1v_2}{v_1 + v_2}
  \end{gather*}
\end{problem}

\begin{problem} %133
  (a)
  \begin{gather*}
  (x + \dfrac{1}{x})^2 = x^2 + \dfrac{1}{x^2} + 2\\
  7^2 = x^2 + \dfrac{1}{x^2} + 2\\
  x^2 + \dfrac{1}{x^2} = 49 - 2 = 47
  \end{gather*}

  (b)
  \begin{gather*}
    (x + \dfrac{1}{x})^3 = x^3 + \dfrac{1}{x^3} + 3(x + \dfrac{1}{x})\\
    x^3 + \dfrac{1}{x^3} = (x + \dfrac{1}{x})^3 - 3(x + \dfrac{1}{x})\\
    x^3 + \dfrac{1}{x^3} = 7^3 - 3 \times 7\\
    x^3 + \dfrac{1}{x^3} = 322
  \end{gather*}
\end{problem}

\begin{problem} %134
  We can use the general expansion of the expression $(x + y)^n$ and look at the coefficients from the Pascal's triangle.
  
  \[
    (x + y)^n = x^n + C_1x^{n - 1}y + C_2x^{n - 2}y^2 + ... + C_{n - 1}xy^{n - 1} + y^n
  \]

  One important point to note is that as per Pascal's triangle if $n$ is odd then all the coefficients exist in pairs, for
  instance $C_1$ will be same as $C_{n - 1}$. But if $n$ is even then the middle term's coefficient will not have a pair. But this
  does not matter since that coefficient is an integer itself. Going back to the above equation and substituting $\dfrac{1}{x}$ instead of $y$.

  \[
    (x + \dfrac{1}{x})^n = x^n + C_1x^{n - 1}\dfrac{1}{x} + C_2x^{n - 2}\dfrac{1}{x^2} + ... + C_{n - 1}x\dfrac{1}{x^{n - 1}} + \dfrac{1}{x^n}
  \]

  rearrange the terms

  \[
    (x + \dfrac{1}{x})^n = x^n + \dfrac{1}{x^n} + C_1x^{n - 1}\dfrac{1}{x} + C_{n - 1}x\dfrac{1}{x^{n - 1}} + C_2x^{n - 2}\dfrac{1}{x^2} + C_{n - 2}x^2\dfrac{1}{x^{n - 2}} ... 
  \]

  \[
    (x + \dfrac{1}{x})^n = x^n + \dfrac{1}{x^n} + C_1x^{n - 2} + C_{n - 1}\dfrac{1}{x^{n - 2}} + C_2x^{n - 4} + C_{n - 2}\dfrac{1}{x^{n - 4}} ... 
  \]

  We also know that the corresponding coefficients (from the earlier statement) exists in parirs/middle one for even $n$ is alone.
  Thus $C_1 = C_{n - 1}$ and so on. We reduce the equation to the following.\\
  \[
    (x + \dfrac{1}{x})^n = x^n + \dfrac{1}{x^n} + C_1\left(x^{n - 2} + \dfrac{1}{x^{n - 2}}\right) + C_2\left(x^{n - 4} + \dfrac{1}{x^{n - 4}}\right) ...
  \]

  The smallest term in this equation will be $x^2 + \dfrac{1}{x^2}$. This we can show is an integer already.
  \[
    (x + \dfrac{1}{x})^2 = x^2 + \dfrac{1}{x^2} + 2
  \]
  \[
    x^2 + \dfrac{1}{x^2} = (x + \dfrac{1}{x})^2 - 2
  \]
  The right hand side is an integer minus $2$. Now we can look at $x^4 + \dfrac{1}{x^4}$.
  \[
    (x^2 + \dfrac{1}{x^2})^2 = x^4 + \dfrac{1}{x^4} + 2
  \]
  \[
    x^4 + \dfrac{1}{x^4} = (x^2 + \dfrac{1}{x^2})^2 - 2
  \]
  The right hand side here too is an integer. Thus, we can go down the rabbit hole and say as we go up in the higher powers of 
  even numbers the term $x^{2k} + \dfrac{1}{x^{2k}}$ will always be an integer for a given integer $k > 0$.

  Thus the original expression collapses to the following:
  \[
    (x + \dfrac{1}{x})^n = x^n + \dfrac{1}{x^n} + C_1\left(x^{n - 2} + \dfrac{1}{x^{n - 2}}\right) + C_2\left(x^{n - 4} + \dfrac{1}{x^{n - 4}}\right) ...
  \]
  \[
    integer_1 = x^n + \dfrac{1}{x^n} + integer_2
  \]
  Hence we have proved that $x^n + \dfrac{1}{x^n}$ is always an integer.

  Please note the proper way of proving this is via mathematical induction but the authors have not introduced this technique yet.
\end{problem}

\begin{problem} %135
  The pattern is that of a \href{https://en.wikipedia.org/wiki/Fibonacci_sequence}{Fibonacci} sequence. For the two instances we simply subsititute the given 
  $\dfrac{2x + 1}{3x + 2}$ first and get the first answer then substitute that answer once more to get the second answer.\\

  The first answer is $\dfrac{3x + 2}{5x + 3}$\\

  The second answer is $\dfrac{5x + 3}{8x + 5}$\\

  The general term can be got as the $k_{th}$ term in a Fibonacci sequence.
  \[
    \dfrac{k_{n - 1}x + k_{n - 2}}{k_nx + k_{n - 1}}
  \]
\end{problem}

%%%%%%%%%%%%%%%%%%%%%%%%%%%%%%%%%%%%%%%%%
% Chapter End
%%%%%%%%%%%%%%%%%%%%%%%%%%%%%%%%%%%%%%%%%

%%%%%%%%%% 35
\section{Polynomial and rational fractions in one variable}

\begin{problem} %136
  We need to only worry about the highest degree term and that will be $(2x)^5$, that is $32x^5$.
\end{problem}

\begin{problem} %137
  Polynomials $P$ has a degree $m$, thus the highest degree term will be something like $C_1x^m$. Similarly the other polynomial
  $Q$ will have its highest degree term as $C_2x^n$. The product of the polynomials $P \times Q$ will have its highest degree term as
  $C_1.C_2x^{m + n}$.
\end{problem}

\begin{problem} %138
  (a) Since it is the sum of the two polynomials and one of them is higher than the other we are sure that the highest degree of the resultant
  polynomial will be 9.

  (b) Since both the polynomials have the same degree there is one possibility that they cancel each other out if the coefficients are equal but
  of opposite signs. Thus the degree of the resultant polynomial will be 7 or less.
\end{problem}

\begin{problem} %139
  In this case what we are essentially doing is $(y^7)^{10}$. Thus the degree of the resultant polynomial will be $70$.
\end{problem}

%%%%%%%%%%%%%%%%%%%%%%%%%%%%%%%%%%%%%%%%%
% Chapter End
%%%%%%%%%%%%%%%%%%%%%%%%%%%%%%%%%%%%%%%%%

%%%%%%%%%% 36
\section{Division of polynomials in one variable; the remainder}

\begin{problem} %140
  The quotient will have a degree of 4 which is got by $\dfrac{x^7}{x^3}$ whereas the remainder will be a proper fraction
  such that the numerator will be of a degree which is at the maximum equal to 4 or lower. So the range of the degree of the
  remainder polynomial will be from $0$ to $1$.
\end{problem}

\begin{problem} %141
  The solution is given in the book. $P$ is divided by $S$ which gives the following.
  \begin{gather*}
    P = Q_1S + R_1\\
    P = Q_2S + R_2
  \end{gather*}

  $R_1$ and $R_2$ will have a degree smaller than $S$. Then 
  \[
  Q_1S + R_1 = Q_2S + R_2
  \]
  \[
  R_1 - R_2 = (Q_2 - Q_1)S
  \]

  $R_1 - R_2$ has a smaller degree than $S$ since individually too they have a smaller degree. They cannot be a multiple of $S$ either. Therefore,
  $Q_1 - Q_2 = 0$, that is $Q_1 = Q_2$ and thus $R_1 = R_2$.
\end{problem}

\begin{problem} %142
  In this case the fraction cannot be further divided as such. The quotient is 0 and the remainder is the dividend itself.
\end{problem}

\begin{problem} %143
  (a)
  \[
  \polylongdiv{x^3 - 1}{x - 1}
  \]

  (b)
  \[
  \polylongdiv{x^4 - 1}{x - 1}
  \]

  (c)
  \[
  \polylongdiv{x^{10} - 1}{x - 1}
  \]

  (d)
  \[
  \polylongdiv{x^3 + 1}{x + 1}
  \]

  (e)
  \[
  \polylongdiv{x^4 + 1}{x + 1}
  \]
\end{problem}

\begin{problem} %144
  This one is just a specific case of the formula, a geometric progression where $b = 1$.

  \[
   (a^n - b^n) = (a - b) (a^{(n - 1)} + a^{(n - 2)}b + a^{(n - 3)}b^2 + ... + ab^{(n - 2)} + b^{(n - 1)})
  \]
  \[
   (a^n - 1) = (a - 1) (a^{(n - 1)} + a^{(n - 2)} + a^{(n - 3)} + ... + a + 1)
  \]

  Let us rearrange the equation.

  \[
    (1 + a + a^2 + a^3 + a^4 + ... + a^{n - 2} + a^{n - 1}) = \dfrac{(a^n - 1)}{(a - 1)}
  \]

  Let us substitute $a = 2$ in the above equation.

  \[
    (1 + 2 + 2^2 + 2^3 + 2^4 + ... + 2^{n - 2} + 2^{n - 1}) = \dfrac{(2^n - 1)}{(2 - 1)} = (2^n - 1)
  \]

  We can see the left hand side is the summation of all the powers of 2 up to the number $(n - 1)$ and the right hand side is one less
  than $2^n$.
\end{problem}

%%%%%%%%%%%%%%%%%%%%%%%%%%%%%%%%%%%%%%%%%
% Chapter End
%%%%%%%%%%%%%%%%%%%%%%%%%%%%%%%%%%%%%%%%%

%%%%%%%%%% 37
\section{The remainder when dividing by $x - a$}

\begin{problem} %145
  (a) For all powers of $x$ we will get $x^n = 1$ when we substitute $x = 1$.

  (b) For only odd powers of $x$ we will get $x^n = -1$ when we substitute $x = -1$.
\end{problem}

\begin{problem} %146
  (a)
  We can easily observe that at $x = 1$ and $x = -2$ the polynomial will yield $0$. So we know $(x - 1)$ and $x - 2$ will
  be factors. Then dividing the polynomial with these two factors.
  \[
  \polylongdiv{x^4 + 0x^3 + 0x^2 + 5x - 6}{x^2 + x - 2}
  \]

  Thus the factorization is
  \[
    x^4 + 5x - 6 = (x - 1)(x + 2)(x^2 - x + 3)
  \]
  If we try to factorize $(x^2 - x + 3)$ we will end up with complex roots to this quadratic so we will skip that.


  (b)
  Here we see all terms are positive so we should look for negative numbers which would turn the polynomial into $0$.
  The first factor we can see is $(x + 1)$. Now we will divide the polynomial by this factor.
  \[
  \polylongdiv{x^4 + 3x^2 + 5x + 1}{x + 1}
  \]
  Now we have to figure out whether the polynomial $x^3 - x^2 + 4x + 1$ can be further factorized or not. By visual inspection
  we can deduce that any integer from $-5$ to $+5$ will not turn the polynomial $0$.
  
  So the factorization is
  \[
  x^4 + 3x^2 + 5x + 1 = (x + 1)(x^3 - x^2 + 4x + 1)
  \]

  (c)
  The visual inspection immediately gives is the following two factors $(x + 1)$ and $(x - 2)$. Dividing now.

  \[
  \polylongdiv{x^3 - 3x - 2}{x^2 - x -2}
  \]
  
  Thus the factorization is
  \[
  x^3 - 3x - 2 = (x + 1)^2 (x - 2)
  \]
\end{problem}

\begin{problem} %147
  The proof is given in the book with the common error made when highlighted. We have used this logic in the earlier problem
  already when we divided the polynomial by the product of all its known factors. The authors state that $P = (x - 1)Q$. Then
  they substitute 2 and find that 2 is a root of $Q$. Finally arriving at the expression $P = (x - 1)(x - 2)R$.
\end{problem}

\begin{problem} %148
  This is simply the highest degree of the polynomial. The authors show this with an example. But we can simply state the following.

  \[
  P(x) = (x - n_1)(x - n_2)(x - n_3)(x - n_4) ... 1
  \]
  Here the $R$ equivalent polynomial is of 0 degree and the constant is 1.
\end{problem}

\begin{problem} %149
  We simply check if the polynomial gives a remainder of 0 when divided by $(x - 1)$ and $(x + 1)$. The other easier way
  is to substitute $1$ and $-1$ in the polynomial and see if they are roots.
\end{problem}

\begin{problem} %150
  This is only possible when $n$ is even. We can do a short proof of the same but it is fairly visible when we look at the problem.

  If $n$ was even we see the proof below and it is correct.
  \[
  x^n - 1 = x^{2k} - 1 = (x^k - 1) (x^k + 1)
  \]

  If $n$ was odd then.

  \[
  x^n - 1 = x^{2m + 1} - 1
  \]
  For $(x - 1)$ the right hand side will yield $0$ but $(x + 1)$ will yield $-2$. Hence for $n$ as an odd number this will not
  hold true.
\end{problem}

\begin{problem} %151
  We can substitute $x = 1$ and $x = -1$ in the equation $P(x) = (x^2 - 1) + (ax + b)$. This will give us two linear equations in
  $x$. We can solve them simultaneously and arrive at the values of $a$ and $b$.
\end{problem}

\begin{problem} %152
  We can write the given division as
  \[
  P = Q (x^2 - 1) + (5x - 7) = Q (x - 1)(x + 1) + (5x - 7)
  \]

  One way to look at this is that $Q (x + 1)$ is a polynomial in $x$ and serves as the quotient when divided by $(x - 1)$.
  In this case the remainder is $(5x - 7)$.

  The other way is to divide by $(x + 1)$. Thus we need to do the below division to arrive at the remainder.

  \[
  \polylongdiv{5x - 7}{x - 1}
  \]

  We cam write that as
  \[
  P = \left(Q (x + 1)\right) (x - 1) + 5(x - 1) - 2 = \left(Q (x + 1)\right) (x - 1) + 5x - 7
  \]
  Here too we get the remainder as $(5x - 7)$.

\end{problem}

\begin{problem} %153
  (a)

  We know as the authors assure us that when we substitute $x_1$ in the polynomial we will get $0$ (a root). Thus let us make the
  a polynomial with the new roots from the old one.

  \[
  (x - x_1)(x - x_2)(x - x_3) = x^3 + x^2 - 10x + 1
  \]

  In this let us put the new roots. But in doing so we want our right hand side to also adjust so that when we substitute the root the resulting
  polynomial is $0$. So for the root $(x_1 + 1)$ we will have to subtract $1$ from the $x$s in the polynomial.

  \[
  (x - (x_1 + 1))(x - (x_2 + 1))(x - (x_3 + 1)) = (x - 1)^3 + (x - 1)^2 - 10(x - 1) + 1
  \]

  The right hand side can be simplified to the following.
  \[
  x^3 - 2x^2 - 9x + 1
  \]

  (b)

  Similar logic as the previous one but in this case we will need to divide by $2$ in the polynomial for all the$x$ terms.

  \[
  \left(\dfrac{x}{2}\right)^3 + \left(\dfrac{x}{2}\right)^2 - 10\left(\dfrac{x}{2}\right) + 1
  \]

  \[
  \dfrac{x^3}{8} + \dfrac{x^2}{4} - 5x + 1
  \]

  Getting integer coefficients

  \[
  x^3 + 2x^2 - 40x + 8
  \]

  (c)

  Same logic here but we take reciprocal in the polynomial.

  \[
  \dfrac{1}{x^3}+ \dfrac{1}{x^2} - \dfrac{10}{x} + 1
  \]

  Getting integer coefficients

  \[
  x^3 - 10x^2 + x + 1
  \]
  
\end{problem}

\begin{problem} %154
  The immediate thing to visually see is that $(x^2 - 3x + 2)$ is $(x - 1)(x - 2)$. Thus if we substitute 1 or 2 in the cubic polynomial
  we should get 0. Thus we will have our two equations in two unknowns and we can solve for $a$ and $b$.

  \begin{gather*}
    1 + a + 1 + b = 0\\
    8 + 4a + 2 + b = 0
  \end{gather*}

  These two give us $a = -\dfrac{8}{3}$ and $b = \dfrac{2}{3}$.
\end{problem}

%%%%%%%%%%%%%%%%%%%%%%%%%%%%%%%%%%%%%%%%%
% Chapter End
%%%%%%%%%%%%%%%%%%%%%%%%%%%%%%%%%%%%%%%%%

%%%%%%%%%% 38
\section{Values of polynomials, and interpolation}

\begin{problem} %155
  The table is below\\

\begin{tabular}{ccc}
\toprule
$P(x)$ & $x^3 - 2$  & value \\
\midrule
0 & $0^3 - 2$ & -2 \\
1 & $1^3 - 2$ & -1 \\
1 & $2^3 - 2$ & 6 \\
3 & $3^3 - 2$ & 25 \\
4 & $4^3 - 2$ & 62 \\
5 & $5^3 - 2$ & 123 \\
6 & $6^3 - 2$ & 214 \\
\bottomrule
\end{tabular}
\\
\end{problem}

\begin{problem} %156
  This is just a substitution problem. We can either plug in numbers and show that successive differences will result in
  a constant difference in this case 2. Or we can consider a generic number $k$ and show a generic answer. Let us work with $k$.

\begin{tabular}{c c c c}
\toprule
$P(x)$ & $x^2 - x - 4$  & Expansion & Value \\
\midrule
$k - 1$ & $(k - 1)^2 - (k - 1) - 4$ & $k^2 + 1 - 2k - k + 1 - 4$ & $k^2 - 3k -2$ \\
$k$ & $k^2 - k - 4$ & $k^2 - k - 4$ & $k^2 - k - 4$ \\
$k + 1$ & $(k + 1)^2 - (k + 1) - 4$ & $k^2 + 1 + 2k - k - 1 - 4$ & $k^2 + k - 4$ \\
$k + 2$ & $(k + 2)^2 - (k + 2) - 4$ & $k^2 + 4 + 4k - k - 2 - 4$ & $k^2 + 3k - 2$ \\
\bottomrule
\end{tabular}
\\

Now we just take first successive differences.\\

\begin{tabular}{c c}
\toprule
First Difference & Value \\
\midrule
$k^2 - k - 4 - k^2 + 3k + 2$ & $2k - 2$\\
$k^2 + k - 4 - k^2 + k + 4$ & $2k$\\
$k^2 + 3k - 2 - k^2 - k + 4$ & $2k + 2$\\
\bottomrule
\end{tabular}
\\

The value columns already shows that the difference in these terms is $2$. So if we take another set of differences in these values
we will get the series as a constant 2.
\end{problem}

\begin{problem} %157
  This is a generalization of the above solution. Assume the polynomial is $ax^2 + bx +c$. Then we can build a similar table.

\begin{tabular}{c c c}
\toprule
$P(x)$ & $ax^2  bx + c$  & Value \\
\midrule
$k - 1$ & $a(k - 1)^2 + b(k - 1) + c$ & $ak^2 - 2ak + bk + a - b + c$ \\
$k$ & $ak^2 + bk + c$ & $ak^2 + bk + c$ \\
$k + 1$ & $a(k + 1)^2 + b(k + 1) + c$ & $ak^2 + 2ak + bk + a + b + c$ \\
$k + 2$ & $a(k + 2)^2 + b(k + 2) + c$ & $ak^2 + 4ak + bk + 4a + 2b + c$ \\
\bottomrule
\end{tabular}
\\

Now we just take first successive differences.\\

\begin{tabular}{c c}
\toprule
First Difference & Value \\
\midrule
$ak^2 + bk + c - ak^2 + 2ak - bk - a + b - c$ & $2ak - a + b$\\
$ak^2 + 2ak + bk + a + b + c - ak^2 - bk - c$ & $2ak + a + b$\\
$ak^2 + 4ak + bk + 4a + 2b + c - ak^2 - 2ak - bk - a - b - c$ & $2ak + 3a + b$\\
\bottomrule
\end{tabular}
\\

The value column can again be observed for successive differences which is
\[
2ak + a + b - (2ak - a + b)
\]
and
\[
2ak + 3a + b - (2ak + a + b)
\]

The above two differences yield a constant difference of $2a$. Thus we have proved all differences will be $2a$.
\end{problem}

\begin{problem} %158
  In case of a polynomial with degree 3 we can deduce that the "third difference" will be a constant difference.
  We can quickly verify this with the a test third degree polynomial $(x + 1)^3$.

  First difference gives us the following values.
  \[
  3x^2 + 3x + 1
  \]
  \[
  3x^2 + 9x + 7
  \]
  \[
  3x^2 + 15x + 19
  \]

  Then the second difference gives us
  \[
  6x + 6
  \]
  \[
  6x + 9
  \]

  Finally the "third difference" is a constant of 3.

  Actually we can prove the same with a generic third degree polynomial such as $ax^3 + bx^2 + cx + d$ but the above
  deduction is sufficient.
\end{problem}

\begin{problem} %159
  This is a classic number theory problem and I will show it with Lisp code too. Pleasantly surprised to see this problem in this book.
  Would love it when my daughter and son come across this for the first time in their lives. I hope they admire the
  mathematics of Euler as much as I do. 

  \begin{quote}
    \itshape
    ``Euler, the master of us all.''
    
    \hfill --- Carl Friedrich Gauss
  \end{quote}

  We can test whether a number is prime if we can check whether the greatest divisor for that number is the number itself and no other divisor greater
  than qual to 2 exists. This algorithm has probably been used for centuries.

  So we will check for every value of $n$ starting from 1 and go on. Manually it is very easy to check up to $n = 10$ since that results in
  only 151 and we notice that all numbers ${43, 47, 53, 61, 71, 83, 97, 113, 131, 151}$ are prime. But a general solution is pretty difficult.
  In fact this was Problem number \#6 in the International Math Olympiad in 1987! Stating the problem below:\\

  Let $n$ be an integer greater than or equal to $2$. Prove that if $k^2 + k + n$ is prime for all integers $k$ such
  that $0 \le k \le \sqrt{n / 3}$, then $k^2 + k + n$ is prime for all integers $k$ such that $0 \le k \le n - 2$.

  We know the answer but now let us make our computer wizard do the job of computation. Here is the Scheme/Lisp code and the output generated.

\begin{lstlisting}

#lang sicp

;; check whether a value of a given polynomial is prime or not


;; polynomial given x^2 + x + 41
(define (poly x)
  (+ (* x x) x 41))

;; print value of polynomial and whether it's prime or not
(define (print-polyval-prime from to)
  (define (iter n)
    (if (> n to)
        (quote done)
        (let ((value (poly n)))
          (display "n = ") (display n)
          (display "    f(n) = ") (display value)
          (display "    prime? -> ") (display (if (prime? value) "yes" "no"))
          (newline)
          (iter (+ n 1)))))
  (iter from))

;; Prime testing procedure
(define (prime? n)
  (= n (smallest-divisor n)))

(define (smallest-divisor n)
  (find-divisor n 2))

(define (find-divisor n test-divisor)
  (cond ((> (square test-divisor) n) n)
        ((divides? test-divisor n) test-divisor)
        (else (find-divisor n (+ test-divisor 1)))))

(define (square x)
  (* x x))

(define (divides? x y)
  (= (remainder y x) 0))

\end{lstlisting}

Below is the output printed in the REPL in Dr.Racket. We can see that at $n = 40$ we get an output value of 1681 which is not a prime.
1681 is the square of 41. 

\begin{lstlisting}
> (print-polyval-prime 1 41)
n = 1    f(n) = 43    prime? -> yes
n = 2    f(n) = 47    prime? -> yes
n = 3    f(n) = 53    prime? -> yes
n = 4    f(n) = 61    prime? -> yes
n = 5    f(n) = 71    prime? -> yes
n = 6    f(n) = 83    prime? -> yes
n = 7    f(n) = 97    prime? -> yes
n = 8    f(n) = 113    prime? -> yes
n = 9    f(n) = 131    prime? -> yes
n = 10    f(n) = 151    prime? -> yes
n = 11    f(n) = 173    prime? -> yes
n = 12    f(n) = 197    prime? -> yes
n = 13    f(n) = 223    prime? -> yes
n = 14    f(n) = 251    prime? -> yes
n = 15    f(n) = 281    prime? -> yes
n = 16    f(n) = 313    prime? -> yes
n = 17    f(n) = 347    prime? -> yes
n = 18    f(n) = 383    prime? -> yes
n = 19    f(n) = 421    prime? -> yes
n = 20    f(n) = 461    prime? -> yes
n = 21    f(n) = 503    prime? -> yes
n = 22    f(n) = 547    prime? -> yes
n = 23    f(n) = 593    prime? -> yes
n = 24    f(n) = 641    prime? -> yes
n = 25    f(n) = 691    prime? -> yes
n = 26    f(n) = 743    prime? -> yes
n = 27    f(n) = 797    prime? -> yes
n = 28    f(n) = 853    prime? -> yes
n = 29    f(n) = 911    prime? -> yes
n = 30    f(n) = 971    prime? -> yes
n = 31    f(n) = 1033    prime? -> yes
n = 32    f(n) = 1097    prime? -> yes
n = 33    f(n) = 1163    prime? -> yes
n = 34    f(n) = 1231    prime? -> yes
n = 35    f(n) = 1301    prime? -> yes
n = 36    f(n) = 1373    prime? -> yes
n = 37    f(n) = 1447    prime? -> yes
n = 38    f(n) = 1523    prime? -> yes
n = 39    f(n) = 1601    prime? -> yes
n = 40    f(n) = 1681    prime? -> no
n = 41    f(n) = 1763    prime? -> no
done
\end{lstlisting}
\end{problem}

\begin{problem} %160
  This problem is solved but it is essentially solving linear equations.
  \[
  P(x) = ax + b
  \]

  Substituting 1 and 2 as given we get

  \begin{gather*}
    a + b = 7\\
    2a + b = 5
  \end{gather*}

  We compute from the above two equations that $a = -2$ and $b = 9$. Thus $P(x) = -2x + 9$.
\end{problem}

\begin{problem} %161
  We can think of this as a straight line on a co-ordinate plane. This line at $x = 1$ gives $y = 0$ so the point $(1,0)$ lies on it.
  The other point is $(2,0)$. We notice that both the points lie on the $X$ axis which means the line is the $X$ axis itself.

  Thus $P(x) = 0$.
  
  \begin{tikzpicture}[scale=1.2]

  % Axes
  \draw[->] (-0.5,0) -- (3,0) node[right] {$x$};
  \draw[->] (0,-0.5) -- (0,1.5) node[above] {$y$};

  % Ticks
  \foreach \x in {1,2}
    \draw (\x,0.05) -- (\x,-0.05) node[below] {\x};

  % Points
  \fill (1,0) circle (2pt) node[above] {$(1,0)$};
  \fill (2,0) circle (2pt) node[above] {$(2,0)$};

\end{tikzpicture}
\end{problem}

\begin{problem} %162
  Now the polynomial can be linear or quadratic. We can state it as $P(x) = (x - a)(x - b)Q(x)$. Also one important point to note issue
  that a curve when it cuts/crosses the $X$ axis where $y = 0$ those are the roots of the equation represented by that curve. Thus
  we can factor it appropriately. 

  \[
  P(x) = (x - 1)(x - 2)Q(x)
  \]

  Here $Q(x)$ is some polynomial but we are given that the degree of $P(x)$ cannot exceed 2 so it means that $Q(x)$ will be a constant number.

  Thus the polynomial is $P(x) = x^2 - 3x + 2 + k$ which is not $0$. $k$ refers to effect of a constant number originating from $Q(x)$.
\end{problem}

\begin{problem} %163
  The problem is exactly like the last one.

  We can write it as:

  \[
  P(x) = (x - 1)(x - 2)Q(x)
  \]
  
  Given that $P(3) = 4$ we can make the following substitutions. 
  \begin{gather*}
    4 = (3 - 1) (3 - 2) Q(x)\\
    4 = 2Q(x)\\
    Q(x) = 2
  \end{gather*}

  We could have reasoned out at the start itself that $Q(x)$ will actually be a constant because we are told that $P(x)$ cannot exceed
  degree $2$.

  Thus $P(x) = 2(x - 1)(x - 2)$ which is $2x^2 - 6x + 4$.
\end{problem}

\begin{problem} %164
  This is solved in the book and uses proof by contradiction approach to solve. Summarizing the solution. If $P(x)$ and $Q(x)$ are
  two polynomials of degree not more than 2 and for three different numbers the two polynomials are same then both the polynomials are indeed equivalent.

  Let a polynomial be defined as

  \[
  R(x) = P(x) - Q(x)
  \]

  Then for those three numbers we know that
  \[
  R(x_1) = R(x_2) = R(x_3) = 0
  \]

  But $R(x)$ cannot have a degree more than 2 and thus it cannot have three roots/solutions so $R(x)$ indeed must be equal to $0$.
  From this we derive that $P(x) = Q(x)$.
\end{problem}

\begin{problem} %165
  This is fairly simple in terms of looking at it from a point of view of simultaneous equations.
  \begin{gather*}
    16a + 4b + c = 0\\
    49a + 7b + c = 0\\
    100a + 10b + c = 0
  \end{gather*}

  Solve the first two we get 
  \[
  11a + b = 0
  \]

  From the second and third we get 
  \[
  17a + b = 0
  \]

  Solving these two we get $a = 0$ and then we get $b = 0$ plugging these two back in one of the original equations we also getting
  $c = 0$.
\end{problem}

\begin{problem} %166
  As guided we can use the same line of reasoning for the previous problem for degree 2.

  Let the polynomial be $P(x)$ and $Q(x)$ be equal for numbers $x_1$ up to $x_{n + 1}$. Now we have another polynomial which is given as 
  $R(x) = P(x) - Q(x)$. From this we state that
  \[
  R(x_1) = R(x_2) = ... = R(x_n) = R(x_{n + 1}) = 0
  \]

  So $R(x)$ has $(n + 1)$ roots which is not possible since it can have only $n$ roots since its degree is at the maximum $n$, thus $R(x) = 0$.
  Hence we get that $P(x) = Q(x)$ showing that such a polynomial is unique.
\end{problem}

\begin{problem} %167
  (a)
  Given 2 roots already
  \begin{gather*}
    P(x) = (x - 1)(x - 2)Q(x)\\
    P(3) = 4 = (3 - 1)(3 - 2)Q(x)\\
    Q(x) = 2
  \end{gather*}

  Thus $P(x) = 2 (x - 1)(x - 2)$.\\

  (b)
  Same as preceding problem
  \begin{gather*}
    P(x) = (x - 1)(x - 3)Q(x)\\
    P(2) = 2 = (2 - 1)(2 - 3)Q(x)\\
    Q(x) = -2
  \end{gather*}

  Thus $P(x) = -2 (x - 1)(x - 3)$.\\

  (c)
  Same as preceding problem
  \begin{gather*}
    P(x) = (x - 2)(x - 3)Q(x)\\
    P(1) = 6 = (1 - 2)(1 - 3)Q(x)\\
    Q(x) = 3
  \end{gather*}

  Thus $P(x) = 3 (x - 2)(x - 3)$.\\

  (d)
  Though the problem is solved in two ways in the book I find the easier method is to simply solve linear equations.

  Let 
  \[
  P(x) = ax^2 + bx + c
  \]

  We know that the degree cannot exceed 2 thus the above form. Now plug in given values we get the following set of linear equations.

  \begin{gather*}
    a + b + c = 6\\
    4a + 2b + c = 2\\
    9a + 3b + c = 4
  \end{gather*}

  Solving for $a$, $b$, and $c$ gives us.
  \begin{gather*}
    a = 3\\
    b = -13\\
    c = 16
  \end{gather*}

  Hence the polynomial is $P(x) = 3x^2 -13x + 16$.
\end{problem}

\begin{problem} %168
  Again this could be solved as linear equations. Let the polynomial be $ax^3 + bx^2 + cx + d$. Plugging in the given values.

  \begin{gather*}
    - a + b - c + d = 2\\
    d = 1\\
    a + b + c + d = 2\\
    8a + 4b + 2c + d = 7
  \end{gather*}

  Substitute $d = 1$

  \begin{gather*}
    - a + b - c = 1\\
    d = 1\\
    a + b + c = 1\\
    8a + 4b + 2c = 6
  \end{gather*}

  From the first and third equation we get $b = 1$. Updating the equations again.

  \begin{gather*}
    a = -c\\
    b = 1\\
    d = 1\\
    a = -c\\
    4a  + c = 1
  \end{gather*}

  From above we get

  \begin{gather*}
    a = \dfrac{1}{3}\\
    b = 1\\
    c = -\dfrac{1}{3}\\
    d = 1
  \end{gather*}

  Thus the polynomial can be stated as
  \[
  P(x) = \dfrac{x^3}{3} + x^2 - \dfrac{x}{3} + 1
  \]
\end{problem}

\begin{problem} %169
  This is similar to earlier problems in this chapter. We will use the same reasoning as in problem 164 and 166 but modified for this problem.

  Given 
  \begin{gather*}
    P(x_1) = y_1\\
    P(x_2) = y_2\\
    ...\\
    P(x_{10}) = y_{10}
  \end{gather*}

  Let another polynomial $Q(x)$ also have the same outputs with the given input arguments.
  \begin{gather*}
    Q(x_1) = y_1\\
    Q(x_2) = y_2\\
    ...\\
    Q(x_{10}) = y_{10}
  \end{gather*}

  Now we have another polynomial which is the difference of the above two, $R(x) = P(x) - Q(x)$. We can derive the following relationship
  from this for $i \in [1, 10]$
  \[
  R(x_i) = P(x_i) - Q(x_i)
  \]

  But we know that $P(x_i) = y_i$ and also $Q(x_i) = y_i$. Substituting this in the equation above.

  \[
  R(x_i) = P(x_i) - Q(x_i) = y_i - y_i = 0
  \]
 
  If $R(x_i) = 0$ then $P(x_i) = Q(x_i)$. Thus $P(x)$ is unique.
\end{problem}

\begin{problem} %170
  This is a nice problem.

  There are a few ways to do this. The most obvious is solve for the unknowns but the authors say do not do that. So let us keep
  that aside.

  The other method is just visual inspection. We see the coefficients of $a$ to be square numbers $10^2$, $6^2$, and $2^2$. Also the bases 
  of these are coefficients of $b$. So we have a generic structure like $ax^2 + bx + c = k$. We have boiled down the question to what we have been 
  proving multiple times in this chapter. We have a polynomial of degree 2 $P(x)$ such that only one unique polynomial exists since we are given that 
  $P(10) = 18.37$, $P(6) = 0.05$, and $P(2) = -3$. Thus $a$, $b$, and $c$ will exist uniquely.

  But this is the solution which did not come to my mind immediately. A very traditional way to show that there exists unique solutions to 
  these equations is to use matrices. If we show that the determinant of the matrix of coefficients of these equations is non-zero then there is a solution to this. 

  \[
  A = 
  \begin{pmatrix}
    100 & 10 & 1\\
    36 & 6 & 1\\
    4 & 2 & 1
  \end{pmatrix}
  \]

  This coefficient matrix is not linearly dependent so determinant is non-zero. Anyways we can still check it. Doing some row operations.
  $R_1 - R_2$ and $R_2 - R_3$, we get 
  \[
  A = 
  \begin{pmatrix}
    64 & 4 & 0\\
    32 & 4 & 0\\
    4 & 2 & 1
  \end{pmatrix}
  \]

  Another row operation $R_1 - R_2$, this gives 
  \[
  A = 
  \begin{pmatrix}
    32 & 0 & 0\\
    32 & 4 & 0\\
    4 & 2 & 1
  \end{pmatrix}
  \]

  We get $\det(A) \neq 0$. Thus unique solution exists.

  There is a third way to look at this problem. This is when we think of a 3-D co-ordinate system with axis $a$, $b$, and $c$. Each of 
  the 3 given equations should represent a unique plane in this 3 dimensional co-ordinate system for there to be a unique solution. These 3 planes 
  would then intersect at a single point. Assume (proof by contradiction) that if at least any of the two planes are either parallel to each other 
  or are equal (basically the same). We immediately observe that given the coefficients none of these two conditions are possible because either the 
  coefficients would have been same or reducible (after dividing by a suitable number) to a the same numbers. Thus these planes will always 
  intersect at one point.

  Let us try to see this using Lisp programming too.

  \begin{lstlisting}
#lang racket

(require plot)

(plot-new-window? #t)

;; 3 planes assume c to be 0 or we can use an offset constant either ways

(define (plane1 x y)
  (- 18.37 (* 100 x) (* 10 y)))

(define (plane2 x y)
  (- 0.05 (* 36 x) (* 6 y)))

(define (plane3 x y)
  (- -3 (* 4 x) (* 2 y)))

;; plot the planes

(plot3d
 (list
  (surface3d plane1 -1 1 -1 1
             #:color "red"
             #:alpha 0.6)
  (surface3d plane2 -1 1 -1 1
             #:color "blue"
             #:alpha 0.6)
  (surface3d plane3 -1 1 -1 1
             #:color "green"
             #:alpha 0.6))
 #:x-label "a"
 #:y-label "b"
 #:z-label "c")
  \end{lstlisting}

\begin{figure}[H]
\centering
\includegraphics[width=0.7\linewidth]{images/problem170.png}
\label{fig:point intersection of 3 unique planes}
\end{figure} 

The intersection point is not very clear but the actual co-ordinate for the planes used in the Lisp code returns
$(0.4771875, -3.05625, -3.265)$.
\end{problem}

\begin{problem} %171
  The answer is given but the solution isn't.

  We can say that this polynomial has 10 roots and they are numbers from 1 to 10, there could be more roots but
  we know these 10 for sure. Since the highest coefficient of $P(x)$ is 1 
  it means that no other coefficient will be greater than 1. The smallest degree polynomial which we can form is
  \[
  P(x) = (x - 1)(x - 2)(x - 3)(x - 4)(x - 5)(x - 6)(x - 7)(x - 8)(x - 9)(x - 10)
  \]

  If we substitute $x = 11$ then we basically get $10!$ which computes to 3628800.
\[
P(11) = (11 - 1)(11 - 2)(11 - 3)(11 - 4)(11 - 5)(11 - 6)(11 - 7)(11 - 8)(11 - 9)(11 - 10)
\]
\end{problem}

%%%%%%%%%%%%%%%%%%%%%%%%%%%%%%%%%%%%%%%%%
% Chapter End
%%%%%%%%%%%%%%%%%%%%%%%%%%%%%%%%%%%%%%%%%

%%%%%%%%%% 39
\section{Arithmetic progressions}
\begin{problem} %172
  First $+2$, second $-1$.
\end{problem}

\begin{problem} %173
  The difference is $-7$. Thus $-2 - 7 = -9$. So $-9$.
\end{problem}

\begin{problem} %174
  This is a subset of natural numbers starting from 2. If it were starting from 1 the $1000_{th}$ number would be 1000. Since we 
  are starting from 2 we move a number forward so the answer is 1001. Note that the set of natural numbers is also an arithmetic 
  progression where the common difference between two successive elements is 1.
\end{problem}

\begin{problem} %175
  Without churning out formulas we can derive them ourselves.

  The $2_{nd}$ term is first term plus the common difference multiplied by the common difference. We can generalize that by
  \[
  T_n = a + (n - 1) \times d
  \]

  Where $T_n$ is the $n_{th}$ term, $a$ is the first term and $d$ is the constant difference in this arithmetic progression.

  Thus the $1000_{th}$ term will be $(2 + (1000 - 1) \times 2)$ which is $2000$.
\end{problem}

\begin{problem} %176
  We apply the same logic as the earlier problem and arrive at $1 + (1000 - 1) \times 2$ which is $1999$.
\end{problem}

\begin{problem} %177
  We just derived the formula in problem number 175. The $n_{th}$ is given by 
  \[
  T_n = a + (n - 1) \times d
  \]

  Where $T_n$ is the $n_{th}$ term, $a$ is the first term and $d$ is the constant difference in this arithmetic progression.
\end{problem}

\begin{problem} %178
  Yes this too is an arithmetic progression. We can write it as say the last term being $n$, the second last term being $n - d$, 
  the third last term being $n - 2d$ and so on. Thus the common difference can be denoted as $-d$ and the $n_{th}$ can be stated as 

  \[
  T_n = last_{term} - (n - 1) \times d
  \]

  where $n$ is counted from the end.
\end{problem}

\begin{problem} %179
  Yes the resultant series is also an arithmetic progression. The common difference now doubles.

  \begin{gather*}
    a, \cancel{a + d}, a + 2d, \cancel{a + 3d}, a + 4d, \cancel{a + 5d}, a + 6d, \cancel{a + 7d}, a + 8d, \cancel{a + 9d}, a + 10d, ...\\
    a, a + 2d, a + 4d, a + 6d, a + 8d, a + 10d, ...
  \end{gather*}
\end{problem}

\begin{problem} %180
  No. Now its not an arithmetic progression. The common difference pattern will become after omission as following 
  \[
  0, d, 2d, d, 2d, d, 2d, ...
  \]
\end{problem}

\begin{problem} %181
  The first term is 5 and suppose the common difference is $d$ then the second term will be $5 + d$. The third term should 
  be $5 + 2d$ but we are given that is 8. Therefore 
  \begin{gather*}
    5 + 2d = 8\\
    d = \dfrac{3}{2}
  \end{gather*}

  The second term will be 
  \begin{gather*}
    5 + d\\
   =  5 + \dfrac{3}{2}\\
   = \dfrac{13}{2}\\
   =  6.5
  \end{gather*}
\end{problem}

\begin{problem} %182
  This is simply the average technically since the middle number should be equidistant from the first and third number. This is one 
  way to look at this problem. The other way is again the usual way of having the numbers as $a$, $a + d$ and $a + 2d$. Then 
  equating $a + 2d$ with $b$ which gives $d$ as $\dfrac{b - a}{2}$. The middle term then can be $a + \dfrac{(b - a)}{2}$ which 
  is $\dfrac{(a + b)}{2}$.
\end{problem}

\begin{problem} %183
  We can write the progression as \\
  $a, a + d, a + 2d, a + 3d, ...$\\

  We are given that $a + 3d = b$\\

  This gives the difference $d$ as $\dfrac{(b - a)}{3}$. The second and third term will be. 

  Second term\\
  \begin{gather*}
    a + d\\
    a + \dfrac{(b - a)}{3}\\
    \dfrac{2a + b}{3}
  \end{gather*}

Third term\\
  \begin{gather*}
    a + 2d\\
    a + \dfrac{2(b - a)}{3}\\
    \dfrac{a + 2b}{3}
  \end{gather*}
\end{problem}

\begin{problem} %184
  The first term $a$ is 1. The difference $d$ is 2. The last term is 999. If there are $n$ terms then the last term can be written as 
  $a + (n - 1)d$. Let us equate.

  \begin{gather*}
    a + (n - 1)d = 999\\
    1 + (n - 1)2 = 999\\
    n = 501
  \end{gather*}

  There are 500 terms in this progression.
\end{problem}

%%%%%%%%%%%%%%%%%%%%%%%%%%%%%%%%%%%%%%%%%
% Chapter End
%%%%%%%%%%%%%%%%%%%%%%%%%%%%%%%%%%%%%%%%%

%%%%%%%%%% 40
\section{The sum of arithmetic progression}

\begin{problem} %185
  We should do this the way the child Gauss would do. Let the sum of the series be $S$

  \[
  S = 1 + 3 + 5 + 7 + ... + 999
  \]
  \[
  S = 999 + 997 + 995 + 993 +... + 1
  \]
  Adding these two we get\\
  \[
  2S = 1000 + 1000 + 1000 + ... + 1000
  \]

  The important thing is to know how many terms are there. We can easily deduce that since $999 = 1 + (n - 1) 2$. This gives $n$ as 
  500. Back to the earlier equation.
  \[
  2S = 1000 \times 500
  \]
  \[
  S = 250000
  \]

\end{problem}

\begin{problem} %186
  This is similar to the previous problem.

  \[
  S = a + (a + 1.d) + (a + 2.d) + ... + (a + (n - 1) d)
  \]
  Here $(a + (n - 1)d) = b$.

  Adding these two\\
  \[
  S = a + (a + 1.d) + (a + 2.d) + ... + (a + (n - 1) d)
  \]
  \[
  S = (a + (n - 1) d) + ... + (a + d) + a
  \]

  We get 

  \[
  2S = (a + a + (n - 1)d) + (a + a + (n - 1)d) + ... + (a + a + (n - 1)d)
  \]
  Which essentially is\\
  \[
  2S = (a + b) + (a + b) + ... + (a + b)
  \]
  since there are $n$ terms on the right hand side
  \[
  2S = n(a + b)
  \]
  \[
  S = \dfrac{n(a + b)}{2}
  \]
\end{problem}

\begin{problem} %187
  The reasoning is given in the book but we did not make this mistake. The explanation withe figure is quite good.\\
  Essentially if we notice we fold the series over each other once reversed. This is exactly the child Gauss method.
\end{problem}

\begin{problem} %188
  The hint given is a good one especially visually speaking. But let us try and solve it with equations.

  An odd number can be given as $2k +1$ for every $k > 0$ where $k$ is an integer. Therefore, the sum of the first $n$
  odd numbers will be.

  \[
  S_{odd} = [(2k + 1)] + [(2k + 1) + (1 \times 2)] + [(2k + 1) + (2 \times 2)] + ... + [(2k + 1) + ((n - 1) \times 2)]
  \]

  Let us rearrange the terms.

  \[
  S_{odd} = 2kn + n + 2(1 + 2 + 3 + ... + (n -1))
  \]
  
  We are in familiar territory of an arithmetic progression.

  \[
  S_{odd} = 2kn + n + \dfrac{2n(n - 1)}{2}
  \]
  \[
  S_{odd} = 2kn + n + n^2 - n
  \]
  \[
  S_{odd} = 2kn +  n^2
  \]

  Please note the question asks us that "the sum of $n$ first odd numbers" meaning that $2k + 1$ should have been $1$, therefore 
  $k = 0$ in this instance. We get: 
  \[
  S_{odd} = n^2
  \]

  Hence proved.
\end{problem}

%%%%%%%%%%%%%%%%%%%%%%%%%%%%%%%%%%%%%%%%%
% Chapter End
%%%%%%%%%%%%%%%%%%%%%%%%%%%%%%%%%%%%%%%%%

%%%%%%%%%% 41
\section{Geometric progressions}

\begin{problem} %189
  For the first one $\dfrac{6}{3}$ which is 2.\\

  For the second one $\dfrac{2}{6}$ which is $\dfrac{1}{3}$.
\end{problem}

\begin{problem} %190
  Common ratio is $\dfrac{3}{2}$. Thus third term will be
  \[
  3 \times \dfrac{3}{2} = \dfrac{9}{2}
  \]
\end{problem}

\begin{problem} %191
  The first term let us start labeling it as $a$ and the common ratio as $r$. Here 

  $a = 3$\\

  $r = \dfrac{6}{3} = 2$\\

  So the terms are in the below sequence for $n$ terms: 

  \[
  a, ar, ar^2, ar^3, ar^4,...,ar^{n - 1}
  \]

  For this specific problem we have the 1000th term as:

  \[
  ar^{1000 - 1} = 3 \times 2^{1000 - 1} = 3 \times 2^{999}
  \]
\end{problem}

\begin{problem} % 192
  We just solved this in the previous problem. The $n$th term is given by $a \times q^{n - 1}$.
\end{problem}

\begin{problem} %193
  In this problem $a = 1$ and $ar^3 = 4$/. We compute $r$ as $\pm 2$. The second term will be $ar$ which will be $\pm 2$.
\end{problem}

\begin{problem} %194
  
\end{problem}

%%%%%%%%%% 42
\section{The sum of a geometric progression}

%%%%%%%%%% 43 
\section{Different problems about progressions}

%%%%%%%%%% 44
\section{The well-tempered clavier}

%%%%%%%%%% 45
\section{The sum of an infinite geometric progressions}

%%%%%%%%%% 46
\section{Equation}

%%%%%%%%%% 47
\section{A short glossary}

%%%%%%%%%% 48
\section{Quadratic equations}

%%%%%%%%%% 49
\section{The case $p = 0$. Square roots}

%%%%%%%%%% 50
\section{Rules for square roots}

%%%%%%%%%% 51
\section{The equation $x^2 + px + q = 0$}

%%%%%%%%%% 52
\section{Vieta's theorem}

%%%%%%%%%% 53
\section{Factoring $ax^2 + bx + c$}

%%%%%%%%%% 54
\section{A formula for $ax^2 + bx + c = 0$ (where $a \neq 0$)}

%%%%%%%%%% 55
\section{One more formula concerning quadratic equatons}

%%%%%%%%%% 56
\section{A quadratic equation becomes linear}

%%%%%%%%%% 57
\section{The graph of a quadratic polynomial}

%%%%%%%%%% 58
\section{Quadratic inequalities}

%%%%%%%%%% 59
\section{Maximum and minimum values of a quadratic polynomial}

%%%%%%%%%% 60
\section{Biquadratic equations}

%%%%%%%%%% 61
\section{Symmetric equations}

%%%%%%%%%% 62
\section{How to confuse students on an exam}

%%%%%%%%%% 63
\section{Roots}

%%%%%%%%%% 64
\section{Non-integer inequalities}

%%%%%%%%%% 65
\section{Proving inequalities}

%%%%%%%%%% 66
\section{Arithmetic and geometric means}

%%%%%%%%%% 67
\section{The geometric mean does not exceed the arithmetic mean}

%%%%%%%%%% 68
\section{Problems about maximum and minimum}

%%%%%%%%%% 69
\section{Geometric illustrations}

%%%%%%%%%% 70
\section{The arithmetic and geometric means of several numbers}

%%%%%%%%%% 71
\section{The quadratic mean}

%%%%%%%%%% 72
\section{The harmonic mean}

\end{document}