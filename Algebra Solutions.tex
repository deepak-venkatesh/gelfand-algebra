\documentclass[10pt]{article}
\usepackage{amsmath}
\usepackage{hyperref}
\usepackage{listings}
\usepackage{longdivision}
\usepackage{graphicx}
\usepackage{tikz}
\usepackage{float}
\usepackage{cancel}
\usepackage{xlop}
\usepackage{xcolor} % for colors
\lstdefinelanguage{Scheme}{
  keywords={define,lambda,if,cond,let,begin,car,cdr,cons,quote},
  sensitive=true,
  morecomment=[l]{;},
  morestring=[b]",
}

\lstset{
  language=Scheme,
  basicstyle=\ttfamily\small,
  keywordstyle=\color{blue}\bfseries,
  commentstyle=\color{gray},
  stringstyle=\color{orange},
  frame=none,
  columns=flexible,
  showstringspaces=false,
}

\setlength{\parindent}{0pt}
\setlength{\parskip}{0.5em}

% Add license
\usepackage{ccicons}
\usepackage[
    type={CC},
    modifier={by-sa},
    version={4.0},
]{doclicense}

\title {Solutions to \textit{Algebra}\\
[0.5em]\large by I.M. Gelfand \& A. Shen}

\author{Deepak Venkatesh}
\date{\today}


% --- Custom Problem Environment ---
\newcounter{problem}
\newenvironment{problem}[1][]{
  \refstepcounter{problem}
  \par\noindent\textbf{Problem~\theproblem. #1}\par
  \vspace{0.5em}
}{\vspace{1em}}



\begin{document}
\noindent
\maketitle
\begin{abstract}
\noindent
\textit{Algebra} by I.M.Gelfand and A.Shen, first published in September 1993, is a 150 page book
covering 72 topics related to school level algebra. The book presents 342 problems some with
solutions, and others without. This booklet aims to provide correct solutions to all
the 342 problems listed in \textit{Algebra}. Each solution is carefully checked, either by hand (particularly for proofs)
or programmatically using Scheme (a dialect of Lisp) in a REPL environment. DrRacket serves as
the IDE, and the implementation language used is Racket, a modern variant of Scheme. LLMs have helped
me in typing this out in \LaTeX. All errors are my own, please report
any issue \href{https://github.com/deepak-venkatesh/gcpm/issues}{here}.
\end{abstract}
\vspace{27em}
\doclicenseThis
\newpage
\tableofcontents
\newpage % start sections on a fresh page



%%%%%%%%%% 1
\section{Introduction}
No problems
%%%%%%%%%%%%%%%%%%%%%%%%%%%%%%%%%%%%%%%%%
% Chapter End
%%%%%%%%%%%%%%%%%%%%%%%%%%%%%%%%%%%%%%%%%

%%%%%%%%%% 2
\section{Exchange of terms in addition}

No problems

%%%%%%%%%%%%%%%%%%%%%%%%%%%%%%%%%%%%%%%%%
% Chapter End
%%%%%%%%%%%%%%%%%%%%%%%%%%%%%%%%%%%%%%%%%

%%%%%%%%%% 3
\section{Exchange of terms in multiplication}

No problems

%%%%%%%%%%%%%%%%%%%%%%%%%%%%%%%%%%%%%%%%%
% Chapter End
%%%%%%%%%%%%%%%%%%%%%%%%%%%%%%%%%%%%%%%%%

%%%%%%%%%% 4
\section{Addition in the decimal number system}

\begin{problem} %1
Stack 8s knowing that $8\times5$ ends with a 0 (that is 40). This gives a carry over of 4.
So we need $4 + 8 + 8$ to get a number that ends with 0 for the tens place. This gives a carry
over of 2. This 2 can get added to 8 in the hundreds place. The tens place structure shown
below.

\begin{center}
\[
\begin{array}{r}
\cdots 8\\
\cdots 8\\
\cdots 8\\
+\,\cdots 8\\
\hline
\cdots 0
\end{array}
\]
\end{center}

\begin{center}
\[
\begin{array}{r}
888\\
088\\
008\\
+$ $008\\
\hline
1000
\end{array}
\]
\end{center}

Answer is self verifiable.

\end{problem}

\begin{problem} %2

\begin{center}
\[
\begin{array}{r}
AAA\\
+$ $BBB\\
\hline
AAAC
\end{array}
\]
\end{center}

The solution lies in the fact that in the answer the thousandth's place A has to be 1. This
is so because whenever there is a carry over the tens digit will be 1 in addition (in this structure).
At the maximum level it would be 9 + 9 = 18 for instance.\\
So we have A as 1.

\begin{center}
\[
\begin{array}{r}
111\\
+$ $BBB\\
\hline
111C
\end{array}
\]
\end{center}

Now for B it has to be 9 because if it was any other number then the answer could not have 1s
in the places it has now. If B was 0 then the thousandth place 1 in the answer would not
materialize. So we have now.

\begin{center}
\[
\begin{array}{r}
111\\
+$ $999\\
\hline
111C
\end{array}
\]
\end{center}

We can easily see that C is 0 now. So we have
\[A = 1\]
\[B = 9\]
\[C = 0\]

Answer is self verifiable.
\end{problem}

%%%%%%%%%%%%%%%%%%%%%%%%%%%%%%%%%%%%%%%%%
% Chapter End
%%%%%%%%%%%%%%%%%%%%%%%%%%%%%%%%%%%%%%%%%

%%%%%%%%%% 5
\section{The multiplication table and the multiplication algorithm}

\begin{problem} %3
This looks tricky but is fairly easy to understand the pattern once written down. 1001 multiplied by any 3 digit
number will be that 3 digit number repeating twice. This is so because the `001' in 1001 when multiplied by the
3 digit number gives itself and then the `1' in the thousandth's place in 1001 and gives the 3 digit number. It 
is like concatenation of a 3 digit number to itself when multiplied by 1001.

\begin{center}
\[
\begin{array}{r}
715\\
\times$ $001\\
\hline
715
\end{array}
\]
\end{center}

\begin{center}
\[
\begin{array}{r}
715\\
\times$ $1001\\
\hline
715\\
+$ $715000\\
\hline
715715
\end{array}
\]
\end{center}

Answer is self verifiable.

Answer is 715715.

\end{problem}

\begin{problem} %4
This is similar to the previous problem except that we have a 2 digit number getting multiplied by `01'. It will
still result in a concatenation.

Verified in Scheme
\begin{lstlisting}
> (* 101010101 57)
5757575757
\end{lstlisting}

Answer is 5757575757

\end{problem}

\begin{problem} %5
This is on the same lines as previous two problems.
\begin{center}
\[
\begin{array}{r}
1020304050\\
\times$ $10001\\
\hline
1020304050\\
+$ $10203040500000\\
\hline
10204060804050
\end{array}
\]
\end{center}

Verified in Scheme
\begin{lstlisting}
> (* 10001 1020304050)
10204060804050
\end{lstlisting}

Answer is 10204060804050

\end{problem}

\begin{problem} %6
This is a trick I have been teaching all kids.

To look at an easier version of the problem say we have to $11 * 11$. This is 121. Two 1s is getting
multiplied by two 1s (eleven in this case). So we have the mnemonic $1..2...then..reverse$. When we
make one of the numbers as three 1s that is $111 * 11$ then we repeat the center digit
$1..2..2...then..reverse$, the answer being 1221.

In this example we have 11111 multiplied by 1111. This should give us 12344321. Two 4s in center.

Let us look at a pattern in Scheme to verify.
\begin{lstlisting}
> (* 1111 1111)
1234321

> (* 11111 1111)
12344321

> (* 111111 1111)
123444321

> (* 1111111 1111)
1234444321
\end{lstlisting}

Answer is 12344321

\end{problem}

\begin{problem} %7
The solution is provided in the book. Its an easy problem where we use the last digits of the 3s
multiplication table.

\begin{center}
\[
\begin{array}{r}
1ABCDE\\
\times$ $3\\
\hline
ABCDE1\\
\end{array}
\]
\end{center}

The only was to get 1 as the answer when 3 is multiplied by $E$ is $3 * 7$. The carryover is 2. So now
we have $(3 * D) + 2$ which should end with E which we know is 7 now. So we get 3 times 5 plus 2 which ends
in 7. Therefore D is 5. We keep going till we reach the end at the final number.

This number is actually important since this is the number which repeats when we divide a number by 7. This number is starting with
1 and with only 0s thereafter. Next few problems have this trick involved.

Answer is verified.

Answer is 142857.

\end{problem}

%%%%%%%%%%%%%%%%%%%%%%%%%%%%%%%%%%%%%%%%%
% Chapter End
%%%%%%%%%%%%%%%%%%%%%%%%%%%%%%%%%%%%%%%%%

%%%%%%%%%% 6
\section{The division algorithm}
\begin{problem} %8
This is inverse of the previous chapter. 

\begin{center}
\longdivision{123123123}{123}
\end{center}

We can see the pattern of 001..001..001 in the answer.
Suppose we had 1234123412341234 divided by 1234.
What would we get? It would be 0001..0001..0001..0001

Verified in Scheme
\begin{center}
\begin{lstlisting}
> (/ 1234123412341234 1234)
1000100010001
> (/ 123123123 123)
1001001
\end{lstlisting}
\end{center}


Answer is 1001001

\end{problem}

\begin{problem} %9
We can simplify the problem here. We have 1111111 (seven 1s) which will divide a long series of 1s.
That means for every group of seven 1s the quotient will be 1. Since there are 100 1s we will
have 14 groups of seven 1s that makes it 98 1s. The last two 1s will be the remainder. So the remainder
is 11.

A smaller pattern here
\begin{center}
\longdivision{111111111}{1111111}
\end{center}

Answer is 11

\end{problem}

\begin{problem} %10
We get here the cyclical nature of the quotient when we divide by 7. example

\begin{center}
\longdivision{1000000}{7}
\end{center}

So 1000000 (7 digit number) when divided by 7 will give a recurring quotient with 142857.
Therefore when we divide 1000...0 (20 zeroes) we have 18 zeroes consumed with 3 times 142857 appearing in the
quotient. Then the last 2 zeroes will give the quotient of 14 and a remainder of 2. Thus the quotient
should be 14285714285714285714 and a remainder of 2.

Verified in Scheme
\begin{center}
\begin{lstlisting}
> (/ 100000000000000000000 7)
14285714285714285714 2/7
\end{lstlisting}
\end{center}

\end{problem}

\begin{problem} %11
As shown in previous problem number 10 this pattern of 142857 will repeat. This the cyclical number we get in this instance.

\end{problem}

\begin{problem} %12
This is fairly similar to the previous two problems. The only difference is that when we start with a
different number the cyclical pattern of division by 7 starts with a different digit, but the pattern holds.
Let us take the example of 2000000 divided by 7.

\begin{center}
\longdivision{2000000}{7}
\end{center}

We see the same pattern but it starts at 2. So the pattern is 285714.

So the answers are\\
2000...00(20 0s): Quotient is 8571428571428571428. Remainder is 4.\\
3000...00(20 0s): Quotient is 42857142857142857142. Remainder is 6.\\
4000...00(20 0s): Quotient is 57142857142857142857. Remainder is 1.\\
5000...00(20 0s): Quotient is 71428571428571428571. Remainder is 3.\\
6000...00(20 0s): Quotient is 85714285714285714285. Remainder is 5.\\

Verified in Scheme
\begin{center}
\begin{lstlisting}
> (/ 200000000000000000000 7)
28571428571428571428 4/7
> (/ 300000000000000000000 7)
42857142857142857142 6/7
> (/ 400000000000000000000 7)
57142857142857142857 1/7
> (/ 500000000000000000000 7)
71428571428571428571 3/7
> (/ 600000000000000000000 7)
85714285714285714285 5/7
\end{lstlisting}
\end{center}

\end{problem}

\begin{problem} %13
The guess here should be that each of the answers will be in some permutation of 142857 barring when multiplied by 7.
Let us check.

\begin{center}
\opmul{142857}{1}
\end{center}

\begin{center}
\opmul{142857}{2}
\end{center}

\begin{center}
\opmul{142857}{3}
\end{center}

\begin{center}
\opmul{142857}{4}
\end{center}

\begin{center}
\opmul{142857}{5}
\end{center}

\begin{center}
\opmul{142857}{6}
\end{center}

\begin{center}
\opmul{142857}{7}
\end{center}

The way to answer this is to start at the one's place digit and work backwards.

Answer is verified above.

\end{problem}

\begin{problem} %14
Let us go for the first 10 natural numbers from 1 to 10.

Case for 1:

Anything divided by 1 is the same thing. So the dividend and quotient is same and there is no remainder.

\begin{center}
\longdivision{1000000}{1}
\end{center}

Case for 2:

Since the dividend ends with 0 it is an even number. So half of dividend is the quotient and remainder is 0.

\begin{center}
\longdivision{1000000}{2}
\end{center}

Case for 3:

In this case we will always get a remainder of 1 since 1 less than 10 or 100 or 1000 is divisible by 3.

\begin{center}
\longdivision{1000000}{3}
\end{center}

Case for 4:

Except for 10 as the dividend where we will get a remainder of 2 and a quotient fo 2 also, the rest of
the dividends will always be one fourths of the dividend since the dividend ends with 2 zeroes. The
remainder will be 0.

\begin{center}
\longdivision{10}{4}
\longdivision{1000000}{4}
\end{center}

Case for 5:

Here every number from 10 onwards will be divisible by 5. There will be no remainder.

\begin{center}
\longdivision{1000000}{5}
\end{center}

Case for 6:

In this case the remainder will always be 4. We will be stuck in an infinite loop of dividing 40 by 6 once we
finish our first division of 10 by 6. The quotient therefore will be 1 followed by all 6s. Example below.

\begin{center}
\longdivision{1000000}{6}
\end{center}

Case for 7 done earlier.

Case for 8:

The pattern in this case is that we have 1, 2, 5 as the quotient. And once there are no remainders
left we keep appending 0s to the quotient of 125.

\begin{center}
\longdivision{10}{8}
\longdivision{100}{8}
\longdivision{1000}{8}
\longdivision{10000}{8}
\longdivision{100000}{8}
\longdivision{1000000}{8}
\end{center}

Case for 9:

The first division by 9 gives a remainder of 1 and then an endless loop of 10 divided by 9. Remainder will
always be 1 and quotient will be 1111....

\begin{center}
\longdivision{1000000}{9}
\end{center}

Case for 10:

Fairly simple. Just remove the last zero from the dividend to get the quotient and there is no remainder.

\begin{center}
\longdivision{1000000}{10}
\end{center}

\end{problem}

%%%%%%%%%%%%%%%%%%%%%%%%%%%%%%%%%%%%%%%%%
% Chapter End
%%%%%%%%%%%%%%%%%%%%%%%%%%%%%%%%%%%%%%%%%

%%%%%%%%%% 7
\section{The binary system}

\begin{problem} %15
Binary to Decimal conversion can be written by:

[(0 or 1) $\times$ $2^0$] + [(0 or 1) $\times$ $2^1$] + [(0 or 1) $\times$ $2^2$]...

The numbers in the given list are basically the set of whole numbers.

\begin{table}[h!]
\centering
\begin{tabular}{c c l}
\textbf{Binary} & \textbf{Decimal} & \textbf{Expanded form} \\ \hline
0     & 0  & 0 \\
1     & 1  & $2^0$ \\
10    & 2  & $2^1$ \\
11    & 3  & $2^1 + 2^0$ \\
100   & 4  & $2^2$ \\
101   & 5  & $2^2 + 2^0$ \\
110   & 6  & $2^2 + 2^1$ \\
111   & 7  & $2^2 + 2^1 + 2^0$ \\
1000  & 8  & $2^3$ \\
1001  & 9  & $2^3 + 2^0$ \\
1010  & 10 & $2^3 + 2^1$ \\
1011  & 11 & $2^3 + 2^1 + 2^0$ \\
1100  & 12 & $2^3 + 2^2$ \\
1101  & 13 & $2^3 + 2^2 + 2^0$ \\
1110  & 14 & $2^3 + 2^2 + 2^1$ \\
1111  & 15 & $2^3 + 2^2 + 2^1 + 2^0$ \\
10000 & 16 & $2^4$ \\
10001 & 17 & $2^4 + 2^0$ \\
10010 & 18 & $2^4 + 2^1$ \\
10011 & 19 & $2^4 + 2^1 + 2^0$ \\
10100 & 20 & $2^4 + 2^2$ \\
10101 & 21 & $2^4 + 2^2 + 2^0$ \\
10110 & 22 & $2^4 + 2^2 + 2^1$ \\
\end{tabular}
\end{table}

Answer is verified

\end{problem}

\begin{problem} %16
This problem is basically binary representation of a natural number. 

Let $S=\{2^0,2^1,\dots,2^{n-1}\}$. Every integer $m$ with $0\le m\le (2^n-1)$
can be written uniquely as a sum of distinct elements of $S$.

The proof for this can be demonstrated using induction but we will skip that here. The
3\textsuperscript{rd} column in the previous problem (problem number 15) already shows the solution for
this problem.

\end{problem}

\begin{problem} %17
We refer back to the table in problem 15. For 14 in decimal the equivalent binary representation is
1110. 10000 binary is $ (1 * 2^4) + (0 * 2^3) + (0 * 2^2) + (0 * 2^1) + (0 * 2^0) $ and that is 16.

\end{problem}

\begin{problem} %18
From the binary representation theorem given in problem 16 let us look at a number of the form
$2^n \le 45$. The biggest $n$ here is 5 where we get $2^5$ = 32. So we have 32 as 100000. We need 13 more. We
apply the same logic and arrive at 1101 for 13. Thus adding the binary representations we get 101101 as the binary form 
of 45.

Answer is 101101
\end{problem}

\begin{problem} %19
10101101 in binary can be converted to decimal easily.

$ 10101101 = (1 * 2^7) + (0 * 2^6) + (1 * 2^5) + (0 * 2^4) + (1 * 2^3) + (1 * 2^2) + (0 * 2^1) + (1 * 2^0) $
$ 10101101 = 128 + 32 + 8 + 4 + 1 = 173 $

Answer is 173

\end{problem}

\begin{problem} %20

Binary Addition

0 + 0 = 0 this is so because $(0 * 2^0) + (0 * 2^0)$\\
0 + 1 = 1 this is so because $(0 * 2^0) + (1 * 2^0)$\\
1 + 1 = 0 this is so because $(1 * 2^0) + (1 * 2^0)$ the answer is 2 which is $(1 * 2^1) + (0 * 2^0)$ thus carry 1\\

\[
\begin{array}{r}
   1010\\
+ \,101\\
\hline
   1111
\end{array}
\]
This is 10 + 5 = 15 in base 10.

\[
\begin{array}{r}
   1111\\
+ \,1\\
\hline
   10000
\end{array}
\]
This is 15 + 1 = 16 in base 10.

\[
\begin{array}{r}
   1011\\
+ \,1\\
\hline
   1100
\end{array}
\]
This is 11 + 1 = 12 in base 10.

\[
\begin{array}{r}
   1111\\
+ \,1111\\
\hline
   11110
\end{array}
\]
This is 15 + 15 = 30 in base 10.

\end{problem}



\begin{problem} %21
Binary Subtraction

0 - 0 = 0 this is so because $(0 * 2^0) - (0 * 2^0)$\\
1 - 0 = 1 this is so because $(1 * 2^0) - (0 * 2^0)$\\
0 - 1 = 1 this is so because $(0 * 2^0) - (1 * 2^0)$ results in a borrow of 10\textsubscript{2}. So now
we have a 2 in decimal subtracted with 1. Thus there is a borrow of 1 in this case\\
1 - 1 = 0 this is so because $(1 * 2^0) - (1 * 2^0)$\\

\[
\begin{array}{r}
   1101\\
- \,101\\
\hline
   1000
\end{array}
\]
This is 13 - 5 = 8 in base 10.

\[
\begin{array}{r}
   110\\
- \,1\\
\hline
   101
\end{array}
\]
This is 6 - 1 = 5 in base 10.

\[
\begin{array}{r}
   1000\\
- \,1\\
\hline
   111
\end{array}
\]
This is 8 - 1 = 7 in base 10.

\end{problem}

\begin{problem} %22
Binary multiplication

0 multiplied by anything is 0 and 1 multiplied by 1 is 1. In the binary case we have\\
0 * 0 = 0 this is so because $(0 * 2^0) * (0 * 2^0)$\\
0 * 1 = 0 this is so because $(0 * 2^0) * (1 * 2^0)$\\
1 * 1 = 1 this is so because $(1 * 2^0) * (1 * 2^0)$\\

\[
\begin{array}{r}
   1101\\
\times \,1010\\
\hline
   0000\\
  11010\\
  000000\\
+ \,1101000\\
\hline
  10000010
\end{array}
\]

In base 10 this is $13 * 10 = 130$.

\end{problem}



\begin{problem} %23
Binary Division

This is same as long division in base 10.

\[
11001_2 \div 101_2 = 101_2 \text{ remainder } 0_2
\]

This is $\frac{25}{5}$ in base 10

\end{problem}



\begin{problem} %24
Binary fractions

For fractions in binary the only difference from that in base 10 is that only when the denominator in binary division
is a power of 2 that is of the form $2^n$ then we get a terminating fraction else we do not.

$\frac{1}{3}$ is 0.3333.. in decimal. In Binary we can do a division of $\frac{1}{11}$. 

\[
\frac{1_2}{11_2} = 0.\overline{01}_2 = \frac{1}{3}.
\]

\end{problem}

%%%%%%%%%%%%%%%%%%%%%%%%%%%%%%%%%%%%%%%%%
% Chapter End
%%%%%%%%%%%%%%%%%%%%%%%%%%%%%%%%%%%%%%%%%




%%%%%%%%%% 8
\section{The commutative law}
No problems
%%%%%%%%%%%%%%%%%%%%%%%%%%%%%%%%%%%%%%%%%
% Chapter End
%%%%%%%%%%%%%%%%%%%%%%%%%%%%%%%%%%%%%%%%%

%%%%%%%%%% 9
\section{The associative law}

\begin{problem} %25
  Tried it. Beg to differ from Gelfand and Shen on this one. The flavor and aroma are different
  in the two processes described in the equation.
\end{problem}

\begin{problem} %26
  First do the addition of 17999 + 1 to get 18000 then add 357 since the last 3 digits of 18000
  are 0s. The answer is 18357.
\end{problem}

\begin{problem} %27
In such cases add 1 and subtract 1 at the end. So we get 18357 from the same steps as problem 26
then we subtract 1. The answer is 18356.
\end{problem}

\begin{problem} %28
  Here we add 899 + 101 first. It is 900 + 100 which is a thousand. 1000 + 1343 is 2343.
\end{problem}

\begin{problem} %29
  $25 . 4$ is done first to give 100. Then then answer is 3700.
\end{problem}

\begin{problem} %30
  In this we do $125.8$ first which again gives us 1000. The final answer is 37000.
\end{problem}

%%%%%%%%%%%%%%%%%%%%%%%%%%%%%%%%%%%%%%%%%
% Chapter End
%%%%%%%%%%%%%%%%%%%%%%%%%%%%%%%%%%%%%%%%%


%%%%%%%%%% 10
\section{The use of parentheses}

\begin{problem} %31
This is not a useful problem from an algebra perspective. This is a combinatorics problem and a good one. Let us try and build a reasoning around how to solve for the simplest cases. 

For every case we need to partition the numbers into 2 groups. Each group will have to stand on
its own. And in each of these sub groups we have a situation for which we would have already done the count prior.

Case 0: No number - We do not need to put any parentheses.
\begin{center}
0 $\rightarrow$ 0 
\end{center}

Case 1: 2 - We do not need any parentheses but can put one like (2).
\begin{center}
1 $\rightarrow$ 1 
\end{center}

Case 2: 2.3 - Just one like (2.3)
\begin{center}
2 $\rightarrow$ 1 
\end{center}

Case 3: 2.3.4 - Partitioning into smaller groups gives a group of 2 and 1. It is (2.3).4 and 2.(3.4)
Thus we get 1 + 1 = 2.\\
Case 3 $\rightarrow$ $1 + 2$ - 2.(3.4) $\rightarrow$ $1 + 1$\\
\begin{center}
3 $\rightarrow$ 2 
\end{center}

Case 4: 2.3.4.5 - The book solves this. Let us make partitions 2.(3.4.5). We notice that we have
simplified it to case 3 here which will repeat twice. The other partition is (2.3).(4.5) which is
two cases before i.e. case 2. So we get the following:
$2 + 2$ from Case 3 + $1$ from Case 1.\\
Case 4 $\rightarrow$ $1 + 3$ - 2.(3.4.5) $\rightarrow$ $2 + 2$\\
Case 4 $\rightarrow$ $2 + 2$ - (2.3).(4.5) $\rightarrow$ $1$\\
\begin{center}
4 $\rightarrow$ 5 
\end{center}

Case 5: 2.3.4.5.6 - This is the question we are asked. The algorithm requires us to partition.
2.(3.4.5.6) is one way to partition and we have reduced the problem to Case 4 above which repeats 
twice. The other partition is (2.3).(4.5.6). In this case we go two steps back.
So we get $5 + 5 + 2 + 2$.\\
Case 5 $\rightarrow$ $1 + 4$ - 2.(3.4.5.6) $\rightarrow$ $5 + 5$\\
Case 5 $\rightarrow$ $2 + 3$ - (2.3).(4.5.6) $\rightarrow$ $2 + 2$\\
\begin{center}
5 $\rightarrow$ 14
\end{center}

Case 6: 2.3.4.5.6.7 - Let us take it a notch higher. The sub cases which can be built are\\
Case 6 $\rightarrow$ $1 + 5$ - 2.(3.4.5.6.7) $\rightarrow$ $14 + 14$\\
Case 6 $\rightarrow$ $2 + 4$ - (2.3).(4.5.6.7) $\rightarrow$ $5 + 5$\\
Case 6 $\rightarrow$ $3 + 3$ - (2.3.4).(5.6.7) $\rightarrow$ $2 \times 2$\\
\begin{center}
6 $\rightarrow$ 42
\end{center}

In fact we can even make it generic. Essentially what we are doing is partitioning numbers.
Then for each partition we are looking back at previous permutations and adding. Note that in some cases
we need to multiply too (for instance last sub case in Case 6). Can we generalize this? Yes, these are
basically \href{https://en.wikipedia.org/wiki/Catalan_number}{Catalan} numbers! The $nth$ Catalan number is given by the expression for all $n\ge 0$
\[
\frac{(2n)!}{(n + 1)! \; n!}
\]
\end{problem}

\begin{problem} %32
This too is a combinatorics problem. 

Let us take a simple trivial case of the problem.\\
2 $\rightarrow$ we do not need any parentheses\\
2.3 $\rightarrow$ we still do not need any parentheses\\
2.3.4 $\rightarrow$ now we need to put the first pair so that it becomes (2.3).4. So for $n = 3$ digits we have 2 i.e. $2(n-2)$ parentheses.\\
2.3.4.5 $\rightarrow$ we need to put one additional pair ((2.3).4).5. So for $n = 4$ digits we have 4 i.e. $2(n-2)$ parentheses.\\


Generalizing, for $n>2$ number of parentheses is
\[
2(n - 2)
\]

In the given question we have the question as 2.3.4.5.....97.98.99.100. These are $(100 - 2) + 1$ digits.
Therefore $n = 99$ and total number of parentheses are $2 \times (99 - 2)$ which is 194.\\

Answer is 194.

\end{problem}

\begin{problem} %33
This is easily doable like the way the child Gauss did. Basically there are 100 elements in
this series

\[
\begin{array}{rcl}
S   &=& 1 + 2 + 3 + \ldots + 99 + 100 \\
+ S   &=& 100 + 99 + 98 + \ldots + 2 + 1 \\
\hline
2S  &=& 101 + 101 + 101 + \ldots + 101\\
\end{array}
\]
\\
\[
2S = 100\times 101 \\
\]
\\
\[
S = \dfrac{100\times 101}{2} = 5050
\]

\end{problem}

%%%%%%%%%%%%%%%%%%%%%%%%%%%%%%%%%%%%%%%%%
% Chapter End
%%%%%%%%%%%%%%%%%%%%%%%%%%%%%%%%%%%%%%%%%


%%%%%%%%%% 11
\section{The distributive law}
\begin{problem} %34
\begin{gather*}
= 1001 \times 20 \\
= (1000 + 1) \times 20 \\
= 20000 + 20 \\
= 20020
\end{gather*}

In this we could simply multiply 1001 by 2 to get 2002 and then append a 0 behind it too.

\end{problem}

\begin{problem} %35
\begin{gather*}
= 1001 \times 102 \\
= (1000 + 1) \times 102 \\
= 102000 + 102 \\
= 102102
\end{gather*}
\end{problem}

\begin{problem} %36
\begin{gather*}
(a + b + c + d + e) (x + y + z)
\end{gather*}
For each number from a to e we will get 3 terms one for each x, y and z. Thus total of 3 * 5 terms. 15 terms.
\end{problem}

%%%%%%%%%%%%%%%%%%%%%%%%%%%%%%%%%%%%%%%%%
% Chapter End
%%%%%%%%%%%%%%%%%%%%%%%%%%%%%%%%%%%%%%%%%

%%%%%%%%%% 12
\section{Letters in algebra}
\begin{problem} %37
Let small vessel volume be $x$ and big vessel volume be $y$, then
\[
x + y = 5
\]
\[
2x + 3y = 13
\]
Solving these two linear equations we get $y = 3$ and $x = 2$. To solve such equations make the
coefficients of one of the unknown same and subtract one equation from the other.
\end{problem}

\begin{problem}%38
The simple explanation is given in the book.
\begin{gather*}
x \\
x + 3 \\
2 . (x + 3) \\
(2x + 6) - x \\
(x + 6) - 4 = x + 2 \\
(x + 2) - x \\
2 
\end{gather*}
\end{problem}

%%%%%%%%%%%%%%%%%%%%%%%%%%%%%%%%%%%%%%%%%
% Chapter End
%%%%%%%%%%%%%%%%%%%%%%%%%%%%%%%%%%%%%%%%%


%%%%%%%%%% 13
\section{The addition of negative numbers}
No problems
%%%%%%%%%%%%%%%%%%%%%%%%%%%%%%%%%%%%%%%%%
% Chapter End
%%%%%%%%%%%%%%%%%%%%%%%%%%%%%%%%%%%%%%%%%

%%%%%%%%%% 14
\section{The multiplication of negative numbers}
No problems
%%%%%%%%%%%%%%%%%%%%%%%%%%%%%%%%%%%%%%%%%
% Chapter End
%%%%%%%%%%%%%%%%%%%%%%%%%%%%%%%%%%%%%%%%%

%%%%%%%%%% 15
\section{Dealing with fractions}
\begin{problem} %39
The explanation of this problem is humorous! But such a nice way to put it.

If only vodka bottles were used in the explanation everything would have fallen in place
with this soviet era content.

Anyways.

$\frac{1}{3} \times \frac{7}{7}$ and $\frac{2}{7} \times \frac{3}{3}$ \\


$\frac{7}{21}$ and $\frac{6}{21}$ \\

We conclude that $\frac{1}{3}$ is bigger.

\end{problem}

\begin{problem} %40
This is a problem also found in the initial assessment for the Gelfand Correspondence Program.

\begin{gather*}
\frac{10001}{10002} = 1 - \frac{1}{10002}
\end{gather*}

Similarly,

\begin{gather*}
\frac{100001}{100002} = 1 - \frac{1}{100002}
\end{gather*}

As we can see the $\frac{1}{100002}$ much more smaller than $\frac{1}{10002}$. Thus
when we subtract a smaller number from 1 we are left with a bigger number.

Answer is $\frac{100001}{100002}$
\end{problem}

\begin{problem} %41
Good problem.

\begin{gather*}
\frac{12345}{54321} \times \frac{54322}{54322}\\
\end{gather*}

\begin{gather*}
\frac{12346}{54322} \times \frac{54321}{54321}\\
\end{gather*}

Now lets only look at the numerators since the denominators are equal.

\begin{gather*}
(12345) \times (54321 + 1) \\
(12345 + 1) \times (54321)
\end{gather*}

Make them both have common terms

\begin{gather*}
(12345 \times 54321) + 12345 \\
(12345 \times 54321) + 54321
\end{gather*}

It is clear now that the second fraction is bigger because it has 54321 in the numerator vs 
12345 in the first fraction when all other terms are same in numerator and denominator.

Answer is $\frac{12346}{54322}$
\\

Just to verify this in Scheme. We get a positive fraction when we subtract the first fraction
from the second one.

\begin{center}
\begin{lstlisting}
> (- (/ 12346 54322) (/ 12345 54321))
6996/491804227
\end{lstlisting}
\end{center}

\end{problem}

\begin{problem} %42
These 3 problems are pretty difficult actually for middle school kids.

(a) We will use proof by contradiction to prove this.

Assume the greatest common divisor of $a$ and $b$ is $m$.
\[
gcd(a,b) = m, m>1
\]

So $m | a$ and $m | b$ ($x|y$ reads \textit{x divides y})

Therefore $m$ should also divide $ad - bc$ i.e. $m|(ad -bc)$

But we know $ad - bc = \pm 1$. So in $\frac{(ad - bc)}{m}$ denominator has to be $+1$

Thus $m$ is $+1$ and $gcd(a,b) = 1$

Hence $\frac{a}{b}$ cannot be further simplified. The same proof can be used for $\frac{c}{d}$.

(b) Now to second part of this problem. I struggled with this one quite a bit. A little bit
background on \href{https://en.wikipedia.org/wiki/Farey_sequence}{Farey Sequence}. Niven and Zuckerman (1972)
defined Farey Sequence as
\begin{quotation}
  \textit{The sequence of all reduced fractions with denominators not exceeding n, listed in order of their size, is called the Farey sequence of order n.}
\end{quotation}
Sometimes the definition is restricted to teh interval 0 to 1. In this interval say we look at
Farey number 3 which is given by 
\[
F_3 = \Bigl\{ \frac{0}{1}, \frac{1}{3}, \frac{1}{2}, \frac{2}{3}, \frac{1}{1} \Bigl\}
\]
The third term minus the second term here is
\[
\frac{1}{2} - \frac{1}{3} = \frac{1}{6}
\]
The numerator is 1. In Farey Sequences the numerator on subtraction of two consecutive elements is always $\pm1$

The problem says that in a Farey sequence of $\frac{a}{b}$ and $\frac{c}{d}$ the fraction $\frac{a+b}{c+d}$ is always between
$\frac{a}{b}$ and $\frac{c}{d}$. Let us plug in the numbers from the $F_3$ sequence quoted above.

\[
\frac{a+b}{c+d} = \frac{1 + 3}{1 + 2} = \frac{4}{3}
\]

So the inequality is now

\[
\frac{1}{3} < \frac{4}{3} < \frac{1}{2}
\]

But $\frac{4}{3}$ is greater than 1 and does not lie there! The point is that there is a typographical mistake in this problem.
The actual fraction between $\frac{a}{b}$ and $\frac{c}{d}$ is called the Mediant fraction and is given as
\[
\frac{a}{b} < \frac{a + c}{b + d}<\frac{c}{d}
\]

To verify
\[
\frac{1}{3} < \frac{2}{5} < \frac{1}{2}
\]

And this is correct.

Back to part (b) of the problem now. We need to prove the below.
\[
\frac{a}{b} < \frac{a + c}{b + d}<\frac{c}{d}
\]

Suppose there existed a fraction $\frac{p}{q}$ in between $\frac{a}{b}$ and $\frac{c}{d}$, then
\[
\frac{a}{b} < \frac{p}{q}<\frac{c}{d}
\]
then, since they are Farey neighbors.
\[
pb - aq = 1
\]
\[
cq - pd = 1
\]
\begin{gather*}
pb - aq = cq - pd\\ 
pb + pd = cq + aq\\
p (b + d) = q (a +c)\\
\frac{p}{q} = \frac{a+c}{b+d}
\end{gather*}

Hence we proved that the mediant fraction between $\frac{a}{b}$ and $\frac{c}{d}$ has to be $\frac{a+c}{b+d}$

(c) Last part of this challenging problem. We have say
\[
\frac{a}{b} < \frac{e}{f}<\frac{c}{d}
\]
\begin{gather*}
\frac{e}{f} - \frac{a}{b} = \frac{be - af}{bf}\\
\end{gather*}
Since $be - af$  will be a positive integer and therefore at least 1, we can say
\begin{gather*}
\frac{e}{f} - \frac{a}{b} \ge \frac{1}{bf}\\
\end{gather*}
Similarly we get
\begin{gather*}
\frac{c}{d} - \frac{e}{f} \ge \frac{1}{df}\\
\end{gather*}

Now we have to visualize. Moving from $\frac{a}{b}$ to $\frac{e}{f}$ and from that finally to $\frac{c}{d}$.
Total Distance is $\frac{1}{bd}$ (from the denominator between $\frac{a}{b}$ and $\frac{c}{d}$). So we get
\[
\frac{1}{bd} = \Bigl( \frac{e}{f} - \frac{a}{b} \Bigl) + \Bigl( \frac{c}{d} - \frac{e}{f} \Bigl)
\]

But we know from the inequalities above.
\[
\frac{1}{bd} \ge  \frac{1}{bf} + \frac{1}{df} = \frac{b + d}{bdf}
\]
\[
\frac{1}{bd} \ge  \frac{b + d}{bdf}
\]
\[
1 \ge  \frac{b + d}{f}
\]
\[
f \ge  b + d
\]

Hence proved that $f$ cannot be less than $b + d$

Difficult problems for middle schoolers!
\end{problem}

\begin{problem}
This is application of mediant formula we derived in problem 42.

\[
\frac{a}{b} < \frac{a + c}{b + d}<\frac{c}{d}
\]

The end 2 pieces of $\frac{1}{20}$ can be omitted so we are left with 18 pieces. There are 6 red marks (7 equal segments will require 6 marks) and
12 green marks (13 equal segments). We can visualize from the left of the stick the first cut would be at $\frac{1}{20}$ which we have omitted
then the next cut will be at $\frac{1}{13}$ and then at $\frac{1}{7}$. Let us find mediant between $\frac{1}{13}$ and $\frac{1}{7}$.
\[
\frac{1}{13} < \frac{1 + 1}{13 + 7}<\frac{1}{7}
\]
\[
\frac{1}{13} < \frac{2}{20}<\frac{1}{7}
\]

Note that the mediant will always lie at $\frac{k}{20}$. Thus the cut will be there and each of the 18 pieces will have
only color either green or red.

The problem is solved in the book.

\end{problem}

\begin{problem} %44
First expression is
\begin{gather*}
5\% \times 7 \times 10^9\\
35 \times 10^7
\end{gather*}

Second expression is
\begin{gather*}
7\% \times 5 \times 10^9\\
35 \times 10^7
\end{gather*}

Thus they are equal.

\end{problem}

\begin{problem} %45
The more systematic way to reason is to say what number $k$ when multiplied by $\frac{2}{3}$ gives $\frac{1}{2}$.

\begin{gather*}
k \times \frac{2}{3} = \frac{1}{2}\\
k = \frac{1}{2} \times \frac{3}{2} = \frac{3}{4}
\end{gather*}

So we fold the string in half once, then re fold it. We get quarter of the original two thirds. Now we cut off one of the one fourths and then
we are left with three fourths of the original two thirds which is now half.

The problem is solved in the book.
\end{problem}

%%%%%%%%%%%%%%%%%%%%%%%%%%%%%%%%%%%%%%%%%
% Chapter End
%%%%%%%%%%%%%%%%%%%%%%%%%%%%%%%%%%%%%%%%%

  

%%%%%%%%%% 16
\section{Powers}

\begin{problem} %46
(a) 1024 is the answer. It is good to have the powers of 2 memorized for quick computation - 1, 2, 4, 8, 16, 32, 64, 128, 256, 512, 1024 (for folks
interested in computers this will be second nature).

(b) 1000 - a thousand.

(c) 10000000 - 10 million - In India this is also called a Crore.

\end{problem}

\begin{problem} %47
Assuming by \textit{decimal digits} the authors mean digits. Then the answer is 10001. 1000 zeroes and a one 1.

\end{problem}

\begin{problem} %48
Total number of seconds in a year

\[
60 \times 60 \times 24 \times 365 = 31536000
\]

Distance traveled in 4 light years will be

\[
4 \times 3 \times 10^5 \times 31536000 = 37843200000000
\]

That is a whopping 37.8432 Trillion Kilometers 

\end{problem}

%%%%%%%%%%%%%%%%%%%%%%%%%%%%%%%%%%%%%%%%%
% Chapter End
%%%%%%%%%%%%%%%%%%%%%%%%%%%%%%%%%%%%%%%%%


%%%%%%%%%% 17
\section{Big numbers around us}

\begin{problem}
(a) $2^{10}$ is 1024 so $2^{20} = 2^{10} \times 2^{10} = 1024 \times 1024$\\
This will 1 followed by 6 other digits. So total digits will be 7.

Verified in Scheme
\begin{lstlisting}
> (expt 2 20)
1048576
\end{lstlisting}

(b) $2^{100}$ = $2^{10} \times 2^{10}$ ....ten times\\
$1024 \times 1024 \times 1024$..... ten times\\
$(10^3 + 24)^2 \times (10^3 + 24)^2$....five times\\
$(1000000 + 576 + 48000) \times (1000000 + 576 + 48000)$ five times\\
$1048576 \times 1048576 \times 1048576 \times 1048576 \times 1048576 $\\
So we get 31 digits. Why? Because the $1000000$ will grow much bigger than the $48576$. This is intuitively speaking.
Ideally we should do it properly. Let us try that.\\

A natural number $N$ is given. How many digits does this have?\\
If it is between 1 (included) to less than 10 it has 1 digit. If it is from 10 (included) to less than 100 then it has 2 digits.
If it is between 100 (included) to less than 1000 it has 3 digits and so on. Representing it in inequality form.\\
\begin{gather*}
10^0 \le N < 10^1 \; \rightarrow 1 \;digit\\
10^1 \le N < 10^2 \; \rightarrow 2 \;digits\\
10^2 \le N < 10^3 \; \rightarrow 3 \;digits\\
10^3 \le N < 10^4 \; \rightarrow 4 \;digits\\
.....\\
10^{k-1} \le N < 10^k \; \rightarrow k \;digits\\
\end{gather*}

So we have an inequality for the number of digits $k$ for a number N. 
\[
10^{k-1} \le N < 10^k
\]

Let us take logarithms on both sides for the number $2^n$. I understand students would not have been taught this yet. But I am sure if kids
are working through this book they are smart enough to pick this up.
\begin{gather*}
\log_{10} 10^{k-1}  \le \log_{10} 2^n < \log_{10} 10^k\\
k - 1 \le n\log 2 < k
\end{gather*}
rearranging the inequality above we get
\begin{gather*}
n \log 2 < k \le n \log 2 + 1
\end{gather*}
To reiterate the number $2^n$ will have $k$ digits and this $k$ is certainly bigger than $n \log 2$ and an integer and also lesser than or equal to $n \log 2 + 1$.\\
We can write $k$ as 
\[
k = \lfloor n \log 2 \rfloor + 1
\]
So we conclude that the number $2^n$ has $\lfloor n \log 2 \rfloor + 1$ digits.\\

Let us try out some examples.\\

$2^{10} = \lfloor 10 \log 2 \rfloor + 1 = \lfloor 10 \times 0.30102999 \rfloor + 1 = 3 + 1 = 4$\\
Indeed $2^{10} = 1024$ which has 4 digits.\\
Note value of $\log 2$ is $0.3010999$.\\

$2^{100} = \lfloor 100 \log 2 \rfloor + 1 = \lfloor 100 \times 0.30102999 \rfloor + 1 = 30 + 1 = 31$\\
We can verify with certainty that now the answer to this question is $31$.\\

(c) For this part of the problem I will have to use programming. I am unsure how else to construct this graph.\\
Using Racket (a Scheme)
\begin{lstlisting}
#lang racket
(require plot)

;; Function for the actual computation
(define (digits-of-2^n n)
  (add1 (floor (/ (* n (log 2)) (log 10)))))

;; Plot digits for n from 1 to 1000
(plot
 (function digits-of-2^n 1 1000)
 #:x-label "n"
 #:y-label "digits of 2^n"
 #:title  "Number of decimal digits in 2^n")
\end{lstlisting}

\begin{figure}[H]
\centering
\includegraphics[width=0.7\linewidth]{images/problem49c.png}
\label{fig:digits2n}
\end{figure} 

\end{problem}

%%%%%%%%%%%%%%%%%%%%%%%%%%%%%%%%%%%%%%%%%
% Chapter End
%%%%%%%%%%%%%%%%%%%%%%%%%%%%%%%%%%%%%%%%%


%%%%%%%%%% 18
\section{Negative powers}
\begin{problem} %50
(a) $\frac{1}{10}$ or 0.1 \\

(b) $\frac{1}{100}$ or 0.01 \\

(c) $\frac{1}{1000}$ or 0.001 \\

\end{problem}

%%%%%%%%%%%%%%%%%%%%%%%%%%%%%%%%%%%%%%%%%
% Chapter End
%%%%%%%%%%%%%%%%%%%%%%%%%%%%%%%%%%%%%%%%%

%%%%%%%%%% 19
\section{Small numbers around us}

\begin{problem} %51
As per notation yes both are true.\\

\begin{gather*}
a^{-n} = \frac{1}{a^n}\\
a^{-(-k)} = \frac{1}{a^{-k}}\\
a^k = \frac{1}{a^{-k}}
\end{gather*}
$a^0$ is 1 so $\frac{1}{1}$ is 1 again.
\end{problem}

\begin{problem} %52
(a) $a^{10}b^4$\\

(b) $2.a^3b^{-2}$\\
\end{problem}

\begin{problem} %53
(a) $\frac{a^3}{b^5}$ \\

(b) $\frac{1}{a^2b^7}$ \\
\end{problem}

%%%%%%%%%%%%%%%%%%%%%%%%%%%%%%%%%%%%%%%%%
% Chapter End
%%%%%%%%%%%%%%%%%%%%%%%%%%%%%%%%%%%%%%%%%

%%%%%%%%%% 20
\section{How to multiply $a^m$ by $a^n$, or why our definition is convenient}

\begin{problem} %54
Not sure of the ask of this question. Probably the answer is  $a^{m-n}$.
\end{problem}

%%%%%%%%%%%%%%%%%%%%%%%%%%%%%%%%%%%%%%%%%
% Chapter End
%%%%%%%%%%%%%%%%%%%%%%%%%%%%%%%%%%%%%%%%%




%%%%%%%%%% 21
\section{The rule of multiplication for powers}
\begin{problem} %55
  (a) $n = 2000 - 1001 = 999$ \\

  (b) $1001 + n = -2$ thus $n = -1003$ \\

  (c) $\frac{1}{1000}$ vs $\frac{1}{1024}$. Thus $10^{-3}$ is bigger.\\

  (d) $1000 - n = 501$, $n = 499$ \\

  (e) $1000 - n = -4$, $n = 1004$ \\

  (f) $2 \times 100 = n$, $n =200$ \\

  (g) $(2 \times 3)^{100} = a^{100}$, $a = 6$ \\

  (h) $10 \times 15 = n$, $n = 150$\\

\end{problem}

\begin{problem} %56
It does not matter what sign $m$ and $n$ have as such. Specifically if $m > 0$ and $n < 0$ then
$(a^m)^{-n} = \frac{1}{(a^m)^n}$.\\

For either of them to be zero the answer would be 1 since one of the powers is zero.
\end{problem}

\begin{problem} %57
Again signs do not make any difference to the formula $(ab)^n = a^n . b^n$
\end{problem}

\begin{problem} %58
  If $a = 0$ then it will $0$.\\
  If $a > 0$ then it will be $-a^{775}$\\
  If $a < 0$ then it will be $a^{775}$
\end{problem}

\begin{problem} %59
  Ideally $b \ne 0$ is the first call out.\\
  Otherwise it is easy to put any number whether integer or fraction as $n$ here.\\
  So not sure of the intent of the problem in this case.
\end{problem}

\begin{problem} %60
  We can manipulate the base $4^{\frac{1}{2}}$ to $(2^2)^{\frac{1}{2}}$. Thus this can be simplified to
  $2^{(2 \times \frac{1}{2})}$ giving $2^1$. But we need to be careful here since $-2 \times -2 = (-2)^2$ also.
  Therefore the answer will $\pm 2$.\\

  Similarly for $27^{\frac{1}{3}}$ should give the third root because of $3 \times 3 \times 3$ giving $3^3$. Here we will not get $-3$ else that would
  make the answer negative and incorrect.
\end{problem}


%%%%%%%%%%%%%%%%%%%%%%%%%%%%%%%%%%%%%%%%%
% Chapter End
%%%%%%%%%%%%%%%%%%%%%%%%%%%%%%%%%%%%%%%%%


%%%%%%%%%% 22
\section{Formula for short multiplication: The square of a sum}

\begin{problem} %61
  Application of $(a + b)^2 = a^2 + b^2 + 2ab$\\

(a) 
\begin{gather*}
  101^2\\
  = (100 + 1) ^2\\
  = 10000 + 1 + 200\\
  = 10201
\end{gather*}

(b)

\begin{gather*}
   1002^2\\
  = (1000 + 2) ^2\\
  = 1000000 + 4 + 4000\\
  = 1004004
\end{gather*}

\end{problem}

\begin{problem} %62
  Let product $p$ be\\
\[
p = m \times n
\]

Now $m$ and $n$ the factors become $10\%$ bigger\\

\begin{gather*}
  (m + 10\% m) \times (n + 10\% n)\\
  1.1m \times 1.1n\\
  1.21 m \times n\\
  1.21 p\\
  (p + 21\% p)
\end{gather*}

Thus the product becomes $21\%$ bigger.
\end{problem}

\begin{problem} %63
  This question is to drive home the point made in the text that "The
  square of the sum of two terms is the sum of their squares plus two times the product of the terms."

  The core message is that \textit{square of the sum} and \textit{sum of the squares} are
  two different things. Rightly so. Students need to be careful, that is all.

  The answer to the problem is `No'. Why?\\

  Case 1: NN is me a man. I, the father, have a son. The father of the son is me. So this refers to me.
  Now my father has a son but he could have more than one son. In my case we are actually two brothers. So not always true.

  Case 2: NN is my wife. My wife's son has a father which is me. But I am not my wife. So this is incorrect already.
  My wife's father does have a son who is my wife's brother. But my brother in law and wife are not the same person.

  Luckily my real family is good to answer this question!
\end{problem}


%%%%%%%%%%%%%%%%%%%%%%%%%%%%%%%%%%%%%%%%%
% Chapter End
%%%%%%%%%%%%%%%%%%%%%%%%%%%%%%%%%%%%%%%%%

%%%%%%%%%% 23
\section{How to explain the square of the sum formula to your younger brother or sister}

\begin{problem} %64
This is a simple representation of the formula $(a + b)^2 = a^2 + b^2 + 2ab$.\\

\begin{center}
\begin{tikzpicture}[scale=0.9, every node/.style={font=\small}]
  % define lengths
  \def\a{3}
  \def\b{2}

  % outer square (a+b) × (a+b)
  \draw (0,0) rectangle (\a+\b,\a+\b);

  % inner grid lines
  \draw (\a,0) -- (\a,\a+\b);
  \draw (0,\a) -- (\a+\b,\a);

  % color the regions and add labels
  \fill[blue!20] (0,0) rectangle (\a,\a);
  \node at ({\a/2},{\a/2}) {$a^2$};

  \fill[red!20] (\a,\a) rectangle (\a+\b,\a+\b);
  \node at ({\a+\b/2},{\a+\b/2}) {$b^2$};

  \fill[green!20] (\a,0) rectangle (\a+\b,\a);
  \node at ({\a+\b/2},{\a/2}) {$ab$};

  \fill[green!20] (0,\a) rectangle (\a,\a+\b);
  \node at ({\a/2},{\a+\b/2}) {$ab$};

  % dimension markers
  \draw[<->] (0,-0.5) -- (\a,-0.5) node[midway,below] {$a$};
  \draw[<->] (\a,-0.5) -- (\a+\b,-0.5) node[midway,below] {$b$};
  \node at ({0.5*(\a+\b)}, -1.1) {$a+b$};

  \draw[<->] (-0.5,0) -- (-0.5,\a) node[midway,left] {$a$};
  \draw[<->] (-0.5,\a) -- (-0.5,\a+\b) node[midway,left] {$b$};
  \node[rotate=90] at (-1.1,{0.5*(\a+\b)}) {$a+b$};
\end{tikzpicture}
\end{center}
\end{problem}

\begin{problem} %65
  (a) $99^2 = (100 - 1)^2 = 10000 + 1 - 200 = 9801 $\\

  (b) $998^2 = (1000 - 2)^2 = 1000000 + 4 - 4000 = 996004 $\\
\end{problem}

\begin{problem} %66

  (a) When $a=b$ then\\
  The square of the sums gives $4a^2$ or $4b^2$\\
  The square of the difference gives $0$\\

  (b) When $a=2b$ then\\
  The square of the sums gives $\frac{9}{4}a^2$ or $9b^2$\\
  The square of the difference gives $\frac{a^2}{4}$ or $b^2$\\
  
\end{problem}

%%%%%%%%%%%%%%%%%%%%%%%%%%%%%%%%%%%%%%%%%
% Chapter End
%%%%%%%%%%%%%%%%%%%%%%%%%%%%%%%%%%%%%%%%%

%%%%%%%%%% 24
\section{The difference of squares}

\begin{problem} %67
\begin{gather*}
(a + b) (a - b) = a^2 \cancel{- ab} \cancel{+ ba} - b^2 = a^2 - b^2
\end{gather*}
  
\end{problem}

\begin{problem} %68
  \begin{gather*}
    101 \times 99 = (100 + 1) (100 - 1) = 100^2 - 1^2 = 9999
  \end{gather*}
\end{problem}

\begin{problem} %69
  We just cut it vertically as shown with the dotted line and then stack the two rectangles with
  $(a-b)$ side matching.

\begin{center}
\usetikzlibrary{calc}

\begin{tikzpicture}[scale=0.8, every node/.style={font=\small}]
  %--------------------------------
  % Parameters
  %--------------------------------
  \def\a{4}     % side of big square
  \def\b{1.5}   % removed square side

  %================================
  % FIRST FIGURE: L-shaped region
  %================================

  %--- Shade the left tall rectangle (a-b) × a
  \fill[blue!15] (0,0) rectangle (\a-\b,\a);

  %--- Shade the lower right rectangle b × (a-b)
  \fill[green!20] (\a-\b,0) rectangle (\a,\a-\b);

  %--- The removed b × b square stays white
  \fill[white] (\a-\b,\a-\b) rectangle (\a,\a);
  \draw (\a-\b,\a-\b) rectangle (\a,\a); % outline it

  %--- Draw the L-shape outline
  \draw (0,0) --
        (\a,0) --
        (\a,\a-\b) --
        (\a-\b,\a-\b) --
        (\a-\b,\a) --
        (0,\a) -- cycle;

  %--- Dotted cut line at x = a - b
  \draw[dashed] (\a-\b,0) -- (\a-\b,\a-\b);

  %--- Labels
  \node[left]  at ($(0,0)!0.5!(0,\a)$) {$a$};
  \node[below] at ($(0,0)!0.5!(\a,0)$) {$a$};
  \node[right] at ($( \a-\b,\a-\b)!0.5!(\a-\b,\a)$) {$b$};
  \node[above] at ($( \a-\b,\a-\b)!0.5!(\a,\a-\b)$) {$b$};


  %================================
  % SECOND FIGURE: rearranged rectangle
  %================================
  \begin{scope}[xshift=8.5cm]

    %--- Blue rectangle (a-b) × a
    \fill[blue!15] (0,0) rectangle (\a-\b,\a);

    %--- Green rectangle (a-b) × b on top
    \fill[green!20] (0,\a) rectangle (\a-\b,\a+\b);

    %--- Outer boundary of total rectangle (a-b) × (a+b)
    \draw (0,0) rectangle (\a-\b,\a+\b);

    %--- Line dividing a and b parts
    \draw (0,\a) -- (\a-\b,\a);

    %--- Side labels
    \node[below] at ($(0,0)!0.5!(\a-\b,0)$) {$a-b$};
    \node[left]  at ($(0,0)!0.5!(0,\a+\b)$) {$a+b$};

    %--- Internal labels
    \node[right] at (\a-\b,0.5*\a) {$a$};
    \node[right] at (\a-\b,\a+0.5*\b) {$b$};
  \end{scope}

\end{tikzpicture}
\end{center}

\end{problem}

\begin{problem} %70
  Let the larger number be $n$ then the other number will be $(n - 2)$. We can write:
  \begin{gather*}
    n (n - 2) + 1 = n^2 - 2n + 1 = (n - 1)^2
  \end{gather*}
  $(n - 1)^2$ is a perfect square and the number $(n - 1)$ is between $n$ and $(n - 2)$.
\end{problem}

\begin{problem} %71
  The difference between the squares of two consecutive numbers $n$ and $(n + 1)$ is
  \begin{gather*}
    (n + 1)^2 - n^2 = n^2 + 1 + 2n - n^2 = 2n + 1
  \end{gather*}

  Now difference between the squares of the next two consecutive numbers $(n + 1)$ and $(n + 2)$ is
  \begin{gather*}
    (n + 2)^2 - (n + 1)^2 = n^2 + 4 + 4n - n^2 - 1 - 2n = 2n + 3
  \end{gather*}

  So the difference between the two differences is
  \begin{gather*}
    (2n + 3) - (2n + 1) = 3 - 1 = 2
  \end{gather*}

  $2$ is the constant difference. This is called an arithmetic progression.
\end{problem}

\begin{problem} %72
  This is a nice trick. Let a number be of the form $n5$. This is a two digit number but we can extend
  the logic for higher digit numbers.

  $n5$ can be written as $10n + 5$.

  So the square will be $(10n + 5)^2$. This can be rearranged as.

  \begin{gather*}
    (10n + 5)^2 = 100n^2 + 25 + 100n = 100n (n + 1) + 25
  \end{gather*}

  The $100 n (n + 1)$ is a number $n$ times $(n + 1)$ i.e. two consecutive numbers. Multiplying
  by 100 gives it the correct place in decimal value system as thousandth for this two digit square.
  We already have the left over 25 for the ending two digits. So we get
  \begin{gather*}
    (n5)^2 = (n \times (n + 1))25
  \end{gather*}

  Thus we can prove this trick.
\end{problem}

\begin{problem} %73
  \begin{gather*}
    (a + b + c)^2 = (a + b + c) \times (a + b + c)\\
    = a^2 + ab + ac + ab + b^2 + bc + ac + bc + c^2\\
    = a^2 + b^2 + c^2 + 2 (ab + bc + ca)
  \end{gather*}

  Visually we see it as below.
  \begin{center}
\begin{tikzpicture}[scale=0.9, every node/.style={font=\small}]
  %--------------------------------
  % Parameters
  %--------------------------------
  \def\a{3}
  \def\b{2}
  \def\c{1.5}

  %================================
  % FIRST: SHADING (background)
  %================================

  % Row 1 (height a)
  \fill[blue!20] (0,\b+\c) rectangle (\a,\a+\b+\c);             % a^2
  \fill[green!20] (\a,\b+\c) rectangle (\a+\b,\a+\b+\c);        % ab
  \fill[green!20] (\a+\b,\b+\c) rectangle (\a+\b+\c,\a+\b+\c);  % ac

  % Row 2 (height b)
  \fill[green!20] (0,\c) rectangle (\a,\b+\c);                  % ba
  \fill[red!20]   (\a,\c) rectangle (\a+\b,\b+\c);              % b^2
  \fill[green!20] (\a+\b,\c) rectangle (\a+\b+\c,\b+\c);        % bc

  % Row 3 (height c)
  \fill[green!20] (0,0) rectangle (\a,\c);                      % ca
  \fill[green!20] (\a,0) rectangle (\a+\b,\c);                  % cb
  \fill[purple!20] (\a+\b,0) rectangle (\a+\b+\c,\c);           % c^2


  %================================
  % SECOND: GRID LINES (foreground)
  %================================

  % Outer boundary
  \draw[thick] (0,0) rectangle (\a+\b+\c,\a+\b+\c);

  % Vertical grid lines
  \draw[thick] (\a,0) -- (\a,\a+\b+\c);
  \draw[thick] (\a+\b,0) -- (\a+\b,\a+\b+\c);

  % Horizontal grid lines
  \draw[thick] (0,\c) -- (\a+\b+\c,\c);
  \draw[thick] (0,\b+\c) -- (\a+\b+\c,\b+\c);


  %================================
  % LABELS
  %================================

  % Row labels (left)
  \node[left] at (0, \c/2) {$c$};
  \node[left] at (0, \c + \b/2) {$b$};
  \node[left] at (0, \c + \b + \a/2) {$a$};

  % Column labels (top)
  \node[above] at (\a/2, \a+\b+\c) {$a$};
  \node[above] at (\a+\b/2, \a+\b+\c) {$b$};
  \node[above] at (\a+\b+\c/2, \a+\b+\c) {$c$};

  % Cell labels
  \node at (\a/2, \c+\b+\a/2) {$a^2$};
  \node at (\a+\b/2, \c+\b+\a/2) {$ab$};
  \node at (\a+\b+\c/2, \c+\b+\a/2) {$ac$};

  \node at (\a/2, \c+\b/2) {$ab$};
  \node at (\a+\b/2, \c+\b/2) {$b^2$};
  \node at (\a+\b+\c/2, \c+\b/2) {$bc$};

  \node at (\a/2, \c/2) {$ac$};
  \node at (\a+\b/2, \c/2) {$bc$};
  \node at (\a+\b+\c/2, \c/2) {$c^2$};

\end{tikzpicture}
\end{center}
\end{problem}

\begin{problem} %74
  \begin{gather*}
    (a + b - c)^2 = a^2 + b^2 + c^2 + 2(ab - bc -ac)
  \end{gather*}
\end{problem}

\begin{problem} %75
Consider $(a + b)$ as say $A$ so we have $(A + c) (A - c)$. We get
\begin{gather*}
  (A + c) (A - c) = A^2 - c^2\\
  (a + b)^2 - c^2\\
  a^2 + b^2 - c^2 + 2ab
\end{gather*}
\end{problem}

\begin{problem} %76
  Consider $(b + c)$ as say $B$ so we have $(a + B) (a - B)$. We get
  \begin{gather*}
    (a + B) (a - B) = a^2 - B^2\\
    a^2 - (b + c)^2\\
    a^2 - b^2 - c^2 - 2bc
  \end{gather*}
  
\end{problem}

\begin{problem} %77
  We can change it to $(a + (b - c)) (a - (b - c))$, this is of the form $(a + B) (a - B)$.
  \begin{gather*}
    a^2 - B^2\\
    a^2 - b^2 + c^2 + 2bc
  \end{gather*}
\end{problem}

\begin{problem} %78

\end{problem}

\begin{problem} %79
  
\end{problem}



%%%%%%%%%% 25
\section{The cube of the sum formula}

%%%%%%%%%% 26
\section{The formula for $(a + b)^4$}

%%%%%%%%%% 27
\section{Formulas for $(a + b)^5$, $(a + b)^6$,... and Pascal's triangle}

%%%%%%%%%% 28
\section{Polynomials}

%%%%%%%%%% 29
\section{A digression: When are polynomials equal?}

%%%%%%%%%% 30
\section{How many monomials do we get?}

%%%%%%%%%% 31
\section{Coefficients and values}

%%%%%%%%%% 32
\section{Factoring}

%%%%%%%%%% 33
\section{Rational expressions}

%%%%%%%%%% 34
\section{Converting a rational expression into the quotient of two polynomials}

%%%%%%%%%% 35
\section{Polynomial and rational fractions in one variable}

%%%%%%%%%% 36
\section{Division of polynomials in one variable; the remainder}

%%%%%%%%%% 37
\section{The remainder when dividing by $x - a$}

%%%%%%%%%% 38
\section{Values of polynomials, and interpolation}

%%%%%%%%%% 39
\section{Arithmetic progressions}

%%%%%%%%%% 40
\section{The sum of arithmetic progression}

%%%%%%%%%% 41
\section{Geometric progressions}

%%%%%%%%%% 42
\section{The sum of a geometric progression}

%%%%%%%%%% 43
\section{Different problems about progressions}

%%%%%%%%%% 44
\section{The well-tempered clavier}

%%%%%%%%%% 45
\section{The sum of an infinite geometric progressions}

%%%%%%%%%% 46
\section{Equation}

%%%%%%%%%% 47
\section{A short glossary}

%%%%%%%%%% 48
\section{Quadratic equations}

%%%%%%%%%% 49
\section{The case $p = 0$. Square roots}

%%%%%%%%%% 50
\section{Rules for square roots}

%%%%%%%%%% 51
\section{The equation $x^2 + px + q = 0$}

%%%%%%%%%% 52
\section{Vieta's theorem}

%%%%%%%%%% 53
\section{Factoring $ax^2 + bx + c$}

%%%%%%%%%% 54
\section{A formula for $ax^2 + bx + c = 0$ (where $a \neq 0$)}

%%%%%%%%%% 55
\section{One more formula concerning quadratic equatons}

%%%%%%%%%% 56
\section{A quadratic equation becomes linear}

%%%%%%%%%% 57
\section{The graph of a quadratic polynomial}

%%%%%%%%%% 58
\section{Quadratic inequalities}

%%%%%%%%%% 59
\section{Maximum and minimum values of a quadratic polynomial}

%%%%%%%%%% 60
\section{Biquadratic equations}

%%%%%%%%%% 61
\section{Symmetric equations}

%%%%%%%%%% 62
\section{How to confuse students on an exam}

%%%%%%%%%% 63
\section{Roots}

%%%%%%%%%% 64
\section{Non-integer inequalities}

%%%%%%%%%% 65
\section{Proving inequalities}

%%%%%%%%%% 66
\section{Arithmetic and geometric means}

%%%%%%%%%% 67
\section{The geometric mean does not exceed the arithmetic mean}

%%%%%%%%%% 68
\section{Problems about maximum and minimum}

%%%%%%%%%% 69
\section{Geometric illustrations}

%%%%%%%%%% 70
\section{The arithmetic and geometric means of several numbers}

%%%%%%%%%% 71
\section{The quadratic mean}

%%%%%%%%%% 72
\section{The harmonic mean}

\end{document}